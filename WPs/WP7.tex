\addtocounter{wpno}{1}
\begin{Workpackage}{\thewpno}
\wplabel{wp:eval}
\WPTitle{\wpname{\thewpno}}
\WPStart{Month 1}
\WPParticipant {GOLEM}{36}
\WPParticipant {PRL}{14}
\WPParticipant {JMOIC}{10}
\WPParticipant {CODEPLAY}{9}
\WPParticipant {AGH}{4}
\WPParticipant {INRIA}{4}
\WPParticipant{USTAN}{3}
\WPParticipant {IBM}{3}

\WPParticipant{SCCH}{1}

\begin{WPObjectives}
The objectives of \theWP{} are to:
\begin{compactitem}

\item Determine overall requirements, metrics, analytics and benchmarks for the \TheProject{} project;
\item Identify and implement smaller-scale, technically-focused benchmarks to test particular technologies developed by \TheProject{};
\item Develop new and adapt the existing large-scale industrial use cases from the automotive, AI and IoT domains;
\item Evaluate the \TheProject{} technologies and techniques on the benchmarks and use cases and demonstrate the benefits in terms of increases in productivity, reliability, security, safety, performance;
\item Provide a technical roadmap for long-term development and use of the \TheProject{} tools, technologies and methodologies in different domains.

\end{compactitem}
\end{WPObjectives}

\begin{WPDescription}
The purpose of this work package is to evaluate the \TheProject{} technology on a set of realistic use-cases taken from the automotive, AI and IoT domains. 
We will define the requirements that the technologies of the project will need to satisfy, based on the identified use cases, and also identify metrics for the evaluation of the \TheProject{} tools and techniques (T7.1).  We will develop focused benchmarks to test specific parts of the overall \TheProject{} technology (T7.2), adapting, where feasible, existing use cases from our industrial partners to apply the \TheProject{} technology, and developing new use cases.
We will use these to demonstrate the suitability of our techniques both 
for porting existing code bases and for developing new code bases (T7.3), 
evaluating  (T7.4) the benefits that our techniques bring in terms of the metrics that will be identified in T7.1, and providing a roadmap for addressing the issues that will arise for the use of the \TheProject{} technology on future-generation data-intensive applications and distributed systems (T7.5).
%The work related to use cases will proceed in three phases: (1) Initial porting/development of the use-case to adapt them for the use of the \TheProject{} technology. (2) Adapting to the smaller-scale distributed systems, based on the evaluation of the initial versions. (3) Improving and validating the use case for deployment on large-scale distributed heterogeneous systems. 

\end{WPDescription}

	

\begin{Task}
\TaskTitle{Requirements Analysis and Metrics}
\TaskParticipant{GOLEM}{2}
\TaskParticipant{INRIA}{1}
\TaskParticipant{CODEPLAY}{1}
\TaskParticipant{JMOIC}{1}
\TaskParticipant{PRL}{0.5}

%\TaskParticipant{SA}{0.5}

\TaskStart{1}
\TaskEnd{24}
\TaskResults{
\ref{del:req1}
\ref{del:req2}
\ref{del:req3}
}
\TaskHeader{}
\tasklabel{task:reqts}

\theTask{} will determine the requirements of the project as a whole. It will identify the main technical challenges that must be addressed by the \TheProject{} approach, in terms of scalability, software development tools, producing qualitative and quantitative technical requirements for the project as a whole. These technical requirements will initially be driven by the planned use case applications. As a part of this, we will also identify the metrics that will be used in the evaluation of our technologies to demonstrate the benefits of the \TheProject{} technologies in term of the use cases that will be provided.
 Work  on \theTask{} will proceed in three phases: the initial requirements analysis will be reported in Deliverable~\ref{del:req1}; these requirements will subsequently be updated in the light of experience with the benchmarks and use cases of \ref{task:benchmarks} and \ref{task:usecases},
and reported in Deliverables~\ref{del:req2} and~\ref{del:req3}. 
%
The identified requirements will be fed into each of WP2-WP6, as well as into later phases of this work package.
They will thus form  a foundation for the design and implementation of the associated methods, tools and use cases. 
\end{Task}

\begin{Task}
\TaskTitle{Benchmarks}
% \TaskParticipant{SCCH}{3}
\TaskParticipant{GOLEM}{3}
\TaskParticipant{AGH}{2}
\TaskParticipant{INRIA}{1}

\TaskStart{1}
\TaskEnd{31}
\TaskResults{
\ref{del:eval1}
\ref{del:eval2}
\ref{del:eval3}
}
\TaskHeader{}
\tasklabel{task:benchmarks}

Based on the requirements that will be determined in \ref{task:reqts}, \theTask{} will develop industrially-relevant benchmarks that will enable us to test the key parts of the \TheProject{}. The set of benchmark that will be 
developed will also contain specific auxiliary tools related to use cases (e.g.~specific traffic and biological organisms simulation for the Smart City use case) that will support development and evaluation of the 
full-scale use cases in T7.3. The benchmarks will cover the full range of technical issues, with a view to enabling us to focus on the evaluation of the limits of the \TheProject{} technologies, to provide rapid feedback on their effectiveness, and to enable rapid improvements to be made to those technologies. In particular, the benchmarks will document and assess both the recurring and non-recurring efforts that are required to apply the results of the other technical workpackages to specific application instances. 
An example of a non-recurring effort might be learning how to use a specific tool, or to understand a new pattern or technique. This task will proceed in three phases, developing increasingly complex benchmarks targeting shared-memory
heterogeneous systems (reported in ~\ref{del:eval1}), small-scale distributed systems (reported in~\ref{del:eval2}) and large-scale distributed systems (reported in~\ref{del:eval3}).
%The benchmarks will be elaborated iteratively. They will be updated and expanded the benchmarks based on lessons learned and accounting for the availability of services/functionality from the technical work packages. 
%In order to prepare the designed tools for industrial use cases  \ref{task:usecases}, that is mainly connected with Smart City monitoring, selected difficult problems of a similar scale will be tackled (e.g. traffic simulation, metaheuristic computing, biological organism simulation---all conducted on HPC-grade supercomputers). Such research will show the applicability of the proposed software for supporting robust and scalable computing and simulation that will be necessary in order to cover the requirements of Smart City monitoring, analysis, prediction etc.
\end{Task}

\begin{Task}
\TaskTitle{Industrial Use Cases}
% \TaskParticipant{SCCH}{20}
% \TaskParticipant{SA}{1}
% \TaskParticipant{AITIA}{0}

\TaskParticipant{GOLEM}{28}
\TaskParticipant{PRL}{11.5}
\TaskParticipant{JMOIC}{8}
\TaskParticipant{CODEPLAY}{6}
\TaskParticipant{IBM}{1.5}
\TaskParticipant{SCCH}{1}
\TaskStart{1}
\TaskEnd{34}
\TaskResults{\ref{del:eval1}, \ref{del:eval2}, \ref{del:eval3}.}
\TaskHeader{}
\tasklabel{task:usecases}

\theTask{} will develop the use case applications that will be used to evaluate the \TheProject{} technology in a realistic setting. The work will proceed in three phases. 
(1) Initial assessment of the refactoring tools for porting/development existing Smart City Monitor for \TheProject{} technology to be reported in~\ref{del:eval1}. 
(2) The key system components refactoring to realise the smaller-scale distributed system and evaluate the initial versions from the first phase to be reported in~\ref{del:eval2}. 
(3) In the final phase, we will further improve and validate and demonstrate the use case for deployment on large-scale distributed heterogeneous system in the Jelgava city and reported in~\ref{del:eval3}. 

\textbf{Smart City Monitor (\GOLEMshort)}. The use case for Smart City domain is undertaken by \GOLEMshort and demonstrated in collaboration with the \textbf{Jelgava municipality} in a real world environment. \GOLEMshort will use its Smart City Monitor system components as a base for experiments and benchmarking. 
While some parts of the system are already run in parallel, the major part of the 
internal modules only use one thread. These modules will be prepared for benchmarking in the first phase of the task. During the second phase, when elementary \TheProject results are available, \GOLEMshort will apply the \TheProject results to the Smart City use case, implement and carry out the benchmarking measurements. This will be iterated one more time within the third phase of the task, when the full project results are available and the system is installed in the demonstration city testbed. 
At this stage three kinds of benchmarking can be carried out: i) existing Smart City Monitor execution; 
ii) applying the \TheProject{} results to the major components; and, 
iii) benchmarking the software version that has been parallelised by experienced 
human programmers. The main use case components requiring principally new approaches to parallelisation are as follows:


\begin{compactitem}
\item	Processing of large number of data from sensors/IoT: an application in an 
infinite loop connects simultaneously to a multitude of data acquisition systems, 
obtains data from them for a specified period of time, converts this data from 
different formats to a single one and makes a common packet and sends to the 
Smart City Monitor engine.
\item	Parallel calculation of indicators: when new data arrives, a large number of 
parallel processes of indicator calculation start automatically accordingly to 
dependency chains in particular CPS model. Currently there is a programmer enabled 
switch ``parallel / serial calculation of chains'' that require in-depth understanding 
of very complex relations between large number of software and hardware components 
(number of cores, threads, etc). The sequential calculation results in significant 
increase of total processing time 
 for a chain and inefficiencies of available hardware 
and energy resources use. \TheProject tools can be effectively applied to an existing 
system implemented in the pilot city and linked to real world data streams for 
evaluation and testing of reliable parallel calculation algorithms for the indicators. 
This shall provide clear performance benchmarks for comparison between the results 
that are obtained for execution of normalised samples of sequential and parallel chains.
\end{compactitem}
\taskbreak
\begin{compactitem}
\item	The computational complexity and relevant processing time for the calculation 
of object statuses in the hierarchical CPS model quickly grow with its size. 
Novel, highly specialised methods of 
parallel processing, prediction and 
combinatorial analysis will be used to address the challenge of supporting real time intelligent analysis of system sustainability.    
\item Searching for errors and opportunities of improving performance, 
security of executable code in already existing parallel processes implemented in 
the Smart City Monitor digital transformation engine with support of \TheProject 
tools.
\end{compactitem}

\textbf{TensorC (\CODEPLAYshort{})}
This use case involves using \TheProject tools to generate optimised implementations of convolutional neural network models for specific hardware platforms. 
%
The performance of a particular convolutional neural network is dependent on both the properties of the model itself (such as the size and dimensionality of tensors) and upon properties of the execution environment/hardware (such as memory hierarchies, core counts etc.). Consequently, producing implementations of neural network operators that offer predictable, portable performance across a wide range of tensors sizes and hardware platforms is challenging.
%
\begin{compactitem}
\item The implementation of high-level descriptions of convolutional neural network models (i.e. AlexNet, ResNet50)
\item The specification of machine models in a form compatible with the optimization interface described in WP5.
\item The use of the optimization framework (also described in WP5) to guide the generation of optimised parallel network implementations, based on network model and hardware-specific properties.
\end{compactitem}

This work will be subdivided into three phases: generating high-level representations of neural network and machine models in phase one; the application of early project outputs to parallel code generation and optimization in phase two; and a further iteration utilizing final project outputs, coupled with benchmarking against networks models implemented in existing machine learning frameworks.

\textbf{Automotive Industry - ADAS (\PRLshort{})}
This use case for Automotive Industry is undertaken by \PRshort and will use the models and automatic code that will be provided by Magna Electronics inc. (USA). The use case will include the following stages:
\begin{compactitem}
\item Initial adaptation of the models and the code from the Magna Electronics Inc. to reproduce the generation of the code;
\item Initial static analysis of the code and generation of the non-compliance diagnostics;
\item Categorisation of the non-compliance diagnostic and selection of the diagnostics that can be potentially fixed (not related to the code generator itself);
\item Modification of the building blocks libraries to fix non-compliance followed by static analysis to show that non-compliance diagnostics have been fixed;
\item Analysis of the possible places for parallelisation in the code;
\item Modification of the building blocks links to involve parallelisation patterns libraries and select different patterns for parallelisation;
\item Generate the parallelised code;
\item Work with Magna Electronics to functionally test the newly generated code and to demonstrate the correctness of new code (this may involve multiple iterations to fix the problems that are found).

\end{compactitem}
As a result of this use case transformation we should have demonstrated a way to fix non-compliance and to introduce parallelisation into real automotive software code that will
go into production.
\end{Task}

\begin{Task}
\TaskTitle{Evaluation}
%\TaskParticipant{SCCH}{5}
\TaskParticipant{SA}{2}
\TaskParticipant{PRL}{2}
\TaskParticipant{INRIA}{2}
\TaskParticipant{GOLEM}{2}
\TaskParticipant{CODEPLAY}{2}
\TaskParticipant{AGH}{2}
\TaskParticipant{IBM}{1.5}
\TaskParticipant{JMOIC}{0.5}

\TaskStart{9}
\TaskEnd{36}
\TaskResults{
\ref{del:eval1}
\ref{del:eval2}
\ref{del:eval3}
}
\TaskHeader{}
\tasklabel{task:evaluation}

\theTask{} will evaluate the \TheProject{} tools and technologies
against the overall project requirements, success criteria and metrics that were identified
in~\ref{task:reqts},
using the
representative use cases that were developed
in~\ref{task:usecases},
and the benchmarks of~\ref{task:benchmarks}.
%
We aim to especially evaluate those aspects that contribute to the creation of impact
by the industrial partners, and to its applicability across application domains. The results
of this task will feed into WP2-WP6 as appropriate, as well as into tasks T7.2 and T7.3.
This will improve subsequent versions of the benchmarks and use cases based on the the evaluation of their
previous versions. The task will proceed in three phases, similar to T7.2 and T7.3, evaluating, respectively,
the initial versions of the use cases (reported in~\ref{del:eval1}), the intermediate versions (reported in~\ref{del:eval2}) and the final versions (reported in~\ref{del:eval3}).

\end{Task}


\begin{Task}
\TaskTitle{Roadmapping}
% \TaskParticipant{SCCH}{1}
% \TaskParticipant{SA}{0.5}

\TaskParticipant{SA}{1}
\TaskParticipant{PRL}{1}
\TaskParticipant{INRIA}{1}
\TaskParticipant{GOLEM}{1}
\TaskParticipant{AGH}{1}
\TaskParticipant{IBM}{0.5}
\TaskParticipant{JMOIC}{0.5}

\TaskStart{30}
\TaskEnd{36}
\TaskResults{}

\TaskHeader{}
\tasklabel{task:roadmapping}

\theTask{} will undertake roadmapping activities that are aimed at situating future 
\TheProject{} technologies in the context of emerging technical developments 
in software engineering methods for complex distributed systems, highlighting potential improvements and 
enhancements to the \TheProject{} tools and techniques, and broadening the long-term impact of the \TheProject{} project. 
This road map will consider
%  new and emerging developments in multi-core hardware, software, software engineerings, and development environments.
% they relate to the design, construction and use of highly paralesystems.
% It will consider
all issues relevant to ICT-16-2017, including raising the level of abstraction for programming models,
adaptivity to different platforms and addressing the non-functional requirements of the applications.
It will show how the \TheProject{} technologies
impact each of these areas, highlight possible applications, and indicate the further work that needs to be done
to bring the \TheProject{} technologies to market.
The result of this task will be a report describing a technical road map,
highlighting the issues for future research and use of \TheProject{}
technology (\ref{del:roadmapping}).

\end{Task}

\vspace{-10pt}
\begin{WPDeliverables}
\begin{compactitem}
\item \ref{del:req1} (Month 3): Report on Initial Requirements Analysis and Metrics.
\item \ref{del:eval1} (Month 11): Report on Initial Implementation of Benchmarks and Use Cases and Evaluation.
\item \ref{del:req2} (Month 12): Report on Updated Requirements Analysis and Metrics.
\item \ref{del:eval2} (Month 22): Report on Intermediate Implementation of Benchmarks and Use Cases and Evaluation.
\item \ref{del:req3} (Month 23): Report on Updated Requirement Analysis and Metrics.
\item \ref{del:eval3} (Month 36): Report on Final Implementation of Benchmarks and Use Cases and Evaluation.
\item \ref{del:roadmapping} (Month 36): Report on Technical Roadmap.
\end{compactitem}
\end{WPDeliverables}

\end{Workpackage}
