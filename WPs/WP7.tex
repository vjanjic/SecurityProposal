\addtocounter{wpno}{1}
\begin{Workpackage}{\thewpno}
\wplabel{wp:usecases}
\WPTitle{\wpname{\thewpno}}
\WPStart{Month 1}
\WPParticipant{SOPRA}{1}
\WPParticipant{COGNI}{1}


\begin{WPObjectives}
The objectives of \theWP{} are to:
\begin{compactitem}
\item	Determine the overall requirements for tools and use cases that will be developed in \TheProject{} and success metrics for demonstrating that the project achieves the determined objectives;
\item	Identify and implement real-world use cases from air traffic, healthcare and baking domains that will demonstrate \TheProject{} tools and techniques;
\item	Evaluate the \TheProject{} technologies and techniques on the developed use cases and against the expected results;
\item	Provide a technical roadmap for long-term development and use of the \TheProject tools and technologies.
\end{compactitem}
\end{WPObjectives}

\begin{WPDescription}
The purpose of this work package is to evaluate the \TheProject technology on real-world use cases from the air traffic, healthcare and banking domains. In all these domains, ensuring security of data analytics, privacy of data access, as well as ensuring the quality of data is of utmost importance. In this workpackage, we will i) define the requirements and success metrics that the \TheProject{} technologies and use cases will need to satisfy in order to achieve the desired results, as set out in project objectives, and to achieve the expected impacts (T7.1); ii) adapt the existing and develop new use cases that will demonstrate that the \TheProject{} tools and techniques can support development of secure and privacy preserving distributed data analytics application for end users (T7.2); iii)  evaluate the effectiveness and benefits that our techniques will bring in terms of the success metrics tidentified in T7.1 (T7.3); and, iv) provide a roadmap for addressing the issues that will arise from the usage of the \TheProject{} technology for future-generation
secure distributed systems (T7.4).
\end{WPDescription}

\begin{Task}
%\TaskTitle{Parallel Computation and Control for High-Level Modelling Languages} % for Aerospace and Automotive Industries}
\TaskTitle{Requirements and Success Metrics}
\TaskParticipant{UOD}{1}

\TaskStart{1}
\TaskEnd{27}
\TaskResults{%
%\ref{del:model1}
}
\TaskHeader{}
\tasklabel{task:requirements}
\theTask{} will determine the requirements of the \TheProject{} project as a whole.  It will identify the main technical and functional challenges that must be addressed by the \TheProject{} approach. The identified  requirements will be fed into each of the technical work packages, forming a foundation for the design and implementation of the associated methods and tools. Complementing the success criteria for the overall expected impacts (Section~\ref{sect:impacts}), application-specific success criteria for each already existing use case, plus the associated measurement criteria, will be defined as part of the task deliverables. This will be done by determining Key Performance Indicators, KPIs, creating clear definitions, undertaking baseline measurement of existing use cases in the first phase of the task, resulting in \ref{del:req1} (M3). This will later be refined in the second phase to take into account feedback from the initial implementations of the use cases, resulting in \ref{del:req2} (M14). This will allow a precise measure of success to be defined for each use case, without over-generalisation.
\end{Task}

\begin{Task}
  \TaskTitle{Use Cases}
\TaskStart{1}
\TaskEnd{27}
\TaskResults{%
%\ref{del:model1}
}
\TaskHeader{}
\tasklabel{task:usecases}
theTask{} will develop the use case applications that will be used to evaluate the \TheProject{} technology in a realistic setting. They will be used to test our technology as a whole in realistic conditions, using both synthetic data produced using data fabrication, as well as realistic data coming from the use cases.

\textbf{Air Traffic Control.} 

\textbf{Banking.} 

\textbf{Healthcare.} 
\end{Task}

\begin{Task}
  \TaskTitle{Evaluation}
\TaskStart{1}
\TaskEnd{27}
\TaskResults{%
%\ref{del:model1}
}
\TaskHeader{}
\tasklabel{task:evaluation}
\theTask{} will evaluate the \TheProject{} tools and technologies against the overall project requirements and success criteria that were identified in T7.1, using the representative metrics and use cases that were developed in T7.2. We aim to especially evaluate those aspects that contribute to the creation of impact by the industrial partners from the selected domains. We will evaluate both how much the security and privacy of the data and data analytics models is improved using the \TheProject{} techniques, using simulated cyber-attacks for testing, and how easy it is for end users to use the \TheProject{} methodology for developing secure distributed data analytics applications. The lessons learned from this task will provide feedback into different technical work packages, steering the development of \TheProject{} technologies in the subsequent phases of the project, as well as into the roadmapping task (T7.5). The task will proceed in three phases, evaluating initial (phase 1), refined (phase 2) and final use cases (phase 3). Results will be reported in \ref{del:eval1} (M12), \ref{del:eval2} (M25) and \ref{del:eval3} (M36), respectively. 
\end{Task}

\begin{Task}
  \TaskTitle{Roadmapping}
  \TaskParticipant{UOD}{1}
  \TaskResults{
  \ref{del:roadmap}.
  }
  \TaskHeader{}
  \tasklabel{task:roadmapping}
  \theTask{} will use the outcomes of T7.3 to undertake roadmapping activities that are aimed at situating future \TheProject{} technologies in the context of emerging technical developments in development of secure distributed data analytics applications and in broadening the long-term impact of the \TheProject{} project. This roadmap will consider new and emerging developments in security, deep-learning, data lakes, blockchains, cloud computing etc. as they relate to the secure analytics of big data. It will highlight possible future applications of the \TheProject{} technology, and indicate the further work that needs to be done to bring the \TheProject{} technologies to market. The result of this task will be a report describing the technical roadmap, highlighting the issues for future research and use of \TheProject{} technology, also being used as input for T8.2 (Exploitation and Use). 
  \end{Task}
  
\begin{WPDeliverables}
  \begin{compactitem}
    \item XX
%\item \ref{del:model1} (Month 10): Report on Initial Block-Diagram Modelling, Patterns and Code Synthesis
\end{compactitem}
\end{WPDeliverables}
\end{Workpackage}
