\addtocounter{wpno}{1}
\begin{Workpackage}{\thewpno}
\wplabel{wp:usecases}
\WPTitle{\wpname{\thewpno}}
\WPStart{Month 1}
\WPParticipant{SOPRA}{37}
\WPParticipant{FRQ}{27}
\WPParticipant{IBM}{16}
\WPParticipant{SA}{3}
\WPParticipant{SCCH}{3}
\WPParticipant{COGNI}{3}
\WPParticipant{UCM}{3}
\WPParticipant{YAG}{3}
\WPParticipant{UOD}{3}



\begin{WPObjectives}
The objectives of \theWP{} are to:
\begin{compactitem}
\item	Determine the overall requirements for tools and use cases that will be developed in \TheProject{}.
\item Determine the success metrics for demonstrating that the \TheProject{} project achieves the determined objectives.
\item	Identify and implement real-world use cases from air traffic management, healthcare and banking domains that will demonstrate \TheProject{} tools and techniques.
\item	Evaluate the \TheProject{} technologies and techniques on the developed use cases and against the expected results.
\item	Provide a technical roadmap for long-term development and use of the \TheProject tools and technologies.
\end{compactitem}
\end{WPObjectives}

\begin{WPDescription}
The purpose of this work package is to evaluate the \TheProject technology on real-world use cases from the \emph{air traffic}, \emph{healthcare} and \emph{banking} domains. In all these domains, ensuring security of data analytics, privacy of data access, as well as ensuring the quality of data, is of utmost importance. We will therefore i) define the requirements and success metrics that the \TheProject{} technologies and use cases will need to satisfy in order to achieve the desired results and expected impacts, as set out in project objectives (\ref{task:requirements}); ii) adapt the \TheProject{} \emph{data fabrication} technique to the particular use cases considered, therefore providing the use cases with large volumes of synthetic ---but realistic--- data on which it is safe to develop and test the \TheProject{} techniques (\ref{task:datafab}); iii) adapt the existing, and develop new, use cases that will demonstrate that the \TheProject{} tools and techniques can support the development of secure and privacy preserving distributed data analytics applications for end users (\ref{task:usecases}); and, iv)  evaluate the effectiveness and benefits that our techniques will bring in terms of the success metrics identified in~\ref{task:requirements}, and produce a roadmap for addressing the issues that will arise from the usage of the \TheProject{} technology for future-generation secure big data analytics systems (\ref{task:evaluation}). The adaptation of data fabrication techniques, use cases and evaluation will proceed in three phases. In the \emph{first} phase, we will develop an initial version of use cases and fabricated data, and evaluate these on private clouds (reported in~\ref{del:eval1}). In the \emph{second} phase, we will extend and evaluate these use cases, and produce additional data for deployment on public clouds (reported in~\ref{del:eval2}). In the \emph{third} and final phase, the use cases and fabricated data will be adapted and evaluated for deployment on hybrid clouds (reported in~\ref{del:eval3}).
\end{WPDescription}

\begin{Task}
%\TaskTitle{Parallel Computation and Control for High-Level Modelling Languages} % for Aerospace and Automotive Industries}
\TaskTitle{Requirements and Success Metrics}
\TaskParticipant{SOPRA}{1}
\TaskParticipant{SA}{1}
\TaskParticipant{IBM}{1}
\TaskParticipant{SCCH}{1}
\TaskParticipant{COGNI}{1}
\TaskParticipant{UCM}{1}
\TaskParticipant{FRQ}{1}
\TaskParticipant{YAG}{1}
\TaskParticipant{UOD}{1}

\TaskStart{1}
\TaskEnd{15}
\TaskResults{%
\ref{del:req1},
\ref{del:req2}
}
\TaskHeader{}
\tasklabel{task:requirements}
In \theTask{}, we will determine the requirements of the \TheProject{} project as a whole, identifying the main technical challenges that must be addressed by the \TheProject{} approach. We will also identify big data analytics databases, filesystems, frameworks and libraries that will be targeted in the project. The identified  requirements will be fed into each of the technical work packages, forming a foundation for the design and implementation of the associated methods and tools. We will also determine KPIs for achieving the overall expected impacts (Section~\ref{knowledgeprotection}), application-specific success criteria for each already existing use case, plus the associated measurement criteria. 
%As a part of this task, we will also devise a certain level of business oriented semantics for multilevel security requirements allocation, detecting potential sensitive business data leaks and planning for building the bridge between business and programmatic views. 
This task will proceed in two phases. In the \emph{first} phase, we will determine the KPIs, create clear definitions, and undertake the baseline measurement of existing use cases (reported in \ref{del:req1}). This will later be refined in the \emph{second} phase to take into account feedback from the initial implementations of the use cases, resulting in \ref{del:req2}. This will allow a precise measure of success to be defined for each use case, without over-generalisation. All of the project partners will contribute to this task.
\end{Task}

\begin{Task}
%\TaskTitle{Parallel Computation and Control for High-Level Modelling Languages} % for Aerospace and Automotive Industries}
\TaskTitle{Data Fabrication for Use Cases}
\TaskParticipant{IBM}{13}
\TaskParticipant{SOPRA}{4}
\TaskParticipant{FRQ}{4}
\TaskStart{3}
\TaskEnd{32}
\TaskResults{%
\ref{del:eval1},
\ref{del:eval2},
\ref{del:eval3}
%\ref{del:model1}
}
\TaskHeader{}
\tasklabel{task:datafab}
In \theTask{}, we will adapt the \IBMshort{} Data Fabrication Platform in order to produce synthetic, but realistic, datasets for the use cases that will be considered. The use case partners will analyse the data from their use cases, and derive, together with \IBMshort{}, the rules and relationships required to generate synthetic data. These rules will then be used to produce arbitrary volumes of synthetic data, as required for development and evaluation of the use cases. This will have a number of benefits, allowing evaluation of the \TheProject{} technologies from their early stages on realistic use cases without the privacy issues arising from the use of real data. It will also allow us to stress-test the system, as we will be able to generate arbitrarily large volumes of data and test the system on it. \IBMshort{} will lead the task, providing expertise in using the IBM Data Fabrication Platform, which will be used by \FRQshort{} and \SOPRAshort{} for their use cases.
\end{Task}

\begin{Task}
\TaskTitle{Use Cases}
\TaskParticipant{SOPRA}{28}
\TaskParticipant{FRQ}{19}
\TaskParticipant{SA}{1}
\TaskParticipant{UOD}{1}
\TaskParticipant{SCCH}{1}
\TaskParticipant{COGNI}{1}
\TaskParticipant{UCM}{1}
\TaskParticipant{YAG}{1}
\TaskParticipant{IBM}{1}
\TaskStart{3}
\TaskEnd{36}
\TaskResults{%
\ref{del:eval1},
\ref{del:eval2},
\ref{del:eval3}
}
\TaskHeader{}
\tasklabel{task:usecases}
In \theTask{}, we will develop the use case applications that will be used to evaluate the \TheProject{} technology in a realistic setting. The use cases will be used to test our technology as a whole, in realistic conditions, using both synthetic data produced using data fabrication, as well as realistic data (where feasible). 
%We will also support a certain level of business oriented semantics for multilevel security requirements allocation, detecting potential sensitive business data leaks and start building the bridge between business views (for instance Air Traffic Management data vs Unmanned Traffic Management data if it makes sense that they have different security requirements) and programmatic views.

\textbf{Air Traffic Management (\FRQshort{}).} 
%% VJ: Move this to the background section
%To address the use case described in section \ref{sec:atm} the Frequentis MosaiX SWIM aviation integration platform is used which provides tools and applications that ANSPs need to overcome these challenges and to facilitate the exchange of information between all industry stakeholders for better-informed decision making and shared situational awareness. It is specifically designed to assist the Aviation industry with its migration to SWIM and to facilitate the exchange of information between all industry stakeholders for better informed decision making and the creation of new applications and services. This reduces vendor lock-in by allowing customers to substitute components and incorporate new technologies in the future. The outcome of this project will ensure that one of the key benefits, is ensured in a comprehensive security solution.
In this use case, we will extend the Frequentis MosaiX SWIM aviation integration platform (described in Section~\ref{sec:atm}), to ensure a comprehensive security solution by detecting and repairing vulnerabilities in the platform. In addition, the customisable metrics and dashboards will be extended by the machine learning outcomes produced within this project. MosaiX SWIM represents a fundamental shift from today’s monolithic solutions, where adopting new technologies requires re-engineering the whole system. We will focus on the scenario of \emph{drone deconfliction}. To perform strategic deconfliction, the flight path of one of the conflicting drones needs to be altered at the leg between two way points at which the conflict takes place. One drone should ascend, the other drone should descend. A conflict occurs if two or more drones flying along their planned flight paths come into close proximity with each other. To be able to perform 4D conflict detection we will need to know the drone flight path way points in 3D space and the planned time of arrival for each way point. For tactical conflict resolution the shortest vector from an approaching other drone shall be calculated and executed immediately. However, such capability is expected to be built into the drone itself, not requiring any external input or network. \emph{We want to prove that we can secure a distribute coding base ranging over several different domains without compromising the overall ecosystem.}



%\begin{itemize}
%    \item Languages
%        \begin{itemize}
%            \item Java
%            \item JavaScript
%			\item Python
%        \end{itemize}
%    \item Platform and Technology
%        \begin{itemize}
%            \item Elasticsearch
%            \item Azure Data Lake Analytics
%            \item Azure AI
%        \end{itemize}
%    \item Storage
%        \begin{itemize}
%            \item Azure Data Lake
%			\item Azure SQL Database
%            \item My-SQL
%        \end{itemize}
%\end{itemize}

\textbf{Digital Banking (\SOPRAshort{}).} \emph{Digital banking} has transformed
%is the digitisation (or transferring online) of all 
traditional banking activities and processing services, which were historically only available to customers when physically inside a bank branch, to online services. Customers can now perform ``touch-less'' transactions for
money deposits, withdrawals and fund transfers, directly executed on mobile devices. The transaction touch-point is now migrated to customers' mobile phones and tablets. Providers that traditionally would have had a fixed address (e.g. place of business) are now mobile, as  mobile devices can now be converted into card terminals. The traditional trusted end-point security model is now changed into a \emph{zero trust architecture}. The addition of Internet-of-Things transactions (e.g. train access gates, unmanned shops and push button ordering devices) means the stable environment of banking is now open to the revolution in the information age. \emph{We want to secure each component to ensure the specific step in the digital banking process is secure and trusted. The core issue is that the different processing steps are no longer solely under the control of traditional banking.}

%\begin{itemize}
%    \item Languages
%        \begin{itemize}
%            \item C++
%            \item Python using C++ libraries
%        \end{itemize}
%    \item Platform and Technology
%        \begin{itemize}
%            \item Dask Cluster
%            \item Azure Data Factory
%            \item Azure Data Lake Analytics
%            \item Azure Synapse Analytics
%            \item Azure Databricks
%            \item Azure AI
%            \item AWS Fargate
%            \item AWS Athena
%        \end{itemize}
%    \item Storage
%        \begin{itemize}
%            \item Samba File Storage
%            \item Azure Data Lake
%            \item Azure Databricks
%            \item AWS Lake Formation
%            \item AWS S3
%            \item AWS Redshift
%        \end{itemize}
%\end{itemize}

\textbf{Healthcare (\SOPRAshort{}).} 
%ORGINAL VERsION
Medical technology has advanced rapidly in recent years %within the last decade, 
with many devices (e.g. pacemakers, insulin pumps) becoming commonplace in medical treatment. These small devices  provide a remote treatment service: they monitor the patients' vital signs, recording and even transmitting this data to hospitals and other medical facilities for diagnosis, where their parameters can be set remotely by clinicians.
%, so that clinicians can make an informed decision in making a treatment plan for the patient; 
%and they provide a quality of care to the patient that they would not otherwise be able to receive. 
In addition, remote consultation, reinforced by  COVID-19, is becoming increasingly common, as diagnosis and treatment can be delivered through digital services, meaning that health-services can be provided to patients who might not be able to gain physical access to a surgery, such as the elderly, people with mobility issues, people who live in remote areas, etc. Moreover, these services can be provided by specialists to local medical practices, hospitals and care homes. This results in a myriad of problems with communicating medical data, from small embedded devices to digital consultations. 
%However, these devices are sometimes connected via the Internet of Things, meaning they are open to new security vulnerabilities, where the transmission of confidential medical information may be exposed.
%devices here mean the little ones like pacemakers and insulin pumps and ipads and laptops used for consultations <- do you think it needs more clarifying? how is it reading? terrible? nice - you are doing a great job!
%I think we need to talk about the "so what" bit now i.e. the problem with the security vulnerability or how it can be addressed
\emph{We want to increase the trust in the widespread use of such devices and remote consultations, and hence
to ensure that they are safe and secure, and have been rigorously scrutinised for guaranteeing that they preserve 
the privacy for the patients at all times.}
% is this what you mean> yay! yes exactly. Yep is good - anything more to add?
% be back in a second
% OK
%back
% the beauty of this is that it actually deals with inequality! something like providing the best healtcare for all
% the scottish government is actually starying to create mechanisms to deal with inequalities - so some people in poorer areas with no devices can go to a nearby facility to receive a consultation with a specialist in edinburgh for instance
% you can talk about people that live in remote areas? yes, or that are imobile, yes

%now connected via the Internet of Things and open to new security vulnerabilities.
%remote treatment (setting parameters remotely for instance) - how do you guarantee that only the medic is doing it and he is doing it right? the convenience of these devices (for the health team and the patient) also created new
%vulnerabilities that can be exploited

%Something else that the current pandemic has reinforced, is the need to support remote consultation 
%through digital services. So we can say something about small embedded devices on the one side, to larger problems of remote consultations where private information may be passed and has to be protected.
%you can add that these services are ways of connecting people with different medical staff: health practitioners, hospitals, care homes. All have to be able to use these systems and trust them.

%Carry on - you are doing good :D

% sorry - go ahead
%sorry - go ahead copycat :D
% I'm trying to say something like:
% world is getting more modernised and connected
% which also means that this means we need a stronger medical thingy 
% to help... e.g. covid... 
% yes yes
% you can think about this. these days a pacer (what you do you call them) can be accessed remotely and therefore introducing security issues... there are a lot of devices - more and more - that can be tampered with
% another is an insulin pump
% aha, perfect

%Global healthcare is becoming a major impact on the global environment, relating to economical impact and ease of travel between countries.  
%The rise in global healthcare costs is causing a disruption to the traditional healthcare model, as patient demographics is now evolving to a model of medical cover in specific home countries. This evolving consumer expectation is generating new market verticals. Complex health and technology ecosystems are forcing new data sharing agreements and requirements onto the global health care providers. People now invest in value-add care, advanced digital technologies and innovative care delivery requirements. At the core of this is a massive set of ever-increasing data interchange and processing requirements. Global healthcare needs a \emph{zero trust architecture} to support these uncertainties in security, and build a smart health ecosystem that can meet the current and future healthcare model. \emph{We want to show that by securing the processing within the core healthcare ecosystem, we create a secure internal architecture.}
%NEW VERSION - may come with ethics issues though
%There is a currently an increased interest in personalised medicine, and how to be able to improve 
%treatments to suit individual patients, their needs and preferences better. 
%Many people currently suffer from one, two or more chronic conditions at some point in their lives, and this
%is the same all over Europe. However, treatments could be improved if better prediction models were available.

%\begin{itemize}
%    \item Languages
%        \begin{itemize}
%            \item C++
%            \item Python using C++ libraries
%        \end{itemize}
%    \item Platform and Technology
%        \begin{itemize}
%            \item Dask Cluster
%            \item Azure Data Factory
%            \item Azure Data Lake Analytics
%            \item Azure Synapse Analytics
%            \item Azure Databricks
%            \item Azure AI
%            \item Azure Bot Service
%            \item Azure AI
%            \item AWS Fargate
%            \item AWS Athena
%            \item Google Cloud Dataproc
%            \item Google Cloud BigQuery
%        \end{itemize}
%    \item Storage
%        \begin{itemize}
%            \item Samba File Storage
%            \item Azure Data Lake
%            \item Azure Databricks
%            \item AWS Lake Formation
%            \item AWS S3
%            \item AWS Redshift
%            \item Google Cloud Data Lake
%        \end{itemize}
%\end{itemize}

%\SOPRAshort{} will lead this task and bring in two use cases whereas \FRQshort{} will provide another one. The other partners will be involved in discussions on these use cases to gain a better understanding of their complexities and problems, and gain insights directly transferable to the tasks in other work packages.
%This is a extension of the work we are completing for Europe via the SERUMS project (\url{https://www.serums-h2020.org/})

%% investigate (https://cordis.europa.eu/project/id/883335)
\end{Task}

\begin{Task}
  \TaskTitle{Evaluation and Roadmapping}
  \TaskParticipant{SOPRA}{4}
  \TaskParticipant{FRQ}{3}
  \TaskParticipant{SA}{1}
  \TaskParticipant{UOD}{1}
  \TaskParticipant{IBM}{1}
  \TaskParticipant{SCCH}{1}
  \TaskParticipant{COGNI}{1}
  \TaskParticipant{UCM}{1}
  \TaskParticipant{YAG}{1}
\TaskStart{3}
\TaskEnd{36}
\TaskResults{%
\ref{del:eval1},
\ref{del:eval2},
\ref{del:eval3}
%\ref{del:model1}
}
\TaskHeader{}
\tasklabel{task:evaluation}
In \theTask{}, we will evaluate the \TheProject{} tools and technologies against the overall project requirements and success criteria that were identified in~\ref{task:requirements}, using the representative metrics and use cases that were developed in~\ref{task:usecases}. We aim to evaluate those aspects that contribute to the impact from the industrial partners within the selected domains. We will evaluate how much the security and privacy of the data, including for any data analytics models, is improved, by using the \TheProject{} techniques. This includes simulated cyber-attacks for testing. We will also evaluate how easy it is for end users to use the \TheProject{} methodology for developing secure distributed data analytics applications. The lessons learned from this task will provide feedback into different technical work packages, steering the development of \TheProject{} technologies in the subsequent phases of the project. We will also undertake roadmapping activities for future \TheProject{} technologies. This roadmapping will be in the context of emerging technical developments in  secure distributed AI-based big data analytics applications, and also in broadening the long-term impact of the \TheProject{} project. This roadmap will consider new and emerging developments in security, privacy, AI, data lakes, cloud computing, etc., as they relate to the analytics of big data. It will highlight possible future applications of the \TheProject{} technology, and indicate the further work that needs to be done to bring the \TheProject{} technologies to market (\ref{del:eval3}). All of the project partners will participate in this task, evaluating the techniques they will develop for the use cases, and providing expertise for the overall project roadmap.
%The result of this task will be a report describing the technical roadmap, highlighting the issues for future research and use of \TheProject{} technology, also being used as input for T8.2 (Exploitation and Use). 
  \end{Task}
  

%\begin{Task}
%  \TaskTitle{What about Assessment of business oriented sensitive data protection ?}
%  \TaskParticipant{YAG}{1}
%  \TaskResults{
%  \ref{del:roadmap}.
%  }
%  \TaskHeader{}
%  \tasklabel{task:BusinessDataFlaws}
%  \color{blue} \textbf{(TBD)}
%  Dealing with data, the YAG-Suite could support a certain level of business oriented semantics for multilevel security requirements allocation, detecting potential sensitive business data leaks and start building the bridge between business views (for instance Air Traffic Management data vs Unmanned Traffic Management data if it makes sense that they have different security requirements) and programmatic views.
%  In \TheProject{} it would be interesting to include such customization in the use cases.
%  
%  \end{Task}

\begin{WPDeliverables}
  \begin{compactitem}
\item \ref{del:req1} (Month 3) : Report on Initial Requirements for \TheProject{} Techniques
\item \ref{del:eval1} (Month 14): Report on Initial Implementation and Evaluation of Use Cases
\item \ref{del:req2} (Month 15) : Report on Final Requirements for \TheProject{} Techniques
\item \ref{del:eval2} (Month 26) : Report on Refined Implementation and Evaluation of Use Cases
\item \ref{del:eval3} (Month 36) : Report on Final Implementation and Evaluation of Use Cases and Project Roadmap
\end{compactitem}
\end{WPDeliverables}
\end{Workpackage}
