\addtocounter{wpno}{1}
\begin{Workpackage}{\thewpno}
\wplabel{wp:authentication}
\WPTitle{\wpname{\thewpno}}
\WPStart{Month 1}
\WPParticipant{COGNI}{35}
\WPParticipant{USTAN}{1}
\WPParticipant{UOD}{5.5}
\WPParticipant{SOPRA}{2.5}
\WPParticipant{SCCH}{1}
\WPParticipant{FRQ}{5}

\begin{WPObjectives}
The objectives of \theWP{} are to:
\begin{compactitem}
\item Define security metrics and policies for identity, authentication and access management;
\item Develop multi-factor authentication schemes and continuous user identification methods based on intelligent biometrics;
\item Develop mechanisms for regulating access and tracking the lineage and provenance of data using a Blockchain-based approach;
\item Design and develop an integrated identity management service, including an interactive dashboard for service integration and analytics;
\item Design, develop and evaluate privacy-preserving methods of biometric data.
\end{compactitem}
\end{WPObjectives}

\begin{WPDescription}
\theWP{} aims to address Objective 5 by designing and developing an AI-driven and user-centered IDentity-as-a-Service (IDaaS) that will allow developers to integrate core identity management components in their applications, aiming to preserve security but at the same time provide a positive security experience to the end-users. The primary goals are to: i) provide high levels of security through multi-factor authentication, and continuous user identification to confirm the identity of each end-user and accordingly authorize access to certain parts of data in the system; ii) improve the usability and user experience through easy-to-use and fluid authentication and continuous user identification methods; and iii) enable developers and administrators to easily integrate, deploy and manage their preferred authentication method based on custom requirements and policies. We will initially adopt state-of-the-art security and usability metrics of token-based and biometric-based user authentication schemes and accordingly design various security metrics and policies (T5.1), we will design and develop working prototypes of multi-factor authentication and continuous user identification based on intelligent biometrics (T5.2), we will develop data access control mechanisms using a Blockchain-based approach (T5.3), we will design and develop an integrated identity management service, including an interactive dashboard for service integration and analytics (T5.4), and we will design, develop and evaluate privacy-preserving methods of biometric data (T5.5).

\end{WPDescription}

\begin{Task}
\TaskTitle{Access Management, Security Metrics and Policies}
\TaskParticipant{COGNI}{3}
\TaskParticipant{SOPRA}{0.5}
\TaskParticipant{FRQ}{?}

\TaskStart{1}
\TaskEnd{26}
\TaskResults{%
\ref{del:auth1},
\ref{del:auth3},
\ref{del:auth4}
}
\TaskHeader{}
\tasklabel{task:ua_metrics}

In \theTask, we will define the methodology and conduct an analysis, elicitation and validation of the security measurements, metrics and policies of the identity management service. It will include an extensive analysis on state-of-the-art research in the area of multi-factor authentication and intelligent biometric authentication aiming to identify important security metrics and policies, as well as current best practices and guidelines in the area. We will further conduct an analysis of works on existing user authentication approaches within the aerospace, healthcare and banking domains in order to identify the peculiarities of identity management in each domain. The task will also entail a series of semi-structured interviews that will be conducted with various stakeholders at the end-user organizations aiming to identify current identity management and user authentication practices, policies and procedures followed at each organization. This task will proceed in three phases, developing access management, security metrics and policy designs tailored to the initial, refined and final versions of the use cases, respectively. This
will be reported in D5.1 (M6), D5.3 (M22) and D5.4 (M34). COGNI will lead this task, contributing expertise in identity management mechanisms. SOPRA will contribute feedback from the use cases. 
\end{Task}

\begin{Task}
\TaskTitle{Multi-Factor Authentication and Continuous User Identification System Design and Development}
\TaskParticipant{COGNI}{20}
\TaskParticipant{USTAN}{1}
\TaskParticipant{SOPRA}{2}

\TaskStart{3}
\TaskEnd{34}
\TaskResults{%
\ref{del:auth2},
\ref{del:auth3},
\ref{del:auth4}
}
\TaskHeader{}
\tasklabel{task:ua_designdevelopment}

In \theTask, we will design and develop the core identification and authentication component. In particular, we will design and develop a series of multi-factor authentication (MFA) schemes, and continuous user identification methods aiming to augment the existing security layer of distributed data analytics application with intelligent biometrics. Continuous authentication will be achieved by combining: i) face biometrics through face recognition and analysis; ii) eye gaze behavior biometrics by analyzing the end-users' eye gaze data and visual behavior during interaction; and iii) physiological biometrics by analyzing the end-users' physiological data such as heart rate, skin conductance, accelerometer, etc. through non-intrusive wearable sensing devices (e.g., smartwatches, smartbands). A User-Centered Design methodology will be adopted for developing and finalizing the MFA and continuous user identification scheme through multiple iterations (three releases are anticipated; initial, refined, final software) that will be used for evaluation studies. The developed schemes and methods will be evaluated through experiments with synthetic data as well as actual users (as part of WP7) aiming to evaluate its security and usability aspects. In the context of conducting studies with actual users with the developed service, we assure that all personal data collection and processing will be carried out according to EU and national legislation. This task will proceed in three phases. In the first phase, we will produce low-fidelity working prototypes with limited functionality as well as paper prototypes based on users’ requirements analysis. In the second phase, the service will be further developed into an extended working prototype including almost all functional requirements, however that has not been optimized in terms of performance and user experience. A fully functional identity management service will be developed in the third phase. Emphasis will be given to the interoperability of the components and maximizing the efficiency and effectiveness of the service running on different platforms (desktop, mobile, wearable devices). The service will be validated based on benchmarking software criteria and qualities ensuring the adequate performance, robustness, scalability, maintainability and security. This task will proceed in three phases. 
\end{Task}

\begin{Task}
\TaskTitle{Access Management with Blockchain}
\TaskParticipant{COGNI}{6}
\TaskParticipant{SOPRA}{?}

\TaskStart{3}
\TaskEnd{34}
\TaskResults{%
\ref{del:auth2},
\ref{del:auth3},
\ref{del:auth4}
}
\TaskHeader{}
\tasklabel{task:ua_blockchain}
We will design and develop a decentralized access control system based on Blockchain technology. In particular, the implemented system will store tamper-free data, such as auditing data, on the Blockchain network, in order to guarantee that the auditing records cannot be altered. In a similar way, access control rules will be stored on the Blockchain network and the owner of the data will be responsible for updating them. This will allow the authorization to be performed over the Blockchain network and will allow developers to seamlessly integrate it in the applications. When a certain component of the developer's application requires authorization of a user and verification of her access rights, it will send a transaction to the decentralized Blockchain network. Since there will not be any centralized authority responsible for validating the access rights, the risk of single point of failure will be eliminated. After the validation of access rights, the Blockchain will respond back to the requestor component. This will respect data privacy since the particular component will not be able to disclose any data without reaching the Blockchain network. This task will proceed in three phases. In the first phase, we will produce low-fidelity working prototypes with limited functionality. In the second phase, the access control system will be further developed into an extended working prototype including almost all functional requirements, however that has not been optimized in terms of performance and user experience. A fully functional access control system will be developed in the third phase.
\end{Task}


\begin{Task}
\TaskTitle{User Interaction Analytics}
\TaskParticipant{COGNI}{3}
\TaskParticipant{SCCH}{1}
\TaskParticipant{FRQ}{2}
\TaskStart{6}
\TaskEnd{34}
\TaskResults{%
\ref{del:auth2},
\ref{del:auth3},
\ref{del:auth4}
}
\TaskHeader{}
\tasklabel{task:ua_analytics}

In \theTask, we will design and develop tools and methods aiming to provide easy and intuitive access to the users’ interaction data collected in different formats (e.g., data and information visualizations, and tabular representations). An interactive dashboard will be developed that will allow developers and administrators to setup their user authentication and policies, and compile and display users’ interaction and behavior data to monitor their experience and highlight any usability issues. In addition, it will support various filtering and aggregation options to focus on a particular subset of the collected data based on, for e.g., time and threshold filters. The interactive dashboard will translate the raw data into inferred knowledge, which will be be useful for the experimental evaluation in WP7. It will be integrated in the overall system's interactive dashboard (WP6). This task will proceed in three phases. 
\end{Task}

\begin{Task}
\TaskTitle{Privacy-Preserving Biometrics in Authentication for Big Data Systems}
\TaskParticipant{UOD}{5.5}
\TaskParticipant{COGNI}{3}
\TaskParticipant{FRQ}{3}
\TaskStart{5}
\TaskEnd{34}
\TaskResults{%
\ref{del:auth2},
\ref{del:auth3},
\ref{del:auth4}
}
\TaskHeader{}
\tasklabel{task:privacyBiometrics}
Bearing in mind that users' biometric data are irrevocable when exposed, it is very important to protect their privacy. In this task, new methods will be designed, developed and evaluated for preserving the privacy of the biometric data used for continuous authentication. Main aim of privacy-preserving biometric authentication is to enable users to verify themselves without disclosing raw and sensitive biometric information. For doing so, current privacy weaknesses and threats in biometric authentication will be analyzed, and novel privacy-preserving methods will be designed and developed to secure the implementation of biometric-based continuous authentication, such as, features' transformation, cancelable biometrics, pseudo-identities, etc.
\end{Task}


\begin{WPDeliverables}
  \begin{compactitem}
\item \ref{del:auth1} (Month 6): Initial Report on Security Metrics and Policies.
\item \ref{del:auth2} (Month 9): Software for the Initial Intelligent Identity Management System.
\item \ref{del:auth3} (Month 22): Software for the Refined Intelligent Identity Management System.
\item \ref{del:auth4} (Month 34): Report on Implementation of Final Intelligent Identity Management System, Security Metrics and Policies.
\end{compactitem}
\end{WPDeliverables}
\end{Workpackage}
