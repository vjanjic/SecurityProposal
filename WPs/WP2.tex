\addtocounter{wpno}{1}
\begin{Workpackage}{\thewpno}
\wplabel{wp:vulnerability}
\WPTitle{\wpname{\thewpno}}
\WPStart{Month 1}
\WPParticipant{IBM}{1}
\WPParticipant{SCCH}{1}
\WPParticipant{SOPRA}{1}
\WPParticipant{USTAN}{1}
\WPParticipant{YAG}{1}

\begin{WPObjectives}
The objectives of \theWP{} are to:
\begin{compactitem}
%% Why just embedded? Are these domains correct? KH
\item Solve the world
\end{compactitem}
\end{WPObjectives}

\begin{WPDescription}
\theWP{} addresses the problem of solving the world.
\end{WPDescription}

\begin{Task}
%\TaskTitle{Parallel Computation and Control for High-Level Modelling Languages} % for Aerospace and Automotive Industries}
\TaskTitle{Advanced Vulnerability Detection in Source Code}
\TaskParticipant{UOD}{1}

\TaskStart{1}
\TaskEnd{34}
\TaskResults{%
%\ref{del:model1}
}
\TaskHeader{}
\tasklabel{task:vulnerability}

In \theTask, we will develop advanced vulnerability detection technology for C/C++ code. We plan to apply IBM's ExpliSAT symbolic interpretation tool to discover known security vulnerability patterns in C/C++code. Moreover, to avoid known limitations of the symbolic execution approach (e.g. path explosion, memory growth, etc.), we will develop a technology that leverages the combination of symbolic execution, static code analysis and white box fuzzing to find exploitable bugs. The idea is to perform symbolic execution as the main technique to discover vulnerabilities combined with tailored fuzzing techniques in specific small areas for the purpose of assisting the symbolic execution engine to consistently make progress. For that end, the symbolic execution engine will query a fuzzer at run-time for information that would assist the symbolic interpreter make progress where it previously couldn't. Static code analysis will assist to find suspected vulnerability locations prior to the invocation of the ExpliSAT tool and to control "hard to execute" use cases.

This task will proceed in three phases ...

\end{Task}

\begin{Task}
\TaskTitle{Vulnerability detection in Source Code based on static analysis and machine learning}
\TaskParticipant{YAG}{1}

\TaskStart{1}
\TaskEnd{34}
\TaskResults{%
%\ref{del:model1}
}
\TaskHeader{}
\tasklabel{task:sast}

Starting from the YAG-Suite capabilities, in \theTask we will provide and improve the advanced vulnerability detection technology which is acheivable with SAST for JAVA and C/C++ source code. To avoid the recurring problem of false positives and duplicate warnings raising from scanning, the results generated by Static Analysis will be automatically qualified by a machine learning based post-processing of SAST warnings. 
The qualification process will support a risk based approach and provide decision making information for each warning such as its relevancy and its impact on CVSS metrics such as confidentiality, integrity, etc.
Vulnerability detection will also include a remediation focus to provide options for fixing vulnerabilities and extract, when available examples of correct source code as (and when) found in the application.
The YAG-Suite will offer an API to partners so that they can query SAST results they need.
Static code analysis will assist to find suspected vulnerability locations prior to the invocation of the ExpliSAT tool and after, in order to refine ”hard to find” vulnerabilities.


\end{Task}


\begin{WPDeliverables}
  \begin{compactitem}
    \item XX
%\item \ref{del:model1} (Month 10): Report on Initial Block-Diagram Modelling, Patterns and Code Synthesis
\end{compactitem}
\end{WPDeliverables}
\end{Workpackage}
