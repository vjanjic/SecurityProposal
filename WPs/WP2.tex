\addtocounter{wpno}{1}
\begin{Workpackage}{\thewpno}
\wplabel{wp:vulnerability}
\WPTitle{\wpname{\thewpno}}
\WPStart{Month 1}
\WPParticipant{IBM}{30}
\WPParticipant{YAG}{30}
\WPParticipant{SCCH}{11}
\WPParticipant{USTAN}{8}
\WPParticipant{UOD}{6}
\WPParticipant{SOPRA}{5}
\WPParticipant{COGNI}{2}



\begin{WPObjectives}
The objectives of \theWP{} are to:
\begin{compactitem}
\item Develop static source code analysis methods to isolate the portions of the end user code that possibly contain security vulnerabilities.
\item Develop dynamic code analysis methods, based on symbolic execution and reinforcement learning that will profile the identified portions of the code to check for security vulnerabilities.
\item Augment the static and dynamic code analysis with runtime monitoring of the applications executed on distributed systems to identify the runtime security attacks on the end user applications.
\item Develop novel source-to-source code refactoring transformations to repair the vulnerabilities identified by the static and dynamic analysis.
\item Develop mechanisms for the runtime adaptation of the application for the situation where runtime security vulnerabilities have been identified or are suspected.
\end{compactitem}
\end{WPObjectives}

\begin{WPDescription}
The aim of \theWP{} is to develop the foundations %techniques 
for identifying and repairing security vulnerabilities in C++ and Java code. In this workpackage we focus on generic techniques, whereas WP3 deals with adaptation of these techniques to the problem of AI-based distributed data analytics. In order to ensure that we catch the widest possible range of security vulnerabilities, we will develop three layers of analysis. Initially we will perform static analysis (\ref{task:staticanalysis}) on the source code in order to identify the portions of code that are susceptible to vulnerabilities. We will then perform dynamic analysis based on symbolic execution (\ref{task:dynamicanalysis}) on the identified portions of code. These two phases will be augmented with runtime monitoring (\ref{task:runtime}) that will allow us to identify threats that can arise during the execution of the application and from dynamic unpredictable interactions, and would therefore be missed by static and dynamic analysis. The second part of the workpackage will be dedicated to developing \emph{self-healing} technology for repairing code vulnerabilities. This will be done both statically, using formalised source-to-source code transformations ({\ref{task:statichealing}}) that will be implemented in the user interface in WP6, and at runtime, using runtime adaptation of the application and environment (\ref{task:runtimehealing}).
\end{WPDescription}

\begin{Task}
\TaskTitle{Static Source Code Analysis for Identifying Security Vulnerabilities}
\TaskParticipant{YAG}{27}
\TaskParticipant{IBM}{5}
\TaskParticipant{SA}{2}
\TaskParticipant{UOD}{2}
\TaskParticipant{SOPRA}{1}
\TaskParticipant{COGNI}{1}
\TaskStart{1}
\TaskEnd{34}
\TaskResults{%
\ref{del:vul1},
\ref{del:vul2},
\ref{del:vul3}
}
\TaskHeader{}
\tasklabel{task:staticanalysis}
In this task we will develop technology for static analysis of the end user application source code to identify potential security vulnerabilities. Static analysis of the source code is relatively cheap in terms of computational complexity, but has a drawback that it produces high volume of information and potentially a significant number of false positives and duplicate warnings. The YAG-Suite includes supervised machine learning capabilities to support a first level of false positive reduction. We will build on the YAG-Suite existing capabilities and extend them with advanced vulnerability detection technology, based on SAST and machine learning. We will extend YAG-Suite capabilities with vulnerability modelling and detection mechanisms to detect a wider spectrum of vulnerabilities that address AI based big data analytics applications, written in Java and C/C++ code, with associated false positive reduction training. We will also develop code mining capabilities with associated APIs (Application Programming Interface) to feed other tasks with source code metrics that will serve as an input to the dynamic vulnerability detection technology from Task \ref{task:dynamicanalysis}, where the portions of code where potential vulnerabilities are identified will be further analysed. The output of this task will also feed into Task \ref{task:statichealing} as part of the inputs to self healing, and Task \ref{task:frontend}, the user interface where the potential vulnerabilities will be displayed to the end users in a structured way. This task will proceed in three phases. In the first phase, we will develop an initial version of the static analysis infrastructure for AI-based distributed data analytics targeting the Java language. This will be reported in software deliverable~\ref{del:vul1}. In the second phase, we will build on the experience of the first phase and extend the tool to address additional issues that arise from execution of Java applications on public clouds, and we will also produce an initial version of the analysis for C++. This will be reported in~\ref{del:vul2}. In the third and final phase, we will produce the final version of the static analysis infrastructure for Java and C++, addressing the issues that arise in execution of applications on public clouds. This will be reported in~\ref{del:vul3}.

\YAGshort{} will lead this task, providing expertise in static analysis based on machine learning and building on their YAG-Suite toolset for identifying vulnerabilities in source code. \IBMshort{} will provide the link between static analysis and dynamic analysis that follows it in the \TheProject{} tool chain. \SAshort{} will contribute with formalising rules for detection of vulnerabilities that will be a part of refactoring-based self-healing. \UODshort{} will provide feedback on the issues related to AI-based big data analytics, while \COGNIshort{} will look into the link of static analysis and extension of the applications with the authentication infrastructure. \SOPRAshort{} will provide feedback from the use cases.


\end{Task}

\begin{Task}
\TaskTitle{Symbolic Execution and Supervised Learning for Identifying Security Vulnerabilities}
\TaskParticipant{IBM}{24}
\TaskParticipant{UOD}{2}
\TaskParticipant{YAG}{1}
\TaskParticipant{COGNI}{1}
\TaskParticipant{SOPRA}{1}
\TaskStart{1}
\TaskEnd{34}
\TaskResults{%
\ref{del:vul1},
\ref{del:vul2},
\ref{del:vul3}
}
\TaskHeader{}
\tasklabel{task:dynamicanalysis}
In this task we will develop technology for dynamic detection of vulnerabilities in the end user code. Focusing on the portions of user code identified by the technologies in Task \ref{task:staticanalysis}, we will apply IBM's ExpliSAT symbolic interpretation tool to verify the presence of security vulnerability patterns in the code. We plan to combine symbolic execution with white box fuzzing to find exploitable bugs. Combining static code analysis with dynamic detection will allow us to avoid the known problems of path explosion and memory growth when only symbolic execution is used, by allowing this technology to focus on relatively small portions of the user code and thus consistently making progress. To that end, the symbolic execution engine will query a fuzzer at run-time for information that would assist the symbolic interpreter make progress where it previously could not. In addition to symbolic execution, we will also develop technology for supervised machine learning post-processing of SAST warnings to further eliminate the possible false positives. This task will proceed in three phases. In the first phase, we will develop an initial version of the dynamic analysis technique for AI-based big data analytics based on ExpliSAT that will analyse identified portions of the code. This will be reported in software deliverable~\ref{del:vul1}. In the second phase, we will extend this with the capabilities for fuzzy querying and improve the linking with the static and runtime analysis. This will be reported in~\ref{del:vul2}. In the third and final phase, we will produce the final version of dynamic analysis infrastructure, incorporating also ML-based post processing of warnings from the static analysis. This will be reported in~\ref{del:vul3}. 

\IBMshort{} will lead this task, providing expertise in dynamic code analysis and building on their ExpliSAT symbolic interpretation tool. \UODshort{} will provide feedback on the issues for dynamic analysis arising from AI-based big data analytics. \YAGshort{} will provide the feedback from the static analysis phase and contribute to linking between the two phases. \COGNIshort{} will provide feedback arising from integrating authentication mechanisms into the applications, while \SOPRAshort{} will provide advice on the specific issues arising from use cases.
\end{Task}

\begin{Task}
\TaskTitle{Runtime Analysis for Detecting Vulnerabilities}
\TaskParticipant{SCCH}{5}
\TaskParticipant{YAG}{2}
\TaskParticipant{UOD}{1}
\TaskParticipant{IBM}{1}
\TaskParticipant{SOPRA}{1}
\TaskStart{1}
\TaskEnd{34}
\TaskResults{%
\ref{del:vul1},
\ref{del:vul2},
\ref{del:vul3}
}
\TaskHeader{}
\tasklabel{task:runtime}
In this task we will extend the vulnerability detection technology with the mechanisms that will monitor the runtime execution of the end user applications. This will allow us to identify vulnerabilities that arise from the unexpected interaction of data processing agents during computations and which could, therefore, not be identified by the static and dynamic analysis mechanisms. We will identify which data has to be collected at runtime in order to detect anomalies, i.e. when there are agents in the distributed parallel system that interact with the other agents in a malicious way. This requires the specification of monitoring agents. Based on this, we will then develop anomaly detection algorithms and their specification by further agents. The focus will be on discovery, whether the interaction sequence of any agent is in line with the specification of the system. This task will proceed in three phases. In the first phase, we will produce the initial version of the runtime analysis to collect the information from the running application. This will be demonstrated in the software deliverable~\ref{del:vul1}. In the second phase, we will extend this analysis to collect additional information related to execution of AI-based data analytics on public clouds. This will be reported in~\ref{del:vul2}. In the third and final phase, we will produce the final version of the analysis addressing the issues that arise in the execution of these analytics on public clouds. This will be reported in~\ref{del:vul3}.

\SCCHshort{} will lead this task, contributing with their expertise in system monitoring and information collection. \YAGshort{} will contribute with providing information relevant to runtime analysis that can be obtained from the analysis of the application source code. \UODshort{} will provide feedback from the specific issues related to AI-based big data analytics and provide the link with the static analysis phase. \IBMshort{} will contribute to linking of the output of the dynamic analysis phase with runtime monitoring. \SOPRAshort{} will provide advice on the issues arising from the use cases considered. 
\end{Task}

\begin{Task}
\TaskTitle{Refactoring for Self Healing}
\TaskParticipant{USTAN}{6}
\TaskParticipant{UOD}{1}
\TaskParticipant{SOPRA}{1}
\TaskStart{1}
\TaskEnd{34}
\TaskResults{%
\ref{del:vul1},
\ref{del:vul2},
\ref{del:vul3}
}
\TaskHeader{}
\tasklabel{task:statichealing}

In \theTask{}, we will develop a number of refactorings that will transform C++ programs into equivalent forms with introduced security patterns (from~\ref{task:patterns}). The refactorings will be defined as source-to-source semi-automatic program transformations. 
%
%
%In \theTask{}, we will develop a number of program transformations to support the end-user by providing tool-support through the  secure-aware programming methodology from \ref{task:methodology} in WPX. These program transformations will be implemented as source-to-source program \emph{refactorings} in T6.1 WP6 and will operate at the program source level. The transformations will be proved correct in Task XX WPX. 
%
These refactorings themselves will be developed based on a set of formal refactoring rules (which will be defined in terms of the refactoring's pre- and post-conditions, together with a set of transformation rules). The output of the refactoring will be an equivalent application, with introduce security patterns, where the refactorings will focus on the introduction of such patterns into the source-code. The result will be a functionally equivalent program with increased security and decreased vulnerability. In this task we will develop a number of program transformation rules based on transforming the target language, together with their implementation, feeding in to~\ref{task:dashboard}. We will consider the outputs from the Static Source Code Analysis (\ref{task:staticanalysis})   and XX to provide the refactoring tool with the static analysis information required to discovery vulnerabilities and information on how to repair them.
%Prototype rules will also be developed for other languages, such as Java and Python. 
%
This task will proceed in three phases. In the \emph{first} phase we will identify transformations for C++  that will introduce the patterns from~\ref{task:patterns} into C++ applications and use-cases (from~\ref{wp:usecases}), by defining the refactoring's  pre- and post-conditions, together with their formal transformations rules. This will be demonstrated in software deliverable~\ref{del:vul1}. In the \emph{second} phase, we will produce implementations of the refactorings defined in the first phase, for C++. This will be reported in~\ref{del:vul2}.
 In the \emph{third} phase, we will integrate our implementations into the dashboard user-interface from~\ref{task:dashboard} and provide further refactorings for some of the advanced patterns identified in~\ref{task:patterns}.
We will also explore prototype transformation rules for some of the simpler patterns for Java. This will be reported in~\ref{del:vul3}.

\SAshort{} will lead this task, contributing with their expertise in software refactoring. \YAGshort{} will contribute with the expertise in static analysis, helping formulate pre- and post-conditions that the code has to satisfy for refactorings. \UODshort{} will contribute with the expertise in big data analytics, identifying specific issues that arise in these applications while \SOPRAshort{} will provide advice based on the use cases considered.

\end{Task}



%\begin{Task}
%	\TaskTitle{Correctness of Refactorings for Self-Healing}
%	\TaskParticipant{USTAN}{24}
%	\TaskParticipant{SCCH}{1}
%	\TaskParticipant{IBM}{1}
%	\TaskStart{1}
%	\TaskEnd{34}
%	\TaskResults{%
%		%%\ref{del:model1}
%	}
%	\TaskHeader{}
%	\tasklabel{task:runtimehealing}
%In \theTask, we will formalise the refactorings from~\ref{task:statichealing} that introduce the security patterns (from~\ref{task:patterns})  developing mechanised proofs of their functional correctness. This will improve confidence to the end-user that the refactorings will not change the functional behaviour of the programs being refactored. Proofs that the refactorings increase the security properties of the program will not be handled here, but instead given in T4.X.
%
%The mechanisations will focus on a tractable subset of the target language, in order to demonstrate functional  equivalence between the original program and the refactored output, based on the identified semantics representation. This will involve identifying a representation of the semantics for the subset of the target language and defining an equivalence relation between the original and refactored programs. We will provide mechanised proofs in the form of implementations in a dependently typed language, such as Idris.
%
%This task will proceed in a number of phases. In the \emph{first} phase, we will identify the formal semantics of a tractable subset of the target language. In the \emph{second} phase, we will define an equivalence relation between the original program and the refactored output w.r.t. to the semantics identified in the first phase. Finally, in the \emph{third} phase will produce mechanised proofs of functional correctness for number of the refactorings produced  in~\ref{task:statichealing}, w.r.t. to the semantics and equivlence relations from the previous phases.
%
%\end{Task}


\begin{Task}
\TaskTitle{Runtime Adaptation for Self Healing}
\TaskParticipant{SCCH}{7}
\TaskParticipant{IBM}{2}
\TaskParticipant{SOPRA}{1}
\TaskStart{1}
\TaskEnd{34}
\TaskResults{%
\ref{del:vul1},
\ref{del:vul2},
\ref{del:vul3}
}
\TaskHeader{}
\tasklabel{task:runtimehealing}
In this task we will develop mechanisms for runtime adaptation of the end user applications in the case when security violations are detected or suspected or when a suspicious agents has been detected that should be removed from the system. Both cases require the partial interrupt of the distributed application and/or infrastructure, the rollback to a consistent state and the restart after some modifications. We will also formulate the repairability proof obligations on the grounds of the research through which it will be specified, specifying after how many steps a secure situation is expected to be restored. This task will proceed in three phases. In the first phase, we will develop an initial runtime adaptation inf
\end{Task}


%\begin{Task}
%%\TaskTitle{Parallel Computation and Control for High-Level Modelling Languages} % for Aerospace and Automotive Industries}
%\TaskTitle{Advanced Vulnerability Detection in Source Code}
%\TaskParticipant{UOD}{1}
%
%\TaskStart{1}
%\TaskEnd{34}
%\TaskResults{%
%%\ref{del:model1}
%}
%\TaskHeader{}
%\tasklabel{task:vulnerability}
%
%In \theTask, we will develop advanced vulnerability detection technology for C/C++ code. We plan to apply IBM's ExpliSAT symbolic interpretation tool to discover known security vulnerability patterns in C/C++code. Moreover, to avoid known limitations of the symbolic execution approach (e.g. path explosion, memory growth, etc.), we will develop a technology that leverages the combination of symbolic execution, static code analysis and white box fuzzing to find exploitable bugs. The idea is to perform symbolic execution as the main technique to discover vulnerabilities combined with tailored fuzzing techniques in specific small areas for the purpose of assisting the symbolic execution engine to consistently make progress. For that end, the symbolic execution engine will query a fuzzer at run-time for information that would assist the symbolic interpreter make progress where it previously couldn't. Static code analysis will assist to find suspected vulnerability locations prior to the invocation of the ExpliSAT tool and to control "hard to execute" use cases.
%
%This task will proceed in three phases ...
%
%\end{Task}
%
%\begin{Task}
%\TaskTitle{Vulnerability detection in Source Code based on static analysis and machine learning}
%\TaskParticipant{YAG}{1}
%
%\TaskStart{1}
%\TaskEnd{34}
%\TaskResults{%
%%\ref{del:model1}
%}
%\TaskHeader{}
%\tasklabel{task:sast}
%
%Starting from the YAG-Suite capabilities, in \theTask we will provide and improve the advanced vulnerability detection technology which is acheivable with SAST for JAVA and C/C++ source code. To avoid the recurring problem of false positives and duplicate warnings raising from scanning, the results generated by Static Analysis will be automatically qualified by a supervised machine learning based post-processing of SAST warnings. 
%The qualification process will support a risk based approach and provide decision making information for each warning such as its relevancy and its impact on CVSS metrics such as confidentiality, integrity, etc.
%Vulnerability detection will also include a remediation focus to provide options for fixing vulnerabilities and extract, when available, examples of correct source code as (and when) found in the application.
%The YAG-Suite will offer an API to partners so that they can query the SAST results they need.
%Static code analysis will assist to find suspected vulnerability locations prior to the invocation of the ExpliSAT tool and after, in order to refine ”hard to detect” vulnerabilities.
%
%
%\end{Task}
%
%\begin{Task}
%\TaskTitle{Anomaly Detection and Adaptation for Self-Healing}
%\TaskParticipant{SCCH}{1}
%
%\TaskStart{1}
%\TaskEnd{34}
%\TaskResults{%
%%\ref{del:model1}
%}
%\TaskHeader{}
%\tasklabel{task:scch}
%In a first step it will be analysed which run-time data has to be collected through monitoring machines that are needed for the detection of any anomaly, i.e. there are agents in the distributed concurrent system that interact with the other agents in a malicious way. This requires the specification of monitoring agents. The second step addresses the development of anomaly detection algorithms and their specification by further agents. The focus will be on discovery, whether the interaction sequence of any agent is in line with the specified system. The third step
%addresses adaptation algorithm in case a security violation is detected or a suspicious agents
%has been detected that should be removed from the system. Both cases require the partial interrupt of the distributed concurrent system, the rollback to a consistent state and the restart after some modifications. The last step undertaken in this task is the formulation of repairability proof obligations on the grounds of the research through which it will be specified, after how many steps a secure situation is expected to be restored.
%\end{Task}







\begin{WPDeliverables}
  \begin{compactitem}
    \item XX
%\item \ref{del:model1} (Month 10): Report on Initial Block-Diagram Modelling, Patterns and Code Synthesis
\end{compactitem}
\end{WPDeliverables}
\end{Workpackage}
