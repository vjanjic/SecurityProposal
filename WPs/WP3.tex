\addtocounter{wpno}{1}
\begin{Workpackage}{\thewpno}
\wplabel{wp:securityBigData}
\WPTitle{\wpname{\thewpno}}
\WPStart{Month 1}
\WPParticipant{SCCH}{1}
\WPParticipant{UOD}{1}
\WPParticipant{COGNI}{1}
\WPParticipant{USTAN}{1}
\WPParticipant{UCM}{1}
\WPParticipant{YAG}{1}
\WPParticipant{FRQ}{1}

\begin{WPObjectives}
The objectives of \theWP{} are to:
\begin{compactitem}
\item Identify security risks in systems for distributed storage that are used in large-scale distributed machine learning based on big data;
%, targetting both distributed NoSQL databases such as ScyllaDB and MongoDb and plain distributed filesystems such as Hadoop HDFS; 
\item Identify security risks in distributed machine-learning based processing of big data, targetting systems based on distributed parallel patterns such as MapReduce;
\item Develop novel mechanisms for evaluating and reducing privacy leakage in distributed machine learning processing systems;
%secure and privacy-preserving mechanisms for data release in machine learning based big data processing; \vjcomment{SCCH part}
\item Develop a novel framework for privacy-preserving transfer of knowledge extracted from big private data sets to other parties;
%techniques for privacy-secured knowledge sharing in machine learning based AI systems; \vjcomment{SCCH part}
\item Support semi-automatic repairing of the user code by integrating the self-healing mechanisms from WPXX into security and privacy-leakage diagnostics mechanisms for machine-learning based big data storage and analysis;
\end{compactitem}
\end{WPObjectives}

\begin{WPDescription}
  The goal of this workpackage is to provide mechanisms for identifying security problems in distributed big data storage and processing, focusing on machine-learning based data analytics. It will exploit the technologies for the source-code level analysis using symbolic execution (WPXX) to identify security risks that appear both in the individual data storage and processing agents in isolation, as well as from the interaction of these agents in a distributed setting. In task \ref{task:storage} we will focus on the problem of identifying vulnerabilities in systems for storage of big data used to train the machine-learning models. These systems most often use distributed NoSQL databases, such as ScyllaDB or MongoDB, or plain distributed filesystems such as HDFS that use very basic security mechanisms. Complementary to that, task \ref{task:processing} focuses on machine-learning based distributed analysis of big data. We restrict our attention to the systems that are based on parallel patterns (most of which being built on top of MapReduce paradigm), as parallel patterns encapsulate all interaction between different distributed agents, therefore localising the possible source of vulnerabilities in the distributed systems. This work will feed into WPXX, where security properties of such systems will be formally established and proved. Task \ref{task:privacyLeakage} focuses on evaluating and reducing privacy-leakage in distributed machine learning, by investigating optimal noise-adding mechanisms for that. Task \ref{task:knowledgeSharing} investigates knowledge sharing in distributed machine learning, developing techniques for privacy-preserving transfer of knowledge extrated from large data sets. Finally, task \ref{task:healing} investigates implementing self-healing mechanisms developed in WPXX into the security and privacy-leakage diagnostics techniques developed in this workpackage. Collectively, the diagnostics and repair mechanisms for security and privacy of machine learning data analytics from this work package will feed feed into the refactoring-based self-healing methodology from WP6.   XXXX   Add large picture, pyramid etc. 
\end{WPDescription}

\begin{Task}
\TaskTitle{Identifying Vulnerabilities in Distributed Data Storage for Machine Learning}
\TaskParticipant{UOD}{1}

\TaskStart{1}
\TaskEnd{36}
\TaskResults{%
%\ref{del:model1}
}
\TaskHeader{}
\tasklabel{task:storage}
In \theTask, we will define the methodology and conduct an analysis of vulnerabilities in big data storage such as NoSQL databases and file systems such as HDFS.  We will select examples of  NoSQL databases constructed in C++ or  based on the JVM ( Java and Scala for instance) and identify vulnerabilities in the source code that could lead to security risks. Open source projects will be selected so that we can exploit the technologies for the source-code level analysis using symbolic execution (WPXX) to identify security risks that appear both in the individual data storage and processing agents in isolation, as well as from the interaction of these agents in distributed setting. In particular, side projects (such as language drivers) will be examined as these provide an attack vector that may be outside the open source projects control. 
\UCM will perform vulnerability analysis of C++ storage software components.
 \end{Task}

 \begin{Task}
 \TaskTitle{Identifying Vulnerabilities in Distributed Pattern-Based Machine Learning Data Analytics}
 \TaskParticipant{UOD}{1}
 
 \TaskStart{1}
 \TaskEnd{36}
 \TaskResults{%
 %\ref{del:model1}
 }
 \TaskHeader{}
 \tasklabel{task:processing}
 In \theTask, we will define the methodology and conduct an analysis of vulnerabilities in big data analysis systems.  These systems provide their own security challenges as data is distributed across a great many systems and whilst analysis is in process the data may be transferred across nodes.  We will not tackle the security of the data whilst it is on the move (on the wire and the network stack) as that are the subject of other research areas.  The task will follow a similar pattern to that of \ref{task:storage} selecting candidates systems from those written in C++ or based on the JVM (Hadoop, Spark, Flink, Storm) exploiting source-code level analysis using symbolic execution (WPXX) to expose security vulnerabilities.
\UCM will perform vulnerability analysis of C++ processing software components.
\end{Task}
 
\begin{Task}
  \TaskTitle{Evaluation and Reduction in Data Leakage for Distributed Machine Learning Data Analytics}
  \TaskParticipant{SCCH}{1}
  
  \TaskStart{1}
  \TaskEnd{36}
  \TaskResults{%
  %\ref{del:model1}
  }
  \TaskHeader{}
  \tasklabel{task:privacyLeakage}
  In this task, we will develop mechanisms to both evaluate and reduce information leakage in distributed machine learning data analytics systems. The methods for evaluation of privacy leakage will be investigated in terms of relationship between sensitive data and released public data. Our approach is to employ a stochastic model for approximating the uncertain mapping between released noise added data and private data. The stochastic model facilitates a variational approximation of privacy-leakage in-terms of mutual information between sensitive private data and released public data. For reducing information leakage, we will develop optimal noise adding mechanisms for preserving privacy in distributed machine learning. We will derive analytically the noise distribution that maximizes a given utility function or equivalently minimizes a given data distortion function. Different use cases may demand different utility functions and thus optimal noise adding mechanism is derived for each utility function separately.    XXXX link T3.3 and T 3.4
 \end{Task}

 \begin{Task}
  \TaskTitle{Knowledge Sharing in Distributed AI}
  \TaskParticipant{SCCH}{1}
  
  \TaskStart{1}
  \TaskEnd{36}
  \TaskResults{%
  %\ref{del:model1}
  }
  \TaskHeader{}
  \tasklabel{task:knowledgeSharing}
  This task will develop an analytical framework to study and optimize the privacy-preserving transfer of knowledge extracted from a large set of labelled private data owned by a party to another party owning a few labelled data samples. The goal is to answer the question: how can a model be transferred from a source to a target domain while preserving privacy of both source and target domains? An information theoretic approach is considered to quantify transferability of knwoledge from source to target domain in-terms of mutual information between source and target data. The privacy secured knowledge sharing framework facilitates development of transfer and multi-task machine learning algorithm while optimizing the privacy-transferability tradeoff.
 \end{Task}

 \begin{Task}
  \TaskTitle{Self-Healing in Distributed Data Processing Systems}
  \TaskParticipant{SCCH}{1}
  
  \TaskStart{1}
  \TaskEnd{36}
  \TaskResults{%
  %\ref{del:model1}
  }
  \TaskHeader{}
  \tasklabel{task:healing}
  In \theTask, we will integrate self-healing mechanisms developed in WPXX into the tools and techniques developed in this work package. Collectively, the diagnostics and repairs of security vulnerabilities developed in this work package will feed into the refactoring-based self-healing methodology from WP6. A sample use case from one of the technologies surveyed in \ref{task:storage}, \ref{task:processing}in  will be selected to demonstrate the application of self healing to that technology. 
 \end{Task}
 
 
 \begin{Task}
  \TaskTitle{Privacy-preserving Biometric Data}
  \TaskParticipant{SCCH}{1}
   \TaskParticipant{COGNI}{1}
  
  \TaskStart{1}
  \TaskEnd{36}
  \TaskResults{%
  %\ref{del:model1}
  }
  \TaskHeader{}
  \tasklabel{task:privacyBiometrics}
  Bearing in mind that users' biometric data are irrevocable when exposed, it is very important to protect their privacy. In this task, new methods will be designed, developed and evaluated for preserving the privacy of the biometric data used for continuous authentication in WP5. Main aim of privacy-preserving biometric authentication is to enable users to verify themselves without disclosing raw and sensitive biometric information. For doing so, current privacy weaknesses and threats in biometric authentication will be analyzed, and novel privacy-preserving methods will be designed and developed to secure the implementation of biometric-based continuous authentication, such as, features' transformation, cancelable biometrics, pseudo-identities, etc.
 \end{Task}
 
 
\begin{Task}
\TaskTitle{Privacy vulnerability detection in Source Code from static analysis}
\TaskParticipant{YAG}{1}

\TaskStart{1}
\TaskEnd{36}
\TaskResults{%
%\ref{del:model1}
}
\TaskHeader{}
\tasklabel{task:PrivacyFromSAST}
In \theTask, we will improve the static analysis tool and its post-processing algorithms to extend vulnerability detection to distributed data specific vulnerabilities, with a particular focus on privacy and anonymization leaks. We will use as an input the description of the specific security and privacy vulnerabilities of distributed Machine learning to be identified in the task XX. We will integrate them in the static analysis detection and qualification algorithms. Programming languages to be scanned will be JAVA and C/C++.

The SAST and post processing algorithms will be developed to support design flaws in order to detect anonymization vulnerabilities and risks.

The vulnerability detection will include a business oriented semantics approach in order to support multilevel privacy and sensitive data requirements.

 \end{Task}


\begin{WPDeliverables}
  \begin{compactitem}
    \item XX
%\item \ref{del:model1} (Month 10): Report on Initial Block-Diagram Modelling, Patterns and Code Synthesis
\end{compactitem}
\end{WPDeliverables}
\end{Workpackage}
