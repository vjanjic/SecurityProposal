\addtocounter{wpno}{1}
\begin{Workpackage}{\thewpno}
\wplabel{wp:dissem}
\WPTitle{\wpname{\thewpno}}
\WPStart{Month 1}
\WPParticipant{UOD}{7}
\WPParticipant{SOPRA}{7}
\WPParticipant{SA}{5}
\WPParticipant{IBM}{5}
\WPParticipant{SCCH}{5}
\WPParticipant{COGNI}{5}
\WPParticipant{UCM}{5}
\WPParticipant{FRQ}{5}
\WPParticipant{YAG}{5}

\begin{WPObjectives}
The objectives of \theWP{} are to:
\begin{compactitem}
  \item Disseminate research results to the scientific community;
  \item Ensure awareness of the results in the user community;
  \item Raise general public awareness of the \TheProject{} project, using an Open Science model and a MOOC;
  \item Define individual exploitation plans;
  and
  \item Manage existing and new intellectual property.
\end{compactitem}
\end{WPObjectives}

\begin{WPDescription}
As described in detail in Section~\ref{sect:dissemination} (page~\pageref{sect:dissemination}), this work package entails the dissemination of research results, the construction of an exploitation plan for the knowledge acquired during the course of the \TheProject{} project, 
the establishment of/extension of the user community for the \TheProject{} tools and techniques,
plus communication activities aimed at improving public awareness of the \TheProject{} project.
It also includes IPR management and data management.
Scientific and technical work aimed directly at ensuring the publication of the project results will be carried out within the relevant technical workpackages.
\end{WPDescription}

\begin{Task}
\TaskTitle{Dissemination and Communication Activities}
\TaskParticipant{UOD}{3}
\TaskParticipant{SA}{3}
\TaskParticipant{SCCH}{3}
\TaskParticipant{UCM}{3}
\TaskParticipant{IBM}{2}
\TaskParticipant{SOPRA}{2}
\TaskParticipant{COGNI}{2}
\TaskParticipant{YAG}{2}
\TaskParticipant{FRQ}{1}
\TaskResults{%
\ref{del:pressrelease1},
\ref{del:website1},
\ref{del:dissemplan1},
\ref{del:dissemplan2},
\ref{del:pressrelease2}.
\ref{del:website2}.
}

\TaskStart{1}
\TaskEnd{36}
\TaskHeader{}

\vjcomment{All partners should have 10\% of their effort here.}

This task comprises all forms of direct dissemination and public communication activities.
%
\textbf{Dissemination activities} will primarily involve the production of high-quality scientific and technical research papers and associated presentations as described in Section~\ref{sect:dissemination}.
It will also involve the production of a project website, including visitor analysis and monitoring tools, promotion through social media (e.g., twitter, facebook, linkedin), technical workshop organisation, creation of advertisement materials such as flyers, posters, and electronic feeds as well as their distribution, and the engagement with key bodies such as \hipeac, ET4HPC, PRACE etc. In addition, the project will engage in open source bodies (such as the Apache Foundation) to publicise its work and organise talks at the associated Open Source conferences.  Furthermore, \TheProject will disseminate its results towards standardisation bodies and working groups.
%
\textbf{Communication activities} will include the production of press releases, outreach activities (seminars, keynote talks, media interviews, CORDIS press releases), general information on the project website, and the use of social media to ensure wider engagement with the general public.
News articles will be produced by experienced professional staff at relevant partners and communicated to local, national and international media, as appropriate.
At least two press releases will be generated in the course of the project.

\end{Task}

\begin{Task}
\TaskTitle{Exploitation and Use}
\TaskParticipant{SOPRA}{4}
\TaskParticipant{FRQ}{4}
\TaskParticipant{IBM}{3}
\TaskParticipant{COGNI}{3}
\TaskParticipant{YAG}{3}
\TaskParticipant{SCCH}{2}
\TaskParticipant{UCM}{1}
\TaskResults{%
\ref{del:data-mgt-plan},
\ref{del:dissemplan1},
\ref{del:dissemplan2}.
}

\TaskStart{1}
\TaskEnd{36}
\TaskHeader{}


This task involves producing, refining and updating the exploitation plan for the project as a whole, starting with the draft exploitation plans that have been outlined earlier on page~\pageref{sect:exploitation-plan}.
Exploitable results that will be produced in the course of \TheProject may lead to commercial innovation activities, to advances in scientific knowledge, and/or to advances in education, as appropriate.
It is the principle of all exploitation activities to use research results to create value within all participating organisations and thus to improve their competitive advantages.
\taskbreak
Hence, this task aims at preparing the transfer of the technology developed in \TheProject{} to the project partners and to other academic and industrial partners that could gain technical, commercial and research benefit from the project results. 
In order for the exploitation to be effective, an integrated approach will be necessary, combining expertise from the development department and solution management, and the involvement of a user base represented by the consortium partners. 
An integral part of this exploitation task will be the \TheProject{} use cases from WP7 which will serve as validation points throughout the project lifetime.
This task also includes the continuous analysis of transfer opportunities and the evaluation of the advancement of the research results against the user requirements/needs throughout the project.
All necessary adjustments to the project plan will be communicated to project partners  in order to ensure the best possible outcome. \TheProject{} will participate in the Horizon 2020 Open Research Data Pilot and therefore will comply with all
requirements including the development of a Data Management Plan (DMP). WP8 will engage the \TheProject{}
partners in the development of a DMP including an open access data policy and data submission instructions will
be prepared and circulated to all partners during the kick‐off meeting. Common data collection protocols and database structures will be agreed at the outset ensuring common
standards, interoperability, backup and storage alongside measures for sharing data including 
encryption, provenance, inter‐communication and knowledge sharing. Any external user surveys will be performed
with the necessary privacy, encryption and anonymisation measures in place.
\end{Task}

\begin{Task}
\TaskTitle{IPR Management}

\TaskParticipant{SA}{1}

\TaskResults{%
\ref{del:dissemplan1};
\ref{del:dissemplan2}
}
\TaskStart{1}
\TaskEnd{36}
\TaskHeader{}

This task involves producing and maintaining a register of exploitable results that will be produced in the course of \TheProject{}, including recording each exploitable foreground IPR result, which partners are involved in generating that IPR, and how the result could be exploited in each case.
The register will be confidential, but available to all partners.
\end{Task}

\begin{Task}
\TaskTitle{Education Program}

\TaskParticipant{UOD}{5}
\TaskParticipant{SA}{1}
\TaskParticipant{SOPRA}{1}
\TaskParticipant{SCCH}{1}

\TaskResults{%
\ref{del:mooc}
\ref{del:dissemplan3}
}
\TaskStart{1}
\TaskEnd{36}
\TaskHeader{}

This task will develop educational and learning materials, and
design and validate an innovative learning programme for delivery through the \TheProject{} MOOC. The MOOC will require planning, design and creation of content, guided by expertise from FutureLearn and CTIL in the UOD.  For example, MOOC material will require 
 text transcripts to be translated from English for participants across the EU. The MOOC will use the \TheProject{}
consortium resources and expertise, with inputs from international science and technology partners within the \TheProject{}
consortia and their networks on methods, protocols and guides for collaborative digital security experiments.  The MOOC will engage a professional film maker to assist the production of videos and an educational consultant to assist in the planning of educational materials.


\TheProject{} will organise at least one open technical workshop each year
(preferably co-located with a major conference or other meeting).
The consortium's academic partners will include project methodologies and achievements in their undergraduate-, graduate- and PhD-level teaching activities within their curricula, will provide web-based short courses and recorded lectures on specific \TheProject topics, and will organise student workshops and summer schools.
\end{Task}

\begin{WPDeliverables}
\begin{compactitem}
\item \ref{del:pressrelease1} (Month 3): Press Release Announcing the start of the \TheProject{} project.
\item \ref{del:website1} (Month 3): Initial Project Website / Presentation.
\item \ref{del:data-mgt-plan} (Month 6): Data Management Plan.
\item \ref{del:dissemplan1} (Month 12): First Interim Report on Dissemination \& Exploitation, including Exploitation Plan, Communication Activities, User Community Building, IPR Management and First Project Workshop.
\item \ref{del:mooc} (Month 24): Report on the first run of the MOOC including learner statistics. 
\item \ref{del:dissemplan2} (Month 24): Second Interim Report on Dissemination \& Exploitation, including Exploitation Plan, Communication Activities, User Community Building, IPR Management, and Second Project Workshop.
\item \ref{del:pressrelease2} (Month 36): Final Press Release Describing the \TheProject{} Results.
\item \ref{del:website2} (Month 36): Final Project Website / Presentation.
\item \ref{del:dissemplan3} (Month 36): Final Report on Dissemination \& Exploitation, including Exploitation Plan, Communication Activities, User Community Building, IPR Management, and Final Project Workshop.\end{compactitem}
\end{WPDeliverables}
\end{Workpackage}

