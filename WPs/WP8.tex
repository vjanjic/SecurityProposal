\addtocounter{wpno}{1}
\begin{Workpackage}{\thewpno}
\wplabel{wp:dissem}
\WPTitle{\wpname{\thewpno}}
\WPStart{Month 1}
\WPParticipant{USTAN}{8}
\WPParticipant{PRL}{6.5}
\WPParticipant{IBM}{5}
\WPParticipant{INRIA}{5}
\WPParticipant{GOLEM}{5}
\WPParticipant{CODEPLAY}{5}
\WPParticipant{SCCH}{4}
\WPParticipant{AGH}{4}
\WPParticipant{JMOIC}{2}


\begin{WPObjectives}
The objectives of \theWP{} are to:
\begin{compactitem}
  \item Disseminate research results to the scientific community;
  \item Ensure awareness of the results in the user community;
  \item Raise general public awareness of the \TheProject{} project, using an Open Science model;
  \item Define individual exploitation plans;
  and
  \item Manage existing and new intellectual property.
\end{compactitem}
\end{WPObjectives}

\begin{WPDescription}
As described in detail in Section~\ref{sect:dissemination} (page~\pageref{sect:dissemination}), this work package entails the dissemination of research results, the construction of an exploitation plan for the knowledge acquired during the course of the \TheProject{} project, 
the establishment of/extension of the user community for the \TheProject{} tools and techniques,
plus communication activities aimed at improving public awareness of the \TheProject{} project.
It also includes IPR management and data management.
Scientific and technical work aimed directly at ensuring the publication of the project results will be carried out within the relevant technical workpackages.
\end{WPDescription}

\begin{Task}
\TaskTitle{Dissemination and Communication Activities}

%% \TaskParticipant{SA}{9}
%% \TaskParticipant{AGH}{4}
%% \TaskParticipant{INRIA}{3.75}
%% \TaskParticipant{IBM}{3}
%% \TaskParticipant{PRL}{3}
%% \TaskParticipant{CODEPLAY}{3}
%% \TaskParticipant{SCCH}{2.5}
%% \TaskParticipant{GOLEM}{2}
%% \TaskParticipant{JMOIC}{1}

\TaskParticipant{USTAN}{4}
\TaskParticipant{INRIA}{4}
\TaskParticipant{SCCH}{2.5}
\TaskParticipant{PRL}{2.5}
\TaskParticipant{AGH}{2.5}
\TaskParticipant{GOLEM}{2.5}
\TaskParticipant{CODEPLAY}{2}
\TaskParticipant{IBM}{1.5}
\TaskParticipant{JMOIC}{0.5}

\TaskResults{%
\ref{del:pressrelease1},
\ref{del:website1},
\ref{del:dissemplan1},
\ref{del:dissemplan2},
\ref{del:pressrelease2}.
\ref{del:website2}.
}

\TaskStart{1}
\TaskEnd{36}
\TaskHeader{}

\khcomment{All partners should have 10\% of their effort here.}

This task comprises all forms of direct dissemination and public communication activities.
%
\textbf{Dissemination activities} will primarily involve the production of high-quality scientific and technical research papers and associated presentations as described in Section~\ref{sect:dissemination}.
It will also involve the production of a project website, including visitor analysis and monitoring tools, promotion through social media (e.g., twitter, facebook, linkedin), technical workshop organisation, creation of advertisement materials such as flyers, posters, and electronic feeds as well as their distribution, and the engagement with key bodies such as \hipeac, ET4HPC, PRACE etc.
\TheProject{} will organise at least one open technical workshop each year
(preferably co-located with a major conference or other meeting).
The consortium's academic partners will include project methodologies and achievements in their undergraduate-, graduate- and PhD-level teaching activities within their curricula, will provide web-based short courses and recorded lectures on specific \TheProject topics, and will organise student workshops and summer schools.
Furthermore, \TheProject will disseminate its results towards standardisation bodies and working groups.
%
\textbf{Communication activities} will include the production of press releases, outreach activities (seminars, keynote talks, media interviews, CORDIS press releases), general information on the project website, and the use of social media to ensure wider engagement with the general public.
News articles will be produced by experienced professional staff at relevant partners including \SAshort, \IBMshort,  \PRLshort, \SCCHshort and \INRIAshort, and communicated to local, national and international media, as appropriate.
At least two press releases will be generated in the course of the project.
\end{Task}

\begin{Task}
\TaskTitle{Exploitation and Use}
\TaskParticipant{CODEPLAY}{2}
\TaskParticipant{USTAN}{2}
\TaskParticipant{PRL}{2}
\TaskParticipant{IBM}{1.5}
\TaskParticipant{GOLEM}{1.5}
\TaskParticipant{INRIA}{1}
\TaskParticipant{JMOIC}{1}
\TaskParticipant{SCCH}{0.5}
\TaskParticipant{AGH}{0.5}

\TaskResults{%
\ref{del:data-mgt-plan},
\ref{del:dissemplan1},
\ref{del:dissemplan2}.
}

\TaskStart{1}
\TaskEnd{36}
\TaskHeader{}

\khcomment{We will lead this, but everyone must put some effort into it.}

This task involves producing, refining and updating the exploitation plan for the project as a whole, starting with the draft exploitation plans that have been outlined earlier on page~\pageref{sect:exploitation-plan}.
Exploitable results that will be produced in the course of \TheProject may lead to commercial innovation activities, to advances in scientific knowledge, and/or to advances in education, as appropriate.
It is the principle of all exploitation activities to use research results to create value within all participating organisations and thus to improve their competitive advantages.
Hence, this task aims at preparing the transfer of the technology developed in \TheProject{} to the project partners and to other academic and industrial partners that could gain technical, commercial and research benefit from the project results.

In order for the exploitation to be effective, an integrated approach will be necessary, combining experience and expertise from the development department and solution management, and the involvement of a user base represented by the consortium partners.
An integral part of this exploitation task will be the \TheProject{} use cases (see~\ref{wp:eval}) which will serve as validation points throughout the project lifetime.

This task also includes the continuous analysis of transfer opportunities and the evaluation of the advancement of the research results against the user requirements/needs throughout the project.
All necessary adjustments to the project plan will be communicated to project partners  in order to ensure the best possible outcome.
\end{Task}

\begin{Task}
\TaskTitle{User Community Building}
\TaskParticipant{PRL}{2}
\TaskParticipant{IBM}{2}
\TaskParticipant{USTAN}{1}
\TaskParticipant{SCCH}{1}
\TaskParticipant{GOLEM}{1}
\TaskParticipant{CODEPLAY}{1}
\TaskParticipant{AGH}{1}
\TaskParticipant{JMOIC}{0.5}
\TaskResults{%
\ref{del:dissemplan1};
\ref{del:dissemplan2}
}
\TaskStart{10}
\TaskEnd{36}
\TaskHeader{}
\tasklabel{task:usercommunity}

\khcomment{Could be rolled into Exploitation and Use, but might be good to separate it. This should be led by a commercial partner.}
This task involves the organisation of activities that are aimed at encouraging the long-term uptake of \TheProject{} tools and technologies.
This includes, \emph{but is not limited to}:
\begin{inparaenum}
\item
running dedicated workshops and tutorials to encourage uptake;
\item
establishing and maintaining a communication strategy for the user community;
\item
production of user-level documentation and training material, including webinars etc.;
\item
promotion and dissemination activities within the targeted communities;
\item
production and distribution of promotional and informational materials, including press releases, flyers etc.;
\item
updating and managing the web portal;
and
\item
preparation of trade stands, presentations etc. for key industry events.
\end{inparaenum}
\end{Task}

\begin{Task}
\TaskTitle{IPR Management}

\TaskParticipant{SA}{1}

\TaskResults{%
\ref{del:dissemplan1};
\ref{del:dissemplan2}
}
\TaskStart{1}
\TaskEnd{36}
\TaskHeader{}
\khcomment{We will lead this.  There is only minimal effort required by the other partners.}
This task involves producing and maintaining a register of exploitable results that will be produced in the course of \TheProject{}, including recording each exploitable foreground IPR result, which partners are involved in generating that IPR, and how the result could be exploited in each case.
The register will be confidential, but available to all partners.
\end{Task}

\begin{WPDeliverables}
\begin{compactitem}
\item \ref{del:pressrelease1} (Month 3): Press Release Announcing the start of the \TheProject{} project.
\item \ref{del:website1} (Month 3): Initial Project Website / Presentation.
\item \ref{del:data-mgt-plan} (Month 6): Data Management Plan.
\item \ref{del:dissemplan1} (Month 12): First Interim Report on Dissemination \& Exploitation, including Exploitation Plan, Communication Activities, User Community Building, IPR Management and First Project Workshop.
\item \ref{del:dissemplan2} (Month 24): Second Interim Report on Dissemination \& Exploitation, including Exploitation Plan, Communication Activities, User Community Building, IPR Management, and Second Project Workshop.
\item \ref{del:pressrelease2} (Month 36): Final Press Release Describing the \TheProject{} Results.
\item \ref{del:website2} (Month 36): Final Project Website / Presentation.
\item \ref{del:dissemplan3} (Month 36): Final Report on Dissemination \& Exploitation, including Exploitation Plan, Communication Activities, User Community Building, IPR Management, and Final Project Workshop.\end{compactitem}
\end{WPDeliverables}
\end{Workpackage}

