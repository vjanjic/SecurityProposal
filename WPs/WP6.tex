\addtocounter{wpno}{1}
\begin{Workpackage}{\thewpno}
\wplabel{wp:methodology}
\WPTitle{\wpname{\thewpno}}
\WPStart{Month 1}
\WPParticipant{UCM}{32}
\WPParticipant{USTAN}{10}
\WPParticipant{SOPRA}{10}
\WPParticipant{COGNI}{5}
\WPParticipant{FRQ}{5}
\WPParticipant{UOD}{4}
\WPParticipant{IBM}{3}
\WPParticipant{YAG}{3}
\WPParticipant{SCCH}{1}



\begin{WPObjectives}
The objectives of \theWP{} are to:
\begin{compactitem}
\item Develop a new intelligent user-interface to support security-aware big-data computing.
\item Develop new security patterns to improve security in big-data applications.
\item Provide new coding standards for security-aware applications.
\item Provide new refactorings to make code compliant with the new security standards.
\item Develop new continuous deployment techniques.
\item Develop a new security-aware software engineering methodology for secure big-data applications.

\end{compactitem}
\end{WPObjectives}

\begin{WPDescription}
This work package will investigate new software engineering techniques and tools for the development of security-aware big-data applications. In \ref{task:iui}, we will design and develop a new intelligent user interface that will integrate the various refactoring into a user interface. New security-aware software patterns will be developed in \ref{task:patterns}, and will be provided to the user in the form of a new software library, with refactoring tool-support for their introduction in \ref{task:statichealing}.  In \ref{task:standards}, we will develop a new coding standard for the development of secure big-data applications, including new refactorings to support tool-supported compliance of applications to the security-standards, in \ref{task:compliance}. A continuous deployment methodology will be produced in \ref{task:deployment}, allowing the \TheProject{} tools to become part of a continuous deployment workflow.
Finally, in \ref{task:methodology}, we will produce a new security-aware software engineering methodology encapsulating all aspects of the \TheProject{} project together, enabling interoperability between the tools developed in the other work packages to form a coherent software tool-chain and methodology for the end-user. 


\end{WPDescription}

\begin{Task}
%\TaskTitle{Parallel Computation and Control for High-Level Modelling Languages} % for Aerospace and Automotive Industries}
\TaskTitle{Intelligent User Interface for Security-Aware Big-Data Computing}
\TaskParticipant{COGNI}{3}
\TaskParticipant{YAG}{2}
\TaskParticipant{USTAN}{1}
\TaskStart{6}
\TaskEnd{30}
\TaskResults{%
%\ref{del:model1}
}
\TaskHeader{}
\tasklabel{task:iui}
% alternative for Intelligent User Interface -> Developer-centered User Interface
In this task we will design and develop an intelligent user interface that will integrate the code refactoring component and the identity management component as a unified interactive workbench for the developers. The refactoring component will be integrated into a front-end development tool, such as a web-based interface and/or integrated into the ParaFormance refactoring tool, supporting both Eclipse and Visual-Studio for C++ applications. A similar implementation for Java and Python-based applications will also be produced. The user interface will display the outcomes from code refactoring including the secured, privacy-preserved and verified code, as well as code recommendations for improving the security and privacy aspects of the existing code. The code recommendations will be displayed following best practices and guidelines with regards to usability, user experience and user-centred design. In addition, we will implement a graphical user interface as a "wizard" that will display the recommendations (e.g., code fixes) along with the severity of each detected vulnerability and recommendation. The wizard will guide the developer through the process of refactoring the code and will provide a mechanism for easily applying the recommended changes. Furthermore, the user interface will provide an access point to the identity management component for integrating multi-factor authentication methods, intelligent biometrics and access management into their code. The results of this task will be used to refactor the use-cases in \ref{wp:usecases}. The task will proceed in three phases, capturing the different phases of the development of the project, reported in D6.1 (M13), D6.3 (M25) and the final phase reported in D6.5 (M36). \textbf{COGNI} will lead this task and develop the user-interface, \textbf{YAG} will provide input into the interface design, \textbf{USTAN} will provide support with integration into the ParaFormance tool.

%Explainable recommendations
%Moreover, the GUI will display the latest news on security-related threads for increasing the security-awareness of developers.


%
%In particular, in this task we will produce new refactorings that will:
%
%\begin{itemize}
%	\item Repair vulnerable code, by taking the output of the Self Healing tooling from XX, 
%	\item Introduce security-aware patterns, based on the implementation of the patterns in Task X.
%	\item Rewrite source-code so that it conforms to the standard as defined in Task X.
%\end{itemize}
\end{Task}

\begin{Task}
	%\TaskTitle{Parallel Computation and Control for High-Level Modelling Languages} % for Aerospace and Automotive Industries}
	\TaskTitle{Security-aware Software Development Patterns}
	\TaskParticipant{USTAN}{3}
	\TaskParticipant{UCM}{3}
	\TaskParticipant{UOD}{2}
	\TaskParticipant{SOPRA}{1}
	\TaskParticipant{FRQ}{1}
	\TaskStart{1}
	\TaskEnd{34}
	\TaskResults{%
		%\ref{del:model1}
	}
	\TaskHeader{}
	\tasklabel{task:patterns}
	
In \theTask{} we will design and implement patterns of security that arise in typical big-data applications. We will investigate the relationship between common security threats and their source-level solutions, producing a high-level domain-specific language (DSL) for describing a set of design patterns for the end-user. This DSL will be made available to the end-user developer as a library of security-aware patterns and will be used by the refactorings from \ref{task:statichealing}. The DSL will be implemented in C++ and Java, using advanced features from e.g. the latest C++ and Java standards, abstracting away the low-level details of programming security models.
%
The task will proceed in three phases. In the \emph{first} phase, we will identify a set of \emph{fundamental} patterns for repairing security vulnerabilities. These patterns will reported as high-level schemas, encapsulating the side-conditions to the pattern, the semantics of the pattern, and design models of the the pattern's implementation (D6.1).
In the \emph{second} phase, we will provide implementations of the fundamental security patterns from the first phase, implemented as a C++ and a Java library (D6.3). In the \emph{third} and final phase, we will extend the fundamental set of patterns, to include some advanced security patterns, and their prototype implementations in C++ and Java (D6.5).  \textbf{USTAN} will lead this task, \textbf{UC3M} and \textbf{SCCH} will provide expertise in C++ and Java, \textbf{UOD} will contribute to the pattern implementation, \textbf{SOPRA} and \textbf{FRQ} will provide feedback on the pattern designs. 
\end{Task}

\begin{Task}
	%\TaskTitle{Parallel Computation and Control for High-Level Modelling Languages} % for Aerospace and Automotive Industries}
	\TaskTitle{New Coding Standards for Security-Aware Applications}
	\TaskParticipant{UCM}{12}
	\TaskParticipant{SOPRA}{2}
	\TaskParticipant{USTAN}{1}
	\TaskParticipant{FRQ}{1}
		\TaskStart{1}
	\TaskEnd{36}
	\TaskResults{%
		%\ref{del:model1}
	}
	\TaskHeader{}
	\tasklabel{task:standards}
	
	In \theTask, we will develop new C++ coding and security standards to facilitate the programming of secure big-data applications. The task will proceed in two phases. 
First, we will start identifying those properties related to security from multiple existing coding standards and guidelines both those that are generic and those that are language specific. We will pay special attention to issues related to programming language features. A key issue in this context is the interaction with the C++ standard, as most guidelines and standards consider C++11 and 14 (with poor consideration to C++17 and none regarding the upcoming C++20). However, by the time this project has ended C++23 is expected to be published. We will also provide prototype standards for Java-like languages. 
%
We will identify for each guideline the specific enforcement means that can be used with a focus in how much of the enforcement can be automated. As a result, we will produce a new set of C++ coding guidelines that will take into consideration the multiple editions of the C++ standard.
%
This task will comprise three phases. An \emph{initial} phase (D6.2); a \emph{second}, refined, phase (D6.4); and a \emph{final} phase (D6.6).
%
\UCM will lead this task and will work in the development of the new set of C++ coding guidelines providing a link with the ISO C++ standards committee. \textbf{USTAN} will provide integration with the refactoring tools, \textbf{FRQ} and \textbf{SOPRA} will provide feedback from the use-cases. 

%First, we will identify those properties that arise in security-aware applications, such as secrecy of variables, branching vulnerabilities, etc.; secondly, we will implement a new security-aware programming standard, encapsulating the properties from the first phase, to create a new C++ coding standard. 
%	\begin{itemize}
%		\item Identify the language-properties relating to security 
%		\item vulnerabilities, branching, etc.? non-functional properties?
%		\item create a new C++ coding style/standard.
%	\end{itemize}
\end{Task}

\begin{Task}
	\TaskTitle{Compliance to coding standards}
	\TaskParticipant{UCM}{10}
	\TaskParticipant{USTAN}{3}
	\TaskParticipant{SOPRA}{2}
	\TaskParticipant{FRQ}{1}


	\TaskStart{3}
	\TaskEnd{36}
	\TaskResults{%
		%\ref{del:model1}
	}
	\TaskHeader{}
	\tasklabel{task:compliance}
	
%	In \theTask, we will provide a common interface to multiple tools
 %       and components aiming to support the enforcement of the coding
  %      guidelines developed in Task~\ref{task:standards}.
        
     	In \theTask{} we will define refactorings that enable application to adhere to the coding standards as defined in~\ref{task:standards}. Similarly as in~\ref{task:statichealing}, we will define the refactorings as a set of formal rewrite rules, based on their pre-and post conditions and transformation rules. The output of the refactoring will be a functionally equivalent C++ program, but with increased conformance to the C++ coding standards. We will also provide implementations of the refactorings that will feed into the end-user interface in~\ref{task:iui}. The refactorings defined in this task will work at a lower granularity than those defined in~\ref{task:statichealing}: in \theTask{}, the refactorings will rewrite specific structural features of the source-code that violate the coding standard. 
     %
     This task will also comprise three phases, mirroring that of \ref{task:standards}. An \emph{initial} phase, where we will identify a set of transformation rules that will rewrite C++ programs in such a way that they conform to the C++ standards from~\ref{task:standards} (D6.2, M13);  a \emph{second}, refined, phase, we will produce implementations of the refactorings identified in the first phase, feeding into the end-user dashboard in~\ref{task:iui} (D6.4, M25); and a \emph{final} phase, where we will develop advanced refactorings, reported in D6.6 (M36).
%
        \UCMshort{} will lead this task with the implementation
        of new specific checks that can be integrated in the open source
        \texttt{clang-tidy} toolset. They will also contribute with specific
        library solutions to mitigate other vulnerabilities.  \textbf{USTAN} will provide implementations of the refactorings, \textbf{SOPRA} and \textbf{FRQ} will provide feedback from the use-cases.
\end{Task}

\begin{Task}
\TaskTitle{Continuous Deployment}
\TaskParticipant{UCM}{6}
\TaskParticipant{SOPRA}{4}
\TaskParticipant{IBM}{2}
\TaskParticipant{UOD}{1}
	\TaskParticipant{COGNI}{1}
		\TaskParticipant{FRQ}{1}



\TaskStart{5}
\TaskEnd{36}
\TaskResults{%
	%\ref{del:model1}
}
\TaskHeader{}
\tasklabel{task:deployment}

In \theTask. we will produce a new software engineering methodology for secure continuous deployment, where code is typically compiled on the fly on a deployment server before being deployed to a test or run time server.  The \TheProject{} tools will become part of the continuous deployment workflow that will use the tools to determine a security rating for the code and if this rating is not acceptable will raise an error. This exception will stop the deployment and prevent the insecure code being deployed. During the auto-healing phase the continuous deployment will ensure the latest code is available for immediate deployment if the deployed code was not compromised. Continuous deployment also ensures that deployments are done as a chain of smaller changes, preventing massive changes that risk compromising the complete system in a single deployment.  We will assess existing continuous deployment methods and their applicability to develop secure-aware applications for big data. We will assess the individual tools and techniques developed on the \TheProject{} project, and their applicability to continuous deployment. The task will have three phases, focusing on private clouds in the \emph{first} phase (D6.1, M13); public clouds in the \emph{second} phase (D6.3, M25); and in the \emph{third} phase, we will focus on \emph{hybrid} clouds (D6.5, M36). \textbf{UC3M} will lead this task, \textbf{SOPRA} will provide support on software engineering methodologies for continuous deployment, and feedback from the use-cases, \textbf{IBM} will contribute with their expertise in software engineering and software development methodologies, \textbf{UOD} will contribute by providing expertise in big-data, \textbf{COGNI} and \textbf{FRQ} will contribute with feedback and testing.
\end{Task}

%\begin{Task}
%	\TaskTitle{Refactorings for Security Standard Conformance}
%	\TaskParticipant{USTAN}{24}
%	\TaskParticipant{SCCH}{1}
%	\TaskParticipant{IBM}{1}
%	\TaskStart{1}
%	\TaskEnd{34}
%	\TaskResults{%
%		%%\ref{del:model1}
%	}
%	\TaskHeader{}
%	\tasklabel{task:runtimehealing}
%	In \theTask{} we will define refactorings that enable application to adhere to the coding standards as defined in~\ref{task:standards}. Similarly as in~\ref{task:statichealing}, we will define the refactorings as a set of formal rewrite rules, based on their pre-and post conditions and tranformation rules. The output of the refactoring will be a functionally equivalent C++ program, but with increased conformance to the C++ coding standards. We will also provide implementations of the refactorings that will feed into the end-user dashboard in~\ref{task:dashboard}. The refactorings defined in this task will work at a lower granularity than those defined in~\ref{task:statichealing}: in \theTask{}, the refactorings will rewrite specific structural features of the source-code that violate the coding standard. 
%	
%	This task will proceed in \emph{three} phases. In the \emph{first} phase, we will identify a set of transformation rules that will rewrite C++ programs in such a way that they conform to the C++ standards from~\ref{task:standards}. In the \emph{second} phase, we will produce implementations of the refactorings of the refactorings identified in the first phase, feeding into the end-user dashboard in~\ref{task:dashboard}.
%	
%\end{Task}

\begin{Task}
	%\TaskTitle{Parallel Computation and Control for High-Level Modelling Languages} % for Aerospace and Automotive Industries}
	\TaskTitle{A Methodology for the Development of Secure Applications}

	\TaskParticipant{USTAN}{2}
	\TaskParticipant{UOD}{1}
	\TaskParticipant{IBM}{1}
    \TaskParticipant{SOPRA}{1}
    \TaskParticipant{SCCH}{1}
    \TaskParticipant{COGNI}{1}
	\TaskParticipant{UCM}{1}
	\TaskParticipant{FRQ}{1}
	\TaskParticipant{YAG}{1}
	
	\TaskStart{8}
	\TaskEnd{36}
	\TaskResults{%
		%\ref{del:model1}
	}
	\TaskHeader{}
	\tasklabel{task:methodology}
	
	In \theTask, we will produce a new software engineering methodology for the development of secure-aware big-data applications. In this task, we will look to assess the suitability of the tools and techniques developed within the project to support different software development models. 
	In the first phase of this task, we will assess existing  software engineering methods and their applicability to develop secure-aware applications for big data.
	In the second phase of this task, we will assess the applicability of the individual tools and techniques developed on the \TheProject{} project, and their applicability to a secure-aware software methodology. 
	Finally, in the final phase of this task, we will ensure inter-operability of the tools produced in WP2, WP3, WP4, WP5 and WP6, producing an inter-operable secure-aware tool-chain. The task will proceed in \emph{three} phases: the \emph{first} phase will develop a methodology for secure big-data analytics on \emph{private} clouds (D6.1, M13); the \emph{second} phase will develop a methodology for \emph{public} clouds (D6.3, M25); and the \emph{final} phase will develop a methodology for \emph{hybrid} clouds (D6.5, M36). \textbf{USTAN} will lead this task, all other partners will provide support and feedback in the implementation of the methodology with their tools and use-cases.
%	\begin{itemize}
%		\item requirements capture (software engineering based?)
%		\item software development models?
%		\item assess applicability of tools ?
%	\end{itemize}
\end{Task}


%\begin{Task}
%	\TaskTitle{How \YAG can contribute. Here are some ideas of what we could do:}
%	\TaskParticipant{YAG}{1}
%	
%	\TaskStart{1}
%	\TaskEnd{27}
%	\TaskResults{%
%		%\ref{del:model1}
%	}
%	\TaskHeader{}
%	\tasklabel{task:SastToFeedRefactoring}
%	
%	In \theTask, we can investigate how our way of doing static analysis can feed the semi automated refactoring process and tool. For instance we can search and detect specific properties of the source code, be they certain or uncertain (code smells), correlate 5them to find if a certain pattern is met and feed the refactoring.
%	We also can provide decision making information out of static analysis on "which code to refactor", on detected pre conditions as well as providing potential different options, based on semantics, to feed the refactoring.
%	\color{blue} \textbf{Limitation:} It will not be possible to modify the static analysis intermediate representation to adapt to refactoring specific needs.
%\end{Task}


%\begin{Task}
%	%\TaskTitle{Parallel Computation and Control for High-Level Modelling Languages} % for Aerospace and Automotive Industries}
%	\TaskTitle{Interoperability of the \TheProject{} Tool-Chain}
%	\TaskParticipant{UOD}{1}
%	
%	\TaskStart{1}
%	\TaskEnd{27}
%	\TaskResults{%
%		%\ref{del:model1}
%	}
%	\TaskHeader{}
%	\tasklabel{task:interoper}
%	
%	In \theTask, we will ensure the interoperability of the tools produced in WP2, WP3, WP4, WP5 and WP6. 
%	
%	\begin{itemize}
%		\item verification tools
%		\item self-healing
%		\item end-users tools
%	\end{itemize}
%\end{Task}




\begin{WPDeliverables}
\begin{compactitem}
\item \ref{del:met1} (Month 13): Report on the Methodology for Developing Secure Big Data Analytics on Private Clouds.
\item \ref{del:cs1} (Month 13) : Report on Initial \TheProject{} Coding Standards.
\item \ref{del:met2} (Month 25) : Report on the Methodology for Developing Secure Big Data Analytics on Public Clouds.
\item \ref{del:cs2} (Month 25) : Refined \TheProject{} Coding Standards.
\item \ref{del:met3} (Month 36) : Software on the Methodology for Developing Secure Big Data Analytics on Hybrid Clouds.
\item \ref{del:cs3} (Month 36) : Final \TheProject{} Coding Standards.
\end{compactitem}
\end{WPDeliverables}
\end{Workpackage}
