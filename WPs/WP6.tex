\addtocounter{wpno}{1}
\begin{Workpackage}{\thewpno}
\wplabel{wp:methodology}
\WPTitle{\wpname{\thewpno}}
\WPStart{Month 1}
\WPParticipant{USTAN}{1}


\begin{WPObjectives}
The objectives of \theWP{} are to:
\begin{compactitem}
\item Develop new refactorings that repair security vulnerabilities.
\item Develop new refactorings that rewrite source-code so that it conforms to a security-aware coding standard.
\item Develop new refactorings that introduce security-aware patterns into the source-code.
\item Develop new security-aware patterns.
\item Develop a new security-aware software engineering methodology for the development of secure big-data applications.
\item Develop a new C++ coding standard for the development of security-aware big-data applications.


\end{compactitem}
\end{WPObjectives}

\begin{WPDescription}
This workpackage will investigate new software engineering techniques and tools for the development of security-aware big-data applications. We will develop new end-user refactoring tool support in \ref{task:refactoring}; new security-aware software development patterns in \ref{task:patterns}, that will be provided to the user in the form of a new software library, with refactoring tool-support for their introduction in \ref{task:refactoring}; a new security-aware software engineering methodology in \ref{task:methodology}, encapsulating new software engineering methods and techniques from all the tools developed as part of the \TheProject{} project. In \ref{task:standards}, we will develop a new C++ coding standard for the development of secure big-data applications in C++. Finally, in \ref{task:interoper}, we will enable interoperability between the tools developed in the other workpackages to form a coherent software tool-chain and methodology for the end-user.


\end{WPDescription}

\begin{Task}
%\TaskTitle{Parallel Computation and Control for High-Level Modelling Languages} % for Aerospace and Automotive Industries}
\TaskTitle{Refactorings for Secure Big-Data Applications}
\TaskParticipant{UOD}{1}

\TaskStart{1}
\TaskEnd{27}
\TaskResults{%
%\ref{del:model1}
}
\TaskHeader{}
\tasklabel{task:refactoring}

In  \theTask{}, we will develop semi-automatic refactorings to support the end-user by providing tool-support through the  secure-aware programming methodology from \ref{task:methodology}. These refactorings will operate at the program source level, where a refactoring will be implemented as a source-to-source program transformation. 

The refactorings themselves will be developed based on a set of formal refactoring rules (which will be defined in terms of the refactoring's pre- and post-conditions, together with a set of transformation rules). The output of the refactoring will be an equivalent application, with increased security, decreased vulnerability and higher conformance to the security-aware coding standards (from \ref{task:standards}).

The refactoring tool will be integrated into the ParaFormance refactoring tool, supporting both Eclipse and Visual-Studio for C++ applications. A prototype implementation for Java-based applications will also be produced. The results of this task will be used to refactor the use-cases in \ref{wp:usecases}.

This task will proceed in three phases. In the \emph{first} phase we will identify transformations for C++ applications that will introduce the patterns from \ref{task:patterns} into C++ applications, by defining their pre- and post-conditions, together with their formal transformations rules. In the \emph{second} phase, we will produce implementations of the refactorings defined in the first phase, implemented in the ParaFormance tool-chain, for C++. In this phase, we will consider the outputs from the Advanced Vulnerability Detection from \ref{task:vulnerability} and the Self-Healing from T XX (as defined in \ref{wp:vulnerability}) to provide the refactoring tool with the static analysis information required to discovery vulnerabilities and information on how to repair them.
 In the \emph{third} phase, we will produce refactorings that further refactor the code so that it conforms to the standards outlined in \ref{task:standards}, where possible. We will also produce prototype implementations of a tractable set of the refactorings from all the phases for Java-like languages.


%
%In particular, in this task we will produce new refactorings that will:
%
%\begin{itemize}
%	\item Repair vulnerable code, by taking the output of the Self Healing tooling from XX, 
%	\item Introduce security-aware patterns, based on the implementation of the patterns in Task X.
%	\item Rewrite source-code so that it conforms to the standard as defined in Task X.
%\end{itemize}
\end{Task}

\begin{Task}
	%\TaskTitle{Parallel Computation and Control for High-Level Modelling Languages} % for Aerospace and Automotive Industries}
	\TaskTitle{Security-aware Software Development Patterns}
	\TaskParticipant{UOD}{1}
	
	\TaskStart{1}
	\TaskEnd{27}
	\TaskResults{%
		%\ref{del:model1}
	}
	\TaskHeader{}
	\tasklabel{task:patterns}
	
In \theTask{} we will formalise and implement patterns of security that arise in typical big-data applications. As part of this process, we will investigate the relationship between common security threats and their source-level solutions, producing a high-level domain-specific language (DSL) for describing a set of design patterns for the end-user. This DSL will be made available to the end-user developer as a library of security-aware patterns and will be used by the refactoring tooling from \ref{task:refactoring}.
\end{Task}

\begin{Task}
	%\TaskTitle{Parallel Computation and Control for High-Level Modelling Languages} % for Aerospace and Automotive Industries}
	\TaskTitle{New Coding Standards for Security-Aware Applications}
	\TaskParticipant{UOD}{1}
	
	\TaskStart{1}
	\TaskEnd{27}
	\TaskResults{%
		%\ref{del:model1}
	}
	\TaskHeader{}
	\tasklabel{task:standards}
	
	In \theTask, we will develop new C++ coding and security standard to facilitate programming of secure big-data applications. The task will proceed in two phases. First, we will identify those properties that arise in security-aware applications, such as secrecy of variables, branching vulnerabilities, etc.; secondly, we will implement a new security-aware programming standard, encapsulating the properties from the first phase, to create a new C++ coding standard. 
\end{Task}

\begin{Task}
	%\TaskTitle{Parallel Computation and Control for High-Level Modelling Languages} % for Aerospace and Automotive Industries}
	\TaskTitle{A Methodology for the Development of Secure Applications}
	\TaskParticipant{UOD}{1}
	
	\TaskStart{1}
	\TaskEnd{27}
	\TaskResults{%
		%\ref{del:model1}
	}
	\TaskHeader{}
	\tasklabel{task:methodology}
	
	In \theTask, we will produce a new software engineering methodology for the development of secure-aware big-data applications. In this task, we will look to assess the suitability of the tools and techniques developed within the project to support different software development models. 
	In the first phase of this task, we will assess existing  software engineering methods and their applicability to develop secure-aware applications for big data.
	In the second phase of this task, we will assess the applicability of the individual tools and techniques developed on the \TheProject{} project, and their applicability to a secure-aware software methodology. 
\end{Task}

\begin{Task}
	%\TaskTitle{Parallel Computation and Control for High-Level Modelling Languages} % for Aerospace and Automotive Industries}
	\TaskTitle{Interoperability of the \TheProject{} Tool-Chain}
	\TaskParticipant{UOD}{1}
	
	\TaskStart{1}
	\TaskEnd{27}
	\TaskResults{%
		%\ref{del:model1}
	}
	\TaskHeader{}
	\tasklabel{task:interoper}
	
	In \theTask, we will ensure the interoperability of the tools produced in WP2, WP3, WP4, WP5 and WP6. 
	
	\begin{itemize}
		\item verification tools
		\item self-healing
		\item end-users tools
	\end{itemize}
\end{Task}




\begin{WPDeliverables}
  \begin{compactitem}
    \item XX
%\item \ref{del:model1} (Month 10): Report on Initial Block-Diagram Modelling, Patterns and Code Synthesis
\end{compactitem}
\end{WPDeliverables}
\end{Workpackage}
