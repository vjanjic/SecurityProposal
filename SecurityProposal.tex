    % !TeX encoding = UTF-8
\documentclass[a4paper,11pt]{article}

\newcommand{\project}[1]{\textbf{#1}\xspace}
\newcommand{\SECURITY}{\project{Security - Digital Fortress}}
\newcommand{\TheProject}{\SECURITY}

%\ifdefined\final
%\else
%\newcommand{\final}{}
%\fi

%% Included Packages
%\usepackage{a4wide}
\usepackage[cm]{fullpage}
\usepackage{fancyhdr}
\usepackage{txfonts}
\usepackage{path}
\usepackage{url}
%\usepackage{multicol}
\usepackage{graphicx}
\usepackage{lscape} % for landscape
\usepackage[official,right]{eurosym}
\usepackage{numbering}
\usepackage{array}
\usepackage{paralist} % for compactitem/compactenum/... environments
\usepackage{xspace} % to allow automatic spacing of the project acronym but still observing punctuation.
% \usepackage{hyperref}
\usepackage{natbib} % squeeze biblio entries
\setlength{\bibsep}{0.0pt}
\usepackage{titlesec}
\usepackage{wrapfig} % Wrap text around figures
\usepackage{microtype} % Fix underfull/overfull text boxes

%% For file splitting
\usepackage{pageslts} %  theCurrentPage
\usepackage{calc} % theCurrentPage+1
\usepackage{blindtext} % Lorem ipsum dolor ...

\newcounter{FirstPage} % counter, for first page number
\setcounter{FirstPage}{1} % Set FirstPage to page 1

\newwrite\BatchFile% Batch-Datei
\immediate\openout\BatchFile=split.sh

\newcommand{\Split}[1]{%
\immediate\write\BatchFile{pdftk\space \jobname.pdf\space cat \theFirstPage -\theCurrentPage\space output\space \jobname-#1.pdf dont_ask}% 
\setcounter{FirstPage}{\theCurrentPage+1}% set FirstPage to new first page
}

% For for-loops
\usepackage{pgffor}
\usepackage{forloop}

\usepackage{longtable}

%% Format workpackage descriptions in a standard way
\usepackage{Workpackage}


%% Use the standard commenting package: change this to final in the submitted version to remove comments
\ifdefined\final
\usepackage[final]{comments}
\else
\usepackage[draft]{comments}
\fi

%\usepackage{ifthen}
\usepackage{fp}

\newcommand{\hl}[1]{\fcolorbox{red}{red}{#1}}

% Euro symbol for financial information
\newcommand{\euros}{\euro{}}

%% Standard page formatting commands
\renewcommand{\thesection}{\arabic{section}}

% Set the page margins etc.  Superceded.
% \advance\topmargin by -1.3cm
% \advance\textheight by 1.3cm
% \advance\leftmargin by -12mm
% \advance\textwidth by 12mm
% \advance\columnsep by 4mm
\parindent= 0pt
\parskip=0.125\medskipamount
\raggedbottom
\pagestyle{empty}

% Fix awkward figures
\renewcommand{\textfraction}{0.1}
\renewcommand{\topfraction}{1.0}
\renewcommand{\floatpagefraction}{0.9}

% Include images from the images directory
\newcommand\includeimage[2][]{%
\IfFileExists{images/#2.pdf}{%
\includegraphics[#1]{images/#2.pdf}%
}
{
\IfFileExists{images/#2}{%
\includegraphics[#1]{images/#2}%
}
{%
\vspace{3in}%
}
}
}


\newcommand{\mydate}{\today}
\setcounter{page}{0}

%% Control table of contents - depth 1 = section, 2 = subsection
\addtocontents{toc}{\protect\setcounter{tocdepth}{2}}

%% Header and Footer
\cfoot{\thepage}
\lfoot{\TheProject}
\rfoot{\mydate}

\lhead{}
\chead{}
\rhead{}
\renewcommand{\headrulewidth}{0pt}
\renewcommand{\footrulewidth}{0.4pt}% default is 0pt

\pagestyle{fancy}

% Squeeze wasted space
\titlespacing{\paragraph}{%
  0pt}{%              left margin
  0.5\baselineskip  plus 0.1\baselineskip minus   0.1\baselineskip}{% space before (vertical)
  1em}%               space after (horizontal)

\titlespacing{\subsubsection}{%
  0pt}{%              left margin
  0.5\baselineskip  plus 0.1\baselineskip minus   0.1\baselineskip}{% space before (vertical)
  1em}%               space after (horizontal)

\titlespacing{\subsection}{%
  0pt}{%              left margin
  0.5\baselineskip  plus 0.1\baselineskip minus   0.1\baselineskip}{% space before (vertical)
  1em}%               space after (horizontal)

\titlespacing{\section}{%
  0pt}{%              left margin
  0.5\baselineskip  plus 0.1\baselineskip minus   0.1\baselineskip}{% space before (vertical)
  1em}%               space after (horizontal)
\sloppy

\renewcommand{\baselinestretch}{0.91}

\usepackage{mdframed}
\usepackage[dvipsnames]{xcolor}
\newenvironment{commentary}[1]{%
\begin{mdframed}[backgroundcolor=blue!15]
% \noindent\fcolorbox{black}{boxgray}{%
%    \parbox{\textwidth}{%
\textbf{#1}%
}%
{%
\end{mdframed}
\bigskip\bigskip
}

\newcommand{\horizontallinedef}[1]{{\vspace{6pt}\color{ProcessBlue}\hrule height#1pt\vspace{6pt}}}

%\newcommand{\horizontalline}{{\vspace{6pt}\color{ProcessBlue}\hrule height1pt\vspace{6pt}}}
\newcommand{\horizontalline}{\horizontallinedef{1}}

\newcommand{\boxpara}[1]{{#1}\newline{}\hspace*{\fill}\rule[0.4ex]{0.9\linewidth}{1pt}\hspace{\fill}{}}

\hbadness=10000
\hfuzz=8pt

\usepackage{textcomp}  % Copyright symbol

% Partners
\newparticipant{SA}{University of St Andrews}{USTAN}{UK}{USTAN}
\newparticipant{UOD}{University of Dundee}{UOD}{UK}{UOD}
\newparticipant{IBM}{IBM Israel Science \& Technology Ltd.}{IBM}{IL}{IBM}
\newparticipant{SOPRA}{Sopra-Steria Ltd.}{SOPRA}{UK}{SOPRA}
\newparticipant{SCCH}{Software Competence Centre Hagenberg}{SCCH}{AT}{SCCH}
\newparticipant{COGNIX}{Cognitive UX Ltd.}{COGNIX}{CY}{COGNIX}

\newcommand{\USTAN}{\SA}
\newcommand{\USTANnr}{\SAnr}
\newcommand{\USTANshort}{\SAshort}
\newcommand{\USTANshorter}{\SAshorter}
\newcommand{\USTANlong}{\SAlong}

% \newcommand{\ZMC}{\ZYD}
% \newcommand{\ZMCnr}{\ZYDnr}
% \newcommand{\ZMCshort}{\ZYDshort}
% \newcommand{\ZMCshorter}{\ZYDshorter}
% \newcommand{\ZMClong}{\ZYDlong}

% \newcommand{\HCB}{\textbf{HCB}\xspace}
% \newcommand{\HCBlong}{Hospital Cl\'{i}nic de Barcelona}

% Coordinator
\newcommand{\coordnr}{1}
\newcommand{\coord}{\participant{\coordnr}}
\newcommand{\coordshort}{\participantshort{\coordnr}}
\newcommand{\coordshorter}{\participantshorter{\coordnr}}

% Evaluation Partner
\newcommand{\evalnr}{\SAnr}
\newcommand{\eval}{\participant{\evalnr}}
\newcommand{\evalshort}{\participantshort{\evalnr}}
\newcommand{\evalshorter}{\participantshorter{\evalnr}}

% Dissemination Partner
\newcommand{\dissemnr}{\SAnr}
\newcommand{\dissem}{\participant{\dissemnr}}
\newcommand{\dissemshort}{\participantshort{\dissemnr}}
\newcommand{\dissemshorter}{\participantshorter{\dissemnr}}


% Personalised comments for each author
\newcommand{\jbcomment}[1]{\comment{JB}{#1}} % by Juliana Bowles, USTAN
\newcommand{\vjcomment}[1]{\comment{VJ}{#1}} % by Vladimir Janjic, UOD
\newcommand{\cbcomment}[1]{\comment{CB}{#1}} % by Vladimir Janjic, UOD


%% Related Projects
\newcommand{\advanceproject}{\mbox{\textsc{Advance}}\xspace}
\newcommand{\embounded}{\mbox{\textsc{EmBounded}}\xspace}
\newcommand{\paraphrase}{\mbox{\textsc{ParaPhrase}}\xspace}
\newcommand{\rephrase}{\mbox{\textsc{RePhrase}}\xspace}
\newcommand{\science}{\mbox{\textsc{SCIence}}\xspace}
\newcommand{\hipeac}{\mbox{\textsc{HiPEAC}}\xspace}
\newcommand{\tacle}{\mbox{\textsc{Tacle}}\xspace}
\newcommand{\discovery}{\mbox{\textsc{Discovery}}\xspace}
\newcommand{\paraphrasing}{\mbox{\textsc{ParaPhrasing}}\xspace}
\newcommand{\paraformance}{\mbox{\textsc{ParaFormance}}\xspace}
\newcommand{\teamplay}{\mbox{\textsc{TeamPlay}}\xspace}
\newcommand{\TeamPlay}{\teamplay}


\begin{document}
\pagenumbering{arabic} % for pageslts

\begin{titlepage}

\begin{center}
{\Huge \textsc{\TheProject}}
\end{center}

\begin{tabular}{lp{5in}r}
\textbf{Title of Proposal:} & \textbf{Intelligent Security and Privacy Management} & \\[4ex] 
\textbf{Date of preparation:} & \textbf{\today} & \comment{}{$
$Revision: 0.0$ $}\\[4ex]
\textbf{List of participants} && \\[1ex]


\end{tabular}

%% Participants Table
\newcounter{p}
\begin{center}
\begin{tabular}{|l|p{5in}|l|l|}\hline
\textbf{Participant no} & \textbf{Participant organisation name} & \textbf{Country}\\ \hline 
1 (Coordinator) & {\sc \longparticipant{1}} \hfill (\shortparticipant{1}) & \country{1}  \\ \hline
\forloop{p}{2}{\value{p} < \theparticipant}{%
\thep & {\sc \longparticipant{\thep}} \hfill  (\shortparticipant{\thep}) & \country{\thep}  \\ \hline}%
\theparticipant & {\sc \longparticipant{\theparticipant}} \hfill  (\shortparticipant{\theparticipant})& \country{\theparticipant}  \\ \hline
\end{tabular}\end{center}

\tableofcontents

\end{titlepage}

% \newpage
\section*{Snags}

\subsection*{Overall}

\begin{itemize}
\item[ALL:]
Put in ideas of what your institution could contribute to the
proposal.
\end{itemize}

\subsection*{Section 1}

\begin{itemize}
\item[ALL:]
  Think about aims and objectives
\end{itemize}



\newpage
\section*{Snags completed}

\subsection*{Overall}
\newpage

\newpage


\pagenumbering{roman}

% ---------------------------------------------------------------------------
%  Section 1: Excellence
% ---------------------------------------------------------------------------

\pagebreak

\pagenumbering{arabic}
\setcounter{page}{2}

\subsection{Contributions from the Partners}

\subsubsection{University of St Andrews - Security Contracts and Verification}
\begin{itemize}
    \item Formal specification of security contracts for pieces of code
    \item Refactoring to introduce secure code in the applications
    \item Formal verification of properties of the code with regard to security contracts
    \item Automated verification and reasoning (model checkers, constraint solvers and theorem provers, sometimes used in combination), logics including distributed temporal logics 
\end{itemize}

\subsubsection{Dundee - Security for Big Data}
\begin{itemize}
    \item Identifying and addressing security risks in large-scale distributed databases, both open- and closed-source ones.
    \item Identifying and analysing security risks in distributed parallel processing.
    \item Security for machine-learning based big data analysis
    \item Security contracts for distributed parallel code
    \item Dissemination
\end{itemize}

\subsubsection{IBM - Vulnerability Detection and Data Fabrication}
\begin{itemize}
\item  Vulnerability Detection
   
We propose to extend the ExpliSAT symbolic execution technology to discover known security vulnerability patterns in C/C++ code and combine it with the tailored fuzzing techniques in specific areas for the purpose of assisting the symbolic execution engine to consistently make progress and thus overcome a known "path explosion" problem of symbolic interpretation technologies.
 
\item Data Fabrication for ML
 
We have recently started to explore a new and very challenging direction of synthetic data fabrication for improving robustness of ML models and AI-based applications. It includes  use cases like data enrichment (missing or not sufficient training data), poisoned data (malicious data that is used as part of training data set to cause machine learning model malfunction) and evasion attack (malicious data that causes malfunction of a trained model).
Most of the recent research work in academia in this field is done based on "unstructured data" like images, video or text. We believe that for real industry use cases fubrication of structured data to test and improve robustness of ML models is more relevant. Our idea is to combine our rule-based (CSP-based) data fabrication approach with machine learning techniques to fabricate synthetic data for ML use cases.
 
\item Data Fabrication Platform
 
It is actually our current DFP product that is used in SERUMS. It can be mentioned and used in the new proposal in combination with any of the two "new" direction listed above.
\end{itemize}

\subsubsection{SCCH - Privacy-Preserving AI}
\begin{itemize}
    \item Informational Privacy
    
    To constrain the information leakage from a data set, we motivate an information theoretic approach to  privacy where privacy is quantified by the mutual information between sensitive private information and the released public data.
    \item Optimal Privacy Secured Data Release Mechanism 
    
    A data release mechanism aims to provide useful data available while simultaneously limiting any reveled sensitive information. The data perturbation approach uses a random noise adding mechanism to preserve privacy, however, results in distortion of useful data and thus utility of any subsequent machine learning and data analytic algorithm is adversely affected. We introduce a novel information theoretic approach for studying privacy-utility trade-off suitable for different data types (including high-dimensional data, signals, and images) and for the cases with unknown statistical distributions. 
    
    \item Privacy Secured Knowledge Sharing
    
    While sharing of knowledge extracted from a relatively large set of labelled data owned by an organization with another organization owning a few or no labelled data, it is intended that 
    \begin{itemize}
        \item privacy of data is preserved;
        \item transferability of knowledge from source to target domain is evaluated for the design and analysis of transfer learning algorithms;
        \item privacy-transferability trade-off is optimized. 
    \end{itemize} 
    We aim at the development of techniques and tools for the study and optimization of privacy and transferability aspects of machine learning based AI systems. 
\end{itemize}
\subsubsection{Cognitive UX}
\begin{itemize}
\item AI-driven and eye gaze-driven behavioral authentication (Topic C and D)

We can bring in multi-factor authentication solutions and more specifically AI-driven behavioral authentication by adding another layer of security in the authentication process to verify the end-users based on their interaction behavior, and/or their eye gaze behavior. For doing so, we currently utilize and can also bring in the project a range of eye-tracking and wearable technologies.

\item Authentication-as-a-Service (Topic C and D)
We are interested to conduct research and extend our product (Cognitive Authentication), which offers an integrated user authentication solution that allows service providers to set their own password policies, authentication types, get insights from end-users' interaction data, etc.

\item User experience and usability evaluations and activities
\item System integration and testing
\item Dissemination activities on Usable Security, HCI, Authentication, Eye-tracking, and Intelligent User Interfaces
\end{itemize}

\subsubsection{Sopra-Steria Limited}

As a commercial partner we want to start investigation into the following areas:
\begin{itemize}
    \item Finance (open banking for CMA9) as data source for a universal banking solution that supports a secure banking interaction with appropriate trust and privacy as expected.
    \item Space-based open Internet capabilities using low Earth orbit satellites and therefor using an open meshed network for communication.
    \item Distributed healthcare (IoT) to support remote healthcare and intervention via open network technology.
\end{itemize}

We propose three business areas to research the develop of automated tools for validating the security and privacy of data, underlying systems, exposed online services used to provision these services to citizens. We want to introduce a practical solution for zero-trust architecture of the system while preserving the ever-increasing use of open and common internet-based communication channels between businesses and citizens without losing the trust and security that each citizen wants.

Sopra-Steria Limited will supply an insight into the real-world case studies that we could use to prove the research is achieving the proposed outcomes.

We also hold various advance technical capability that can be used to support 
97
 the general project to ensure data engineering, machine learning and data science skills are available for use by the projects research life-cycle.


\pagebreak
\section{Excellence}
"The world's most valuable resource is no longer oil, but data" is the now world-famous quote from The Economist that signifies the importance of data analytics in modern world. Dealing with and making sense of the vast amount of data available to us through the variety of sources is an imperative which made data analysts and statisticians the most desirable professions in 2019\footnote{Quote for data analysts and statisticians being the best professions in 2019}. Modern data analysis is necessarily distributed in nature and, for most of the part, based on advanced AI techniques such as deep learning. At the same time, digitisation of the modern world presents completely new challenges for cybersecurity. Highly-distributed world of AI-based big data analytics means there are plenty more possibilities from vulnerabilities, coming both from the way big data is stored (usually in flat files or distributed databases) and also in the way it is processed, with multiple distributed agents interacting and exchanging potentially sensitive information. Security mechanisms that protect the sensitive data in this settings are still not advanced enough to provide enough trust in the whole data analytics process. \emph{The main goal of the \TheProject{} project is to significantly increase the trust in modern data analytics systems by combining formal and practical aspects of code development and analysis.} Our methodology will be based on formally specified, machine readable and verifiable \emph{security contracts} that will specify the security aspects of the sensitive parts of distributed code.

%The emergence of big data, supported by distributed machine-learning based data analytics tools and techniques, presents new and unforeseen challenges for data security and privacy. Security vulnerabilities can come both from the individual components of the distributed data-analytics systems, as well as from the undeterministic way in which these components may interact. The Digital Fortress project aims to develop a novel methodology and the associated tool-chain for implementing secure distributed data processing applications. We will tackle the problem both from theoretical and practical aspect, implementing novel formally-verifiable security contracts that will specify security properties of the distributed code, and supporting this contracts with a novel code refactoring tools and dynamic vulnerability detection techniques. The focus on the project will be on machine-learning based analytics techniques, addressing the issues that come from both storage and processing of the data and learning models. As an additional layer of security, we will implement novel integrated behaviour-based user authentication schemes.


\subsection{Aims and Objectives}
\label{sect:objectives}

\eucommentary{\emph{Describe the specific objectives for the project1, which should be clear, measurable, realistic and achievable within the duration of the project. Objectives should be consistent with the expected exploitation and impact of the project (see section 2).}}


The specific \emph{aims} and \emph{objectives} of the \TheProject{} project are:

\begin{description}
\item[Aim 1:] To produce novel ways for formal specification and verification of security properties of
  distributed data analytics systems, addressing all aspects of the big-data management, from collection of
  data and storage to the end analytics;

\item[Aim 2:] To build on the existing and develop new efficient techniques for identifying generic code patterns that
  represent known security vulnerabilities and data leakages in the distributed AI-based systems;

\item[Aim 3:] To build additional security layers on top of the existing systems, based on novel, intelligent and secure authentication methods;

\item[Aim 4:] To integrate the \TheProject{} tools and techniques into a coherent self-healing methodology for establishing
  security properties of the existing distributed code, as well as for developing new provably-secure code;

\item[Aim 5:]  To demonstrate the applicability of the \TheProject{} tools and
 methodology in building secure real-world application from the medical, aerospace and
 banking domains and to promote their long-term uptake by building a sustainable user community,
 covering both experts in security and normal application developers;

\end{description}

The corresponding concrete \emph{objectives} are: 
\begin{description}

%\item[Objective 1:] To develop a novel concept of machine-readable \emph{security contracts}
%  that will be used to specify precisely security properties for parts of the C++ code, and to develop methods
%  for their formal specification and verification in the new and existing C/C++ code. \emph{This fulfils part of \textbf{Aim 1} and \textbf{Aim 4}}.
%  \comment{CB: It's not clear to me what machine-readable actually means, as, arguably, any arbitrary stream of bytes in a file is machine readable :D I think this needs clarified. There's a worry here that this is the same as what we have done on TeamPlay, where we also developed proofs, contracts and certificates for security, energy and time. What is the novelty here? There should probably be some tie-in, perhaps, to the work done on TeamPlay, CSL, for instance, or the proof system developed there. I think this needs to tie in with the refactoring tooling. One novel approach here is the refactorings... perhaps new refactorings that transform the application in such a way that it can meet a user-defined security specification or contract?  }
%  \comment{JB: If we are treating security contracts as a notion at several levels of abstraction and we can go between them, this would potentially be quite different from TeamPlay? It should not just be at the code level... }
 
\item[Objective 1:] To develop a novel concept of \emph{secure} patterns for distributed data processing and to support these
patterns with strictly defined and formally verifiable multi-level \emph{security contracts}, raising the level of abstraction
for developing secure distributed applications and providing formal guarantees about their security properties; 
\emph{This fulfils part of \textbf{Aim 1}}. 
%  \item[Objective 1:] To develop a novel techniques for verification of security properties of the application code,
%  specified in a formal way by machine-readable \emph{security contracts}, giving strict guarantees about
%  the safety of runtime behaviour of undeterministic large-scale distributed data analytics; \emph{This fulfils part of
%    \textbf{Aim 1}}.

\item[Objective 2:] To identify security and privacy risks and the methods for their mitigation
  in the state-of-the-art distributed machine-learning-based
  data analytics platforms, both in the storage of big data using distributed databases and in the data processing
  frameworks, thus guiding the verification of security contracts in the distributed settings. \emph{This fulfils parts
    of \textbf{Aim 1}, \textbf{Aim 2} and \textbf{Aim 4}}.
  
\item[Objective 3:] To develop novel efficient technologies, based on combination of symbolic execution and fuzzing
  techniques, for discovering known and identifying new security vulnerabilities in the C/C++ code. \emph{This fulfils
    part of \textbf{Aim 2} and \textbf{Aim 4}}.

\item[Objective 4:] To develop new techniques for identifying and controling data leakage in distributed machine-learning
  based data analytics algorithms, as well as techniques for secure transfer of knowledge from machine-learning models
  built on big data sets. \emph{This fulfils parts of \textbf{Aim 2} and \textbf{Aim 4}}.

\item[Objective 5:] To augment the existing security layer of distributed data analytics application with AI-driven behavioural authentication models, implementing Authentication-as-a-Service (AAAS) technology. \emph{This fulfils parts of \textbf{Aim 3} and \textbf{Aim 4}}.

\item[Objective 6:] To develop a novel methodology, based on security contracts, vulnerability detection/mitigation,
  privacy preserving and secure authentication, and supported by semi-automated code refactoring techniques for
  increasing security of the existing and developing secure new distributed data-analytics algorithms. \emph{This fulfils
    part of \textbf{Aim 4}}.

\item[Objective 7:] To demonstrate that the \TheProject{} tools and techniques allows improved security and reduced
  information leakage in distributed learning-based data analytics code, using a variety of real-world security-sensitive
  use cases from healthcare, banking and aerosepace application domains. \emph{This fulfils part of \textbf{Aim 5}};

\item[Objective 8:] To build a sustainable user community involving a variety of stakeholders that will ensure the long-term
  uptake, development and commercial success of the tools and techniques developed over the course of the \TheProject{}
  project. \emph{This fulfils part of \textbf{Aim 5}}.

\end{description}

%\subsubsection{Provisional list of workpackages:}
%\begin{itemize}
%\item WP2: Specification and Verification of Security Contracts for Distributed Data Analytics (USTAN)
%\item WP3: Security and Privacy of Big Data Storage and AI-based Processing (UOD/SCCH)
%\item WP4: Vulnerability Detection (IBM)
%\item WP5: AI-Driven Behavioural Authentication (COGNITIVE UX)
%\item WP6: Refactoring-Based Methodology for Secure Application Development (USTAN)
%\item WP7: Use cases (SOPRA)
%\item WP8: Dissemination and Exploitation (UOD)
%\end{itemize}

\label{sect:objs-detailed}

\TOWRITE{UOD and USTAN}{This needs to be done after
the objectives are defined above   We need about 1-2 paragraphs
per objective, just to flesh it out.}

\subsubsection{Detailed Description of the Objectives}

\subsubsection*{Objective 1: Secure Patterns for Distributed Data Processing}
%subsubsecton*{Security Contracts for Formal Specification of Security Properties}
\vspace{-6pt}
Parallel patterns are a well-established abstraction for programming complex parallel and distributed systems, adopted by a number of major IT companies such as Intel and Microsoft. They allow structuring of the end-user code at a high level and make formal reasoning about the code much easier, as they can provide additional information about the semantics of the code to the underlying formal models. On the other hand, formal treatment of the security properties of the code requires a multilayer approach, where a high-level human-understandable description of the properties of the code is transformed into precise, machine-readable description that can be used for formal reasoning. In the \TheProject{} project, we will develop novel \emph{secure} patterns for distributed large-scale data processing that will, in addition to performance considerations, give strict guarantees about their security properties. This will be achieved by supporting these patterns with both \emph{security certificates}, which will describe in natural way the properties of the code, and the associated lower-level, machine-readable \emph{security contracts} that will be appropriate for automated reasoning. We will also develop mechanisms to automatically prove the properties specified in the security contracts for pieces of C++ and Java code and also mechanisms to translate security certificates into security contracts and vice versa. The end users will then be able to choose a security-certified patterns suitable for their data analytics tasks, and will be presented with guaranteed security properties of the resulting code in an easily-understandable way. 
%Formal treatment of the security properties of the code requires a multilayer approach. On one hand, end users want to see a description of the security properties in a clear, high-level and human-understandable way. On the other hand, formal reasoning, including automatic proving of the properties, requires significantly lower-level approach. The language for formal reasoning needs to be concise and machine-readable. This necessitates different approach to describing properties at different levels. In the \TheProject{} project, we will develop a strict and precise notion of security properties at different levels of abstraction. We will develop a concept of \emph{security certificates}, which will describe in natural way the properties of the code. We will also develop \emph{security contracts}, which will be specified in low-level, machine-readable way appropriate for automated reasoning. We will also develop mechanisms to automatically prove the properties specified in the security contracts for pieces of C++ and Java code and also mechanisms to translate security certificates into security contracts and vice versa. The end users will then be able to specify the required properties using high-level language, and these properties will be automatically checked by lower-level mechanisms.
%\cbcomment{See above comments about similarities with teamplay. CSL is a source-level annotation language that allows the programmer to describe security propoerties in a clear high-level human-understandable way with certificates. Idris provides the proofs and formal reasoning and contracts. I'm also not keen on multiple languages. Is there a reason for both C++ and Java?}


\subsubsection*{Objective 2: Identification of Security and Privacy Risks in Machine-Learning Based Data Analytics}
\vspace{-6pt}

With the emergence of big data, many different technologies for both storing and processing this data have emerged. On the storage side, most commonly used tools are distributed databases and flat filesystems. While providing good reliability via fault tolerance, the security mechanisms used by databases and filesystems are still very basic, making them a security liability. On the data processing side, machine-learning has emerged as the prominent set of methods for making sense of and analysing the vast amount of data available. Distributed machine learning algorithms, however, present a whole set of new challenges, both in terms of security and in terms of privacy. Security risks and privacy leaks can come both from training and applying the machine learning models, as well as from the models themselves. In the \TheProject{} project, we will develop methods to identify known and unknown security risks coming both from the individual agents invovled in the process of storing and analysing big data, and from the interaction between different agents in distributed settings.


\subsubsection*{Objective 3: Symbolic Execution for Discovering Security Vulnerabilities}
\vspace{-6pt}
\vjcomment{IBM to write this part}

\subsubsection*{Objective 4: Identifying and Controlling Data Leakage in Distributed Machine Learning}
\vspace{-6pt}
\vjcomment{SCCH to write this part}

\subsubsection*{Objective 5: Intelligent User Authentication-as-a-Service}
\vspace{-6pt}

User authentication is a cornerstone of security in modern computing systems. Two important quality dimensions of an effective authentication system relate to its \textit{security} and \textit{usability aspects}. The security level determines its strength against adversary attacks, whereas usability levels are commonly determined by task execution efficiency and effectiveness. However, user authentication has become a complex and time-consuming task for any organization today due to constant cybersecurity threats and strict regulatory and new directives such as PSD2 that require strong, multi-factor authentication solutions. At the same time, end-users demand a seamless authentication experience. In the \TheProject{} project, we will develop an Authentication-as-a-Service technology allowing developers to easily integrate, deploy and manage their preferred authentication methods depending on custom requirements and policies. From the users’ point of view, the authentication technology will provide a fluid authentication experience based on state-of-the-art authentication methods such as usable, single-touch user approvals, as well as intelligent and continuous authentication based on users' interaction and eye gaze data analysis. In addition, we will develop data analytics and reporting services aiming to provide intelligent insights from user interactions and eye gaze behavior data allowing developers to deliver personalized security experience and thus increase user acceptance and trust.

\subsubsection*{Objective 6: Methodology for Deveopment of Secure Applications}
\vspace{-6pt}

It will be necessary to bridge the gap between the relatively low-level techniques for verification of the security properties of the code, with the focus on machine-learning based analytics of big data and including the techniques for identification of vulnerabilities and high-level of abstraction that the end users want to have the code and security properties presented. In the \TheProject{}, we will build onto our previous work on \emph{software refactoring} as a part of tool-chains for software engineering of the parallel and distributed applications and develop a refactoring-based programming methodology for developing secure applications. Our methodology will integrate various tools developed over the course of the project, including analysis and automated proving technqiues, security vulnerabilities detection tools and analysis of security and privacy of machine learning models, into a coherent tool-chain that will be supported by the code refactoring technology. In this way, the programmer will be fully included in the development loop, having full control over the incremental transformations that will make their code more secure.

\cbcomment{I wonder if the refactoring is downplayed too much. If the proposal focussed more on the idea of the end-user, refactoring becomes vital to that. Also the idea of refactorings to transform programs to make them more secure. Perhaps meeting the specifications/contracts that are described in Objective 1. Also, maybe the refactoring use some kind of security pattern in the rewriting, which would be a novelty. }

\subsubsection*{Objective 7: Demonstrating \TheProject{} Tools and Technologies on Real-World Use Cases}
%  from automotive, machine learning and IoT domains.}
\vspace{-6pt}
We will validate the \TheProject{} methodology for developing secure applications, together with the associated tools and technologies, on realistic use cases taken from our partners at XX, XX and XX.
%In particular, the \TheProject{} technologies will be applied to the development
%of a \emph{completely new} distributed data analytics application by XX.
Consequently, all of the technologies that will be developed over the course of the \TheProject{} project will be extensively tested in a realistic setting, enabling us to identify new issues and security risks as they arise, and guiding our roadmap for future technological adoption. The use cases will demonstrate that we are able to deal with large-scale distributed data analytic in secure and privacy-preserving way, processing large volumes of data with increased trust in the safety of the operations.At the same time, by using user-guided refactoring-driven methodology, supported by a tool chain, we will reducing the cost of development, deployment and maintenance of the distributed systems.
%Thanks to our \emph{data fabrication} technology, we will be able to test parts of the system
%and a system as a whole on large volumes of synthetic, but realistic,
%data, both during development and during deployment.

\subsubsection*{Objective 8: Long-Term Uptake of \TheProject{} Technologies} 
\vspace{-7pt}
\TheProject{} aims to ensure long-term uptake of the technologies that it will develop by engaging with relevant user, developer and adopter communities, by following an Open Science approach, and by providing a roadmap for the future development and exploitation of the \TheProject{} technologies. We have built in specific user community building activities, including workshops, tutorials, webinars and training sessions, that will serve to actively promote the use of the \TheProject{} technologies. Whenever feasible, our software and results will be made publicly available in open source/open data repositories. We will actively engage with potential users from the chosen domains through exhibitions, demonstrations and presentations at existing events and conferences such as XX, through the annual EU ICT conference, through exploiting our excellent high-level contacts with the XX. Innovation Centres and other relevant organisations, through contacts with national government agencies, including XX and through other relevant expert networks.

\pagebreak
\subsection{Relation to the Workprogramme}

\eucommentary{Indicate the work programme topic to which your proposal relates, and
 explain how your proposal addresses the specific challenge and scope
 of that topic, as set out in the work programme.}

{\color{blue}{
    Specific Challenge:
    
 In order to minimise security risks, ICT systems need to integrate state-of-the-art approaches for security and privacy management in a holistic and dynamic way. Organisations must constantly forecast, monitor and update the security of their ICT systems, relying as appropriate on Artificial Intelligence and automation, and reducing the level of human intervention necessary.

Security threats to complex ICT infrastructures, which are multi-tier and interconnected, computing architectures, can have multi-faceted and cascading effects. Addressing such threats requires organisations to collaborate and seamlessly share information related to security and privacy management.

The increasing prevalence and sophistication of the Internet of Things (IoT) and Artificial Intelligence (AI) broadens the attack surface and the risk of propagation. This calls for tools to automatically monitor and mitigate security risks, including those related to data and algorithms. Moreover, storage and processing of data in different interconnected places may increase the dependency on trusted third parties to coordinate transactions.

Advanced security and privacy management approaches include designing,
developing and testing: (i) security/privacy management systems based
on AI, including highly-automated analysis tools, and deceptive
technology and counter-evasion techniques without necessary human
involvement; (ii) AI-based static, dynamic and behaviour-based attack
detection, information-hiding, deceptive and self-healing techniques;
(iii) immersive and highly realistic, pattern-driven modelling and
simulation tools, supporting computer-aided security design and
evaluation, cybersecurity/privacy training and testing; and (iv)
real-time, dynamic, accountable and secure trust, identity and access
management in order to ensure secure and privacy-enabling
interoperability of devices and systems.

\begin{itemize}
\item[(c)]: Advanced security and privacy solutions for end users or software developers

Proposals should develop automated tools for checking the security and privacy of data, systems, online services and applications, in view to support end users or software developers (possibly including developers of AI solutions) in their efforts to select, use and create trustworthy digital services. Proposals should address real application cases and at least one of the following services: automatic code generation, code and data auditing, trustworthy data boxes, forensics, certification and assurance, cyber insurance, cyber and AI ethics, and penetration testing.

The outcome of the proposal is expected to lead to development up to Technology Readiness level (TRL) 6; please see Annex G of the General Annexes.

The Commission considers that proposals requesting a contribution from the EU of between EUR 2 and 5 million would allow this specific challenge to be addressed appropriately. Nonetheless, this does not preclude submission and selection of proposals requesting other amounts.

Type of Action: Research and Innovation Action

\item[(d)]: Distributed trust management and digital identity solutions

With particular consideration to IoT contexts, applicants should propose and test/pilot innovative approaches addressing both of the following points: (i) distributed, dynamic and automated trust management and recovery solutions; and (ii) developing novel approaches to managing the identity of persons and/or objects, including self-encryption/decryption schemes with recovery ability. Proposals should address real application cases.

The outcome of the proposal is expected to lead to development up to Technology Readiness level (TRL) 5-6; please see Annex G of the General Annexes.

The Commission considers that proposals requesting a contribution from the EU of between EUR 3 and 6 million would allow this area to be addressed appropriately. Nonetheless, this does not preclude submission and selection of proposals requesting other amounts.

Type of Action: Research and Innovation Action
\end{itemize}

}}


\subsection{Concept and Approach}

\eucommentary{Describe and explain the overall concept underpinning the project. Describe the main ideas, models or assumptions involved. Identify any trans-disciplinary considerations;}

\vspace{-6pt}

\subsubsection{Challenges for Security and Privacy}
% massively-parallel heterogeneous systems}



\begin{itemize}
\item \textbf{XX.}
\end{itemize}


The challenges identified above require a new and radical
approach that tackles these issues in a coherent and
holistic way. 

\subsubsection{Achieving The \TheProject{} Vision}


\subsubsection*{Key assumptions}

\vjcomment{This may not be necessary.}

\TheProject{} makes a number of fundamental assumptions that will be tested and verified in the course of the project,
and which form the basis for a register of technical risk.  The most significant assumptions are:

\begin{enumerate}[{A}1)]
\item XX
\end{enumerate}

\subsubsection*{Transdisciplinary concerns}
 \vjcomment{Only if relevant}


\subsubsection{Positioning of the project}
\eucommentary{Describe the positioning of the project e.g. where it is situated in the spectrum from 'idea to application', or from 'lab to market'. Refer to Technology Readiness Levels where relevant.}

In line with the expectations of the XX call, \TheProject{} aims
to achieve overall Technology Readiness Level (TRL) 5-6 (``technology
validated in relevant environment (industrially relevant environment in
the case of key enabling technologies)'').

\begin{center}
  \begin{tabular}{|p{4.9in}|l|l|}
    \hline
    \textbf{Key Enabling Technology} & \textbf{Current TRL} & \textbf{Final TRL} \\
    \hline
     &  & \\
    \hline XX & TRL 5 & TRL 6 \\  
    \hline
  \end{tabular}
\end{center}

\noindent
The specific advances that will be made are described in more detail in Section~\ref{sec:novelty} (page~\pageref{sec:novelty}).


\subsubsection{Linked research and innovation activities}
\label{projects}

\eucommentary{Describe any national or international research and innovation activities which will be linked with the project, especially where the outputs from these will feed into the project;}

\vspace{-8pt}
\paragraph{\teamplay (ICT-779882).}
...

\vspace{-8pt}
\paragraph{\rephrase (ICT-644235).}
%\vspace{-12pt}

The \rephrase project aims to study the software engineering process as
a whole for heterogeneous parallel machines
using C++.  It considers neglected, but important, issues such as
effective testing, debugging, maintenance and quality assurance for
multicore/manycore machines, and involves major industry players such as IBM.
\emph{\TheProject will, however, substantially extend the work that has been done in \rephrase by:
\begin{inparaenum}[i)]
\item XX
\end{inparaenum}}

\vspace{-2pt}
\paragraph{\paraphrase (ICT-288570).}
%\vspace{-12pt}
% \TOWRITE{CB,VJ}{Write about ParaPhrase} 
The \paraphrase project introduced a new structured design and implementation process for
heterogeneous multicore/manycore architectures, in which developers exploit a variety of
parallel patterns to develop component based applications in Erlang and C++. These
component based pattern-applications may then be re-mapped to meet the
application requirements and hardware availability. 

\subsubsection{Overall approach and methodology}

\eucommentary{Describe and explain the overall approach and methodology, distinguishing, as appropriate, activities indicated in the relevant section of the work programme, e.g. for research, demonstration, piloting, first market replication, etc.;}

\TheProject{} comprises XX technical work packages: WP2 on XX;
WP3 on XX;
The relationship between the project objectives and these workpackages is shown below

\vspace{-8pt}
\begin{center}
\begin{tabular}{|l|l|l|}\hline
\textbf{Objective} & \textbf{Purpose} & \textbf{Contributing WPs} \\\hline \hline
Objective 1 & XX & \textbf{WPXX}, \textbf{WPXX} \\\hline
\end{tabular}
\end{center}

\paragraph*{Work Programme for Objective 1.}

Objective 1 aims to build XX.

\paragraph*{Work Programme for Objective 2.}

\subsection{Ambition}

\eucommentary{Describe the advance your proposal would provide beyond the state-of-the-art, and the extent the proposed work is ambitious. Your answer could refer to the ground-breaking nature of the objectives, concepts involved, issues and problems to be addressed, and approaches and methods to be used.}

\subsubsection{Advances Beyond the State-of-the-Art}
\label{sec:novelty}
\label{sect:background}
\label{sect:state-of-the-art}

%\eucommentary{}


\TOWRITE{ALL}{Add sections on your own technologies/update
what's here.}

The \TheProject{} project will revolutionise XX by:
\begin{itemize}
\item XX
\end{itemize}

\subsubsection{Security}
\label{sect:background-first}
\label{sect:security}

Application security is becoming increasingly important in modern computer systems.
Current practice is for security to be ensured at runtime, using efficient monitoring
techniques coupled with dynamic runtime checks such as ISR~\cite{isr}, ASLR~\cite{aslr}
and CFI~\cite{cfi}. These methods are, however, expensive in terms of computational
complexity and energy consumption. As most of the security vulnerabilities come from
code defects, bugs and logic flaws, the most cost-effective way to ensure security is
to follow the secure code best practices~\cite{OWASP} and eliminate vulnerabilities
before code is deployed. This is usually done with static and dynamic code analysers.
Static code checkers, such as Appscan Source~\cite{AppScan} and Coverity~\cite{Coverity}
%and Klocwork~\cite{Klocwork}, 
are based on techniques for quality checking that
have been extended to cover security vulnerabilities in the code. The main drawback of 
these methods is their high rate of false positives they produce. Techniques based on
formal methods, such as model checking and symbolic interpretation, do not report false 
positives, but have potential problem with scaling. AFL~\cite{AFL} uses use fuzz testing 
to dynamically profile the code to expose potential security vulnerabilities. 
None of the above tools includes special algorithms for detecting vulnerabilities 
resulting from parallel execution. Parallelism, due to undeterministic nature of the
application execution, presents many additional problems from the security perspective. These problems have to be dealt with efficiently
in future systems.


Including third party libraries into the application code presents additional
security risks, as these libraries may include intentional or unintentional
vulnerabilities resulting in unwanted behavior or data leakage of sensitive 
information. For example, an open source JavaScript library used in a website 
may contain malicious code that collects data and sends it to the third party. 
%Another example is binary packages which contain vulnerability which is detected by crackers and compromise users??? machines. 
The state of the art solutions to this problem is to use security testing, static 
analysis and dynamic analysis, such as sandboxing~\cite{jsand}, to detect 
vulnerabilities and malicious packages before integrating them into 
the application code. % is proposed to detect malicious JavaScript code, in addition, they 
%suggest wrapping the components to control access to security sensitive operations. 
In~\cite{Cova}, a combination of anomaly detection and emulation systems is built to detect malicious code. OWASP~\cite{OWASP} presents the risks in using third party 
packages, how to determine if these packages are vulnerable or contain malicious code 
and how to deploy these packages. %This problem is not met only in JavaScript packages, it is in many other 3rd party packages, especially, Android 3rd party libraries. 
Other possible solutions to the problem of integrating third party libraries are
based on analysis of the libraries, building control flow graph, looking for 
copies of buggy code, and profiling packages~\cite{Hanna, XinSun}.%~\cite{Hanna, XinSun, Nora, Backes}.
These solutions miss many vulnerabilities or malicious dormant code as a result of 
obfuscation, packing, and the ease in rebuilding new variants of the malicious code, 
or just the rapid release of new buggy version of packages.

There are cases in which it is not beneficial to fix a vulnerability in the 
software or in a third party code integrated into the software. For example, 
in cases where the patch is hard to deploy or where fixing the vulnerability 
has significant impact on performance. In fact, 99\% of attacks are believed to exploit known and fixed vulnerability~\cite{GartnerVulnerability} for which the fix was not deployed. One famous example is the WannaCry attack that was based on a fixed and updated CVE.
To protect the software in these cases,
\emph{virtual patching} may be used. %Virtual patching is a security policy 
%enforcement layer which prevents the exploitation of a known vulnerability. 
A virtual patch is a set of rules that implements complex logic to prevent 
malicious traffic from reaching the application. 
%The virtual patching mechanism will usually be integrated in the organization firewall. 
The challenge is to come up with a set of accurate rules that filter 
exactly the malicious inputs without blocking other inputs. This work is 
currently done manually, and it would heavily benefit from integrating a 
predictive model to automatically detect inputs that may utilize the 
vulnerability.




\TheProject{} will advance the state-of-the-art of security of parallel code in the following directions:
\begin{itemize}
\item XX
\end{itemize}

\subsubsection{Use Cases}
\label{sect:applications}
\label{sect:background-last}

\subsubsection{Key Technologies Used in the \TheProject{} Project}
\label{sect:key-technologies}

\paragraph{\SCCHshort{} Machine Learning Tools.}
\label{sec:mlpp}
\emph{mlpp} is a C++ library with interfaces to several other languages. It 
is currently under heavy development. So far, it contains algorithms for 
linear regression, time series analysis, feature selection and causal 
dependency discovery. Its focus is on building interpretable models. It is hosted 
on SourceForge (\url{https://sourceforge.net/projects/ml-pp/}), and released 
as Open Source (GPL license); paid licenses for commercial use will be provided
with the first mature release.

\paragraph{\IBM{} Security Vulnerability Detection Technology.}

IBM developed vulnerability detection technologies based both on static and dynamic methods. IBM developed Beam for static analysis of C/C++ code and extended it to look in particular for security vulnerabilities with a low ratio of false positive. In addition, IBM developed the ExpliSAT tool which uses symbolic interpretation for more precise program analysis including checks for security vulnerabilities such as buffer overflow detection. During the \rephrase{} project, ExpliSAT was extended to analyse parallel programs. To complement the static methods IBM uses also Fuzz testing technology that detects security vulnerabilities dynamically using genetic algorithms. In order to achieve accurate and scalable security vulnerability detection, IBM is now working on combining the static and dynamic analysis vulnerability detection tools.     

\paragraph{\SAshort{} \paraformance Refactoring and Code Analysis Tools.}


The \paraformance Refactoring tool was originally developed for the EU FP7 \paraphrase{} project, and later extended and enhanced in both \rephrase{} and the Scottish Enterprise innovation project, \paraformance{}. 
\paraformance is a tool-chain designed to \emph{democratise} parallel programming by allowing software developers to quickly and easily write parallel software. \paraformance enables software developers to find the sources of parallelism within their code, automatically (through user-controlled guidance) through a process of \emph{pattern discovery}. \paraformance also offers refactoring support to allow parallel patterns to be introduced directly into the source code of the application, inserting the parallel business logic. \paraformance also has integrated safety checking features, to not only enable  the parallelised code to be thread-safe, but also the checking of sequential code, eliminating potential sources of parallelism errors that occur, such as race conditions and deadlocks. Finally, \paraformance is able to repair some of the parallelism errors detected by the safety checking to automatically make the application thread safe. \paraformance will be extended in \TheProject{} by adding safety checks for the distributed code, in addition to the existing mechanisms for checking the code that is executed on shared-memory machines. We will also extend the tool with features to check for \emph{security} of the code. 


\subsubsection{Innovation Potential}
\label{sec:innovationpotential}
\label{innovationpotential}

\eucommentary{Describe the innovation potential which the proposal represents. Where relevant, refer to products and services already available on the market. Please refer to the results of any patent search carried out.}

\clearpage
\section{Impact}
\label{sec:impact}

\TODO{Look at this once the rest of the project is together.}

\eucommentary{Describe how your project will contribute to:\\
o the expected impacts set out in the work programme, under the relevant topic;\\
o improving innovation capacity and the integration of new knowledge (strengthening the competitiveness and growth of companies by developing innovations meeting the needs of European and global markets; and, where relevant, by delivering such innovations to the markets;\\
o any other environmental and socially important impacts (if not already covered above).}


\subsection{Expected Impacts}

%\eucommentary{

{\color{blue}{
  Expected Impact:
In the short term, project outcomes should make relevant contributions to the following:
\begin{itemize}
\item reduced number and impact of cybersecurity incidents;
\item efficient and low-cost implementation of the NIS Directive and General Data Protection Regulation;
\item effective and timely co-operation and information sharing between and within organisations as well as self-recovery;
\item availability of comprehensive, resource-efficient, and flexible security analytics and threat intelligence, keeping pace with new vulnerabilities and threats;
\item availability of advanced tools and services to the CERTs/CSIRTs and networks of CERTs/CSIRTs;
\item an EU industry better prepared for the threats to IoT, ICS (Industrial Control Systems), AI and other systems;
\item self–recovering, interoperable, scalable, dynamic privacy-respecting identity management schemes.
\end{itemize}

  In the medium to long term, project outcomes should make relevant contributions to the following:
\begin{itemize}
\item availability of better standardisation and automated assessment frameworks for secure networks and systems, allowing better-informed investment decisions related to security and privacy;
\item availability and widespread adoption of distributed, enhanced trust management schemes including people and smart objects;
\item availability of user-friendly and trustworthy on-line products, services and business;
\item better preparedness against attacks on AI-based products and systems;
\item a stronger, more innovative and more competitive EU cybersecurity industry, thus reducing dependence on technology imports;
\item a more competitive offering of secure products and services by
  European providers in the Digital Single Market.
\end{itemize}
}}

%}


The table below summarises how the \TheProject{} project will achieve impact in the areas expected by the Work Programme.


\begin{longtable}{|p{125pt}|p{320pt}|}%\hline

\hline \textbf{Expected impact}&

\textbf{How will \TheProject{} achieve this impact?}\\\\ \hline
\endfirsthead

\multicolumn{2}{c}%
{{\bfseries \tablename\ \thetable{} -- continued from previous
page}} \\ \hline \textbf{Expected impact}&

\textbf{How will \TheProject{} achieve this impact?}\\\\ \hline
\endhead

\hline \multicolumn{2}{|r|}{{Continued on next page}} \\ \hline
\endfoot

\hline \hline
\endlastfoot

\vspace{10pt}
\\
 \hline
\end{longtable}

%\draftpage
% \subsubsection*{Improving Innovation Capacity}
%\draftpage

\pagebreak
\paragraph*{Improving Innovation Capacity}
\noindent

\paragraph*{Societal Impact.}
\noindent
\subsubsection*{Possible Barriers to Achieving the Expected Impacts and Associated Mitigations}

\newcounter{barrier}

\begin{longtable}{|p{125pt}|p{320pt}|}%\hline

\hline \textbf{Possible Barrier}&

\textbf{Mitigation}\\ \hline
\endfirsthead

\multicolumn{2}{c}%
{{\bfseries \tablename\ \thetable{} -- continued from previous
page}} \\ \hline
 \textbf{Possible Barrier}&

\textbf{Mitigation}\\ \hline
\endhead

\hline \multicolumn{2}{|r|}{{Continued on next page}} \\ \hline
\endfoot

\hline \hline
\endlastfoot


\addtocounter{barrier}{1}
\noindent
\emph{Barrier \thebarrier.}
\par \emph{Barrier 1:}

&
\noindent
XX
\end{longtable}

%\draftpage
\subsection{Measures to Maximise Impact}

The impact of \TheProject{} will be maximised by:
\begin{inparaenum}[i)]
\item
  XX
\end{inparaenum}
%
The following section describes these measures in more detail.

\subsection{Dissemination and Exploitation of Results}
\label{sect:dissemination}

\eucommentary{Provide a draft 'plan for the dissemination and exploitation of the project's results' (unless the work programme topic explicitly states that such a plan is not required). For innovation actions describe a credible path to deliver the innovations to the market. The plan, which should be proportionate to the scale of the project, should contain measures to be implemented both during and after the project.
Dissemination and exploitation measures should address the full range of potential users and uses including research, commercial, investment, social, environmental, policy making, setting standards, skills and educational training.
The approach to innovation should be as comprehensive as possible, and must be tailored to the specific technical, market and organisational issues to be addressed\\
o Explain how the proposed measures will help to achieve the expected impact of the project. Include a business plan where relevant.\\
o Where relevant, include information on how the participants will manage the research data generated and/or collected during the project, in particular addressing the following issues:\\
o What types of data will the project generate/collect? o What standards will be used? o How will this data be exploited and/or shared/made accessible for verification and re-use? If data cannot be made available, explain why.
o How will this data be curated and preserved?}



\subsubsection{Draft Dissemination Plan}

We will focus on propagating our results both to the computer science
research community and to potential users of the \TheProject{} technology.
We will do this through a mixture of high-quality publication, presentations
and direct engagement with the user community.
%
The main research communities that we expect to target are:
XX
% 
We anticipate that the primary users of our technology will be:
XX.

\paragraph{Scientific Publications:}  The main routes to good scientific dissemination are
through peer-reviewed publication, and through presentation of results at key scientific events.
\TheProject{} partners will therefore aim to produce high-quality \emph{peer-reviewed
research publications} in relevant leading
conferences, technical workshops and journals.
We will build on the existing good publication records of the \TheProject{} partners,
aiming to produce a sizeable volume of good quality publications in the course of the project. 
%
\noindent
The conferences that we propose to target include:
\vjcomment{Update these.}

\begin{quote}
\textbf{ICSE:} International Conference on Software Engineering;
 \end{quote}

\noindent
We also propose to target relevant high-impact journals such as:
\begin{quote}
\emph{the ACM Transactions on Software Engineering and Methodology (TOSEM)}, 
\end{quote}

\paragraph{Project Web site:}  
A crucial component of the \TheProject{} dissemination strategy is a
high-quality project website. The public section will provide ample
and consistent information about all aspects of the \TheProject{}
project, with the goal of positioning the \TheProject{} website as a
prime information source for relevant scientific and technical
information.  The vast majority of the \TheProject{} deliverables are
public, and full access to these will be provided.  The web site will also contain lists of publications and links
to open-access repositories; copies of technical reports and white
papers; a news feed; technical documentation; downloadable software
and pre-installed virtual machines; video demonstrations; online
tutorials; information about project partners; copies of
presentations, podcasts and other material; data and results; plus
links to the Horizon 2020 programme in general and to related 
Horizon 2020 research projects in order to highlight the role played by \TheProject{} within the
broader EC research framework.

\paragraph{Project News Feed:}  We will set up open public mailing lists/twitter/facebook accounts that will
be used to communicate project news and results to interested parties, whether they are scientists, academics, developers
or the general public.  This news feed will be highlighted on the project web site.

\paragraph{Developer and User Community:} We will engage with the broader
developer and user communities by a series of focused activities
that will include the organisation of dedicated user community workshops, presentations at
 developer and other conferences, 
 the production of posters, delivering tutorials, staffing booths, and providing hands-on
 guidance in the use of our tools and technologies etc.  This direct engagement will be
 supported by the production of video demonstrations, training materials, tutorials and documentation that can
 be accessed through the project web site.  We will also aim to produce white papers and slide sets that
 can be used to explain the benefits of the \TheProject{} approach to prospective users, both developers
 and managers.

 \paragraph{Expert Working Groups and Networks of Excellence:}
In order to ensure good dissemination to the research and development
community, including industrial researchers, we propose to disseminate
our project results through the most relevant scientific/technological
networks and working groups, including HiPeac,%  the TACLe Cost Action on Timing Analysis, the Ercim DES Working Group, 
IFIP Working Groups 2.11 \& 10.3, and other EU and national projects,
as well as national groups such as the UK's network on manycore computing.
\SAshort{} is a full member of HiPeac and a charter member of IFIP working
group 2.11, and has been heavily involved in activities organised by these and other expert groups.
These provide a high-level interface between academic and industrial interests, and a valuable
melting pot for ideas and new technologies. 

\paragraph{Standardisation Committees:}
% \khcomment{We should mention any committees that we are members of. Remove this section if not relevant.}
This proposal is tied to Standards from the beginning, through the end and beyond. It leverages the key leadership position of several members, acting as senior leaders, officers, working group experts, proposal authors, collaboration with other industry and academic experts.
%% Doesn't make sense?
% Current C++ Standards only support CPU, though it is adding parallel programming support starting with C++ 11, enhanced through C++ 17 with parallelSTL. Yet it is still lacking many things that can support Heterogeneous computing, which has been already explored by SYCL and OpenCL. 
%
%

\paragraph{General Scientific and Technical Community:}
We will further engage with the general scientific and technical community
 by participating in relevant workshops, conferences and clustering events (including ones which
 will develop our technologies beyond the bounds of our own research communities), by engaging with the
 different EC-sponsored CORDIS information channels, by presenting at trade
 fairs, and through other dissemination activities.  We will use materials such as presentations,
 papers,  posters, demonstrations and the project web site to do this.


\paragraph{Education:} We will target 
the educational communities through the production of relevant educational materials, that will
also have a training benefit. Young researchers, software
 developers and application programmers will learn how to
 use our software technologies in their
 respective fields. This is aligned with recent EC
 initiatives on the subject such as ``Increasing the Attractiveness
 of Science, Engineering \& Technology Careers''.
 The academic consortium partners will integrate Bachelor,
 Master, Diploma, and PhD students into the \TheProject{}
 project whenever possible, e.g.~through PhD and MSc theses, student
 projects, academic courses, and research seminars. 
 We will also take advantage of opportunities to engage with broader
 audiences through guest lectures at other institutions and summer schools etc.
 These students will
 carry the methodology and techniques of the \TheProject{}
 project into their future work places in the ICT industry.
 In that way the \TheProject{} vision and the project
 results will disseminate into many research groups and
 companies working in the field of security. In order to disseminate security and the work of the consortium to the wider public a MOOC (massively open online course) will be developed to reach an audience far beyond the academic and industrial community.  

% \paragraph{General Communication:} We will adopt a good general communication strategy
% aimed at maximising the outreach of the \TheProject{} project to the public at large.  We will issue
% regular press releases describing relevant events and research results, conduct
% radio/TV interviews, adopt an open policy to disseminating our research results through use of appropriate
% repositories for publications and the project web site, use public media such as
% \emph{twitter} to communicate project results, engage with public lectures and seminars,
% write general articles for newsletters, newspapers etc. as described in \ref{wp:dissem}.

\begin{quote}

\emph{Although all these dissemination activities
will be centrally coordinated, the full and active participation of
all partners is expected. Key project representatives from
the various \TheProject{} academic and industrial
partners will also arrange specific meetings
with scientific, commercial, industrial, and/or
governmental representatives to facilitate public
engagement.}
\end{quote}

\subsubsection{Draft Exploitation Plan}
\label{sect:exploitation-plan}
\vspace{-12pt}

All partners will have the right to use foreground IPR for
the purposes of the project. In order to ensure that all
contributions are recognised, exploitation plans will be
shared with the consortium as a whole. Partners will not,
however, have the right to veto or delay exploitation
unless their own IPR is directly involved.


\horizontalline

\subsection*{Draft Exploitation Plan for \IBMshort{}}


\begin{wrapfigure}{R}{4cm}
\vspace{-1.4cm}
\hfill \includeimage[width=4cm]{logos/ibm.jpg}
\vspace{-0.6cm}
\end{wrapfigure}

\horizontalline

\subsection*{Draft Exploitation Plan for \SCCHshort{}}
\vspace{-6pt}

\begin{wrapfigure}{R}{3.6cm}
\vspace{-1.3cm}
\hfill \includeimage[width=4cm]{logos/SCCH.jpg}
\vspace{-0.8cm}
\end{wrapfigure}

\horizontalline

\subsection*{Draft Exploitation Plan for \SAshort{}}

\begin{wrapfigure}{R}{2cm}
\vspace{-1.4cm}
\hfill \includeimage{logos/st-andrews-logo.jpg}
\vspace{-0.9cm}
\end{wrapfigure}

\subsubsection{Knowledge Management and Protection}
\vspace{-12pt}

\eucommentary{Outline the strategy for knowledge management and protection. Include measures to provide open access (free on-line access, such as the 'green' or 'gold' model) to peer-reviewed scientific publications which might result from the project}

Before the project starts, all project partners will agree on explicit rules concerning IP ownership, access rights to any
Background and Results for the execution of the project and the
protection of intellectual property rights (IPRs) and confidential
information as part of the Consortium Agreement.
As part of the Consortium Agreement, in order to ensure a smooth
execution of the project, the project partners will agree to grant each other
royalty-free Access Rights to their Background and Results for the
execution of the project. The Consortium Agreement will define further
details concerning the Access Rights after the duration of the project 
with respect to Background and Results.

\paragraph{Dissemination and Communication:}
While fully taking into account issues of potential exploitation and IPR ownership by project partners
as governed by the Consortium Agreement,
the project aims to provide good general access to its research results.
Balancing access with cost, the project will therefore generally adopt a ``green'' model to open access for publications,
but has included funding to support targeted ``gold'' open access for key publications.
The academic partners all maintain suitable institutional repositories, which will allow public access to research papers produced in the
course of the project, perhaps with some moratorium.  Some publishers (e.g. the ACM) also provide links that allow
free access to their publications from authors' home pages, and this will be exploited wherever possible.
% y, where publisher charges are not excessive, the project will consider ``gold'' open access to key project publications.
%
Furthermore, and perhaps of most significance, the project website will provide free,
open and publicly searchable access to all the public deliverables, to technical reports, data and results, to software tools
and libraries, to white papers and also to all
the other non-confidential documents that are generated in the course of the project.  
% The material released in this way will represent the vast bulk of the scientific
% and technical output of the project.

%\subsubsection*{Intellectual Property Rights}

\pagebreak
\paragraph{IP Ownership.}

Results shall be owned by the project partner carrying out the work
leading to such Results. If any Results are created jointly by at least
two project partners and it is not possible to distinguish between the
contributions of each of the project partners, such Results, including
inventions and all related patent applications and patents, will be
jointly owned by the contributing project partners. In order to further
the competitiveness of the EU market, and to enhance exploitation of the
Consortium Results, each contributing party shall have full own freedom
of action to exploit the joint IP as it wishes, and further the goals of
the consortium. To promote this effort the contributing party will have
full own consideration regarding their use of such joint Results and will
be able to exploit the joint IP without the need to account in any way to
the other joint contributor(s).Further details concerning jointly owned
Results, joint inventions and joint patent applications will be addressed
in the Consortium Agreement.

\paragraph{Transfer of Results.}

As Results are owned by the project partner carrying out the work leading
to such Results, each project partner shall have the right to transfer
Results to their affiliated companies/organisations without prior notification to the
other project partners, while always protecting and assuring the Access
Rights of the other project partners.  Such use of Results will encourage
competitiveness of the EU market by creating broader uses of the Results
and opening up the markets for the Consortium's Results in all markets.

\paragraph{Open Source and Standards.}

A central aim of this consortium is to provide benefit to the European community.  Some of the project partners may be either using Open Source code in their deliverables or contributing their deliverables to the Open
Source communities. Alternatively, some of the partners may be contributing to Standards, be they open standards or other. Details concerning open source code use and standard contributions will be
addressed in the Consortium Agreement.

The base technologies being developed by the academic partners within the duration of the project will be published under Open Source Licenses except where specified under Intellectual Property Management to allow the broader community to benefit from the outcome of this project.

The major validated project results  will be contributed to the key international standardisation bodies  such as ISO, ITU and others where the consortium members take part in. 


\paragraph{Data Management Plan}
The primary research data that will be produced by the project will be the performance results that are reported in
various research publications.  This data will predominantly be scientific,
without confidentiality restrictions, and it will
therefore be made available through the project website, in line with the agreements that will be
made in the Consortium Agreement.  As far as possible, 
this data will be recorded in a human-readable form, such as plain ASCII text
or XML.  Where this is not possible, converters will be provided to make the data accessible to other researchers
in a human-readable form. 
The research data on the project website will be associated with the relevant research publications. 
We have budgeted for adequate disk and processing capacity to allow for the expected access to this data.
The project website will be maintained after the end of the project, but to ensure long-term continuity
and value, data will also be transferred to the \SAshort{} institutional data repository.

\draftpage
\subsubsection{Communication Activities}
\label{sect:comm-activities}

\eucommentary{Describe the proposed communication measures for promoting the project and its findings during the period of the grant. Measures should be proportionate to the scale of the project, with clear objectives. They should be tailored to the needs of various audiences, including groups beyond the project's own community. Where relevant, include measures for public/societal engagement on issues related to the project.}

As described in the draft dissemination plan above, and in the description of~\ref{wp:dissem} (page~\pageref{wp:dissem}),
the \TheProject{} project aims to communicate itself and its findings intensively to various communities.
Research publications and presentations will aim to target various groups of academic and industrial researcher,
including parallel programmers, big data researchers, and programming language
designers, and will add scientific weight and credibility to our findings.  Press releases and news articles will be used to communicate project results and major
project life events (start, finish, key milestones) to both a technical and general audience.
We will also take advantage of opportunities as they arise for radio/TV interviews, public seminars and general articles
in both the technical and non-technical press.
The project website will be used to provide open access to project results, public deliverables,
software tools, technical reports, white papers, (video) tutorials, podcasts etc., and will serve
as a key resource for those wishing to use the project results, whether they are acting as an academic researcher, scientific, commercial or independent
software developer, public sector worker,  educator or private individual.
By making research results public in this way, we especially aim to engage with the software developer
community, who may not normally have access to academic papers and reports.
We will disseminate information about our tools and standards directly to customers, aiming to
increase engagement with an already motivated group of developers/users.
We will run open technical workshops that will showcase our work to interested parties.
These will generally be co-located with major networking events, such as the annual HiPeac
conference and the HiPeac spring/autumn gatherings.
We will also engage with relevant industrial/developer conferences, grass-roots meetings, workshops etc.,
producing poster and demonstrations as necessary to communicate with the broader developer community
and especially with project managers and decision makers.
Finally, we will actively participate in standardisation activities through e.g. the ISO C++ standard committee and ITU FG-DPM to support IoT and Smart Cities and Communities,
aiming to influence development and awareness of the \TheProject{} results.

\clearpage

% ---------------------------------------------------------------------------
%  Section 3: Implementation
% ---------------------------------------------------------------------------


\clearpage
\section{Implementation}

\subsection{Work Plan --- Work Packages, Deliverables and Milestones}
\label{sect:workplan}


\begin{figure}[tp]
\begin{center}
\vspace{-5mm}
\begin{tabular}{ll}
%\hspace{-0.75in}
%\includeimage[scale=0.5]{RePhorm2Pert.pdf}
%\vspace{-25mm}
\end{tabular}
\caption{Overview of the \TheProject{} Workpackage Structure and Dependencies (PERT chart)}
\label{fig:wps}
\end{center}
\end{figure}

\subsubsection*{Overall Structure of the Work Plan}

The work plan is broken down into XX technical workpackages as shown
in \textbf{Figure~\ref{fig:wps}}: WP2 deals with XX.

%% Deliverables list.
%% Deliverables ordered by Workpackage
%% Workpackages are numbered automatically in sequence - the WP number has no effect

\workpackage{1}{Project Management}
\deliverable{mgt:mailinglists}
\deliverable{mgt:swrepository}
\deliverable{mgt:periodic-rep-1}
\deliverable{mgt:periodic-rep-2}
\deliverable{mgt:periodic-rep-3}
\deliverable{mgt:final-mgt-rep}

\workpackage{2}{Security Contracts for Distributed Data Analytics}

\workpackage{3}{Security and Privacy of Big Data Storage and Analytics}

\workpackage{4}{Vulnerability Detection and Auto-Healing}

\workpackage{5}{AI-Driven Behavioural Authentication}

\workpackage{6}{Refactoring-Based Methodology for Secure Application Development}

\workpackage{7}{Requirements, Use Cases and Evaluation}
\deliverable{del:req1} % Initial requirements analysis
\deliverable{del:eval1} % Initial implementation and evaluation report
\deliverable{del:req2} % Updated requirements analysis
\deliverable{del:eval2} % Intermediate implementation and evaluation report
\deliverable{del:req3} % Updated requirements analysis
\deliverable{del:eval3} % Intermediate implementation and evaluation report
\deliverable{del:roadmapping} % Technical roadmap report

\workpackage{8}{Dissemination, Exploitation, Community Building and Communication}
\deliverable{del:pressrelease1} % Press release.
\deliverable{del:website1} % Project presentation (web site). 
\deliverable{del:data-mgt-plan} % (Month 6): Data Management Plan.
\deliverable{del:dissemplan1} % First plan for using and disseminating knowledge.
\deliverable{del:dissemplan2} % Second plan for using and disseminating knowledge.
\deliverable{del:pressrelease2} % Press release.
\deliverable{del:website2} % Project presentation (web site). 
\deliverable{del:dissemplan3} % Final plan for using and disseminating knowledge.


\bigskip\bigskip
\addtocounter{subsubsection}{1}
\addcontentsline{toc}{subsubsection}{\protect\numberline{\thesubsubsection}Work
Package List}
\fbox{\begin{minipage}{\textwidth}\begin{center}{\Large\bf
        Work package list} % (full duration of project)}
  \end{center}
  \end{minipage}}

\bigskip\bigskip

\begin{tabular}{|p{1.2cm}|p{9cm}|p{0.8cm}|p{1.35cm}|p{1cm}|p{0.9cm}|p{0.9cm}|}
\hline
{\bf Work \mbox{package} No} & {\bf Work package title} &
{\bf Lead \mbox{partic.} no.} &
{\bf Lead short name} &
{\bf Person months} & {\bf Start month} & {\bf End month} \\\hline 

\newcounter{wp}

\addtocounter{wp}{1}
\workpackageentry{\thewp}{USTAN}{24}{1}{36}
\addtocounter{wp}{1}
\workpackageentry{\thewp}{USTAN}{XX}{XX}{XX}
\addtocounter{wp}{1}
\workpackageentry{\thewp}{SCCH}{XX}{XX}{XX}
\addtocounter{wp}{1}
\workpackageentry{\thewp}{IBM}{XX}{XX}{XX}
\addtocounter{wp}{1}
\workpackageentry{\thewp}{UCY}{XX}{XX}{XX}
\addtocounter{wp}{1}
\workpackageentry{\thewp}{USTAN}{XX}{XX}{XX}
\addtocounter{wp}{1}
\workpackageentry{\thewp}{SOPRA}{XX}{XX}{XX}
\addtocounter{wp}{1}
\workpackageentry{\thewp}{UOD}{XX}{XX}{XX}

{\textbf{Total}} & & & &
\textbf{\large XX}&
&
\\\hline
\end{tabular}

\landscape

\subsubsection*{Work Plan Timing: GANTT Chart showing Task Dependencies and Information Flows}


%\vspace{-0.7in}
%\centerline{\hbox to \columnwidth{\hss%
  %  \includeimage[scale=0.9]{RePhorm2Gantt.pdf}
%\hss}}
\label{fig:gantt}
\vspace{-1in} % Fool LaTeX into avoiding unnecessary page break
\endlandscape

\newpage
%\bigskip\bigskip\bigskip

%% Set up the milestone numbers.
%\milestone{mil:initial}
\milestone{mil:req1}
\milestone{mil:tech1}
\milestone{mil:eval1}
\milestone{mil:req2}
\milestone{mil:reconfig1}
\milestone{mil:compliance1}
\milestone{mil:tech2}
\milestone{mil:eval2}
\milestone{mil:req3}
\milestone{mil:components2}
\milestone{mil:tech3}
%\milestone{mil:eval3}
\milestone{mil:final}


\fbox{\begin{minipage}{\textwidth}\begin{center}\Large\bf List of Milestones
  \end{center}
  \end{minipage}}
\label{sect:milestones}

\bigskip

\newcounter{ms}
\renewcommand{\thems}{MS\arabic{ms}}
\begin{minipage}{\textwidth}
\begin{center}
 \begin{tabular*}{\textwidth}{|p{1.5cm}|p{8.3cm}|p{1.2cm}|p{0.6cm}|p{4.2cm}|}  \hline
 \textbf{MS No.} & \textbf{Milestone name} & \textbf{Related WPs} & \textbf{Est. date} & \textbf{Means of
   verification} \\ % (success criteria below)} \\ % (deliverables shown here + success criteria below) \\
\hline
MS1 & Initial Requirements Analysis & WP7 & M3 & D7.1 \\
   \hline
\end{tabular*}
\end{center}
\end{minipage}

\setcounter{ms}{0}
\vspace{20pt}
\begin{center}
\begin{tabular*}{\textwidth}{|p{1.2cm}|p{13.3cm}|p{2.2cm}|}\hline
\textbf{MS No.} & \textbf{Success Criteria} & \textbf{Contributes to
  Objective(s)} \\
  \hline
MS1 & Initial Requirements Analysis Completed & \textbf{1 -- 8} \\
  \hline
\end{tabular*}
\end{center}

%\landscape
\newpage
\fbox{\begin{minipage}{\textwidth}\begin{center}\Large\bf List of Deliverables
  \end{center}
  \end{minipage}}
\label{sect:deliverables}

\begin{minipage}{\textwidth}
\begin{center}
\begin{tabular}{|p{0.8cm}|p{9.25cm}|p{0.8cm}|p{1.15cm}|p{1.6cm}|p{0.8cm}|p{0.8cm}|}  \hline
\textbf{Del. no.}              & \textbf{Deliverable name}        & \textbf{WP no.} & \textbf{Lead}
& \textbf{Type}              & \textbf{Dis. level}   & \textbf{Del. date}
\\ \hline

%% Year 1

\ref{mgt:mailinglists}           & Internal and public mailing lists
                                                                  & WP1 &\coordshort{} & OTHER & CO &  1 \\
  \hline \ref{mgt:swrepository} & Internal software repository & WP1 & \coordshort{} & OTHER & CO & 1 \\
\hline \ref{del:pressrelease1} & Press Release Announcing Start of \TheProject{} & \ref{wp:dissem} & \SAshort{} & DEC & PU & 3 \\
\hline \ref{del:website1} & Initial Project Website / Presentation & \ref{wp:dissem} & \SAshort{} & DEC & PU & 3 \\
\hline \ref{mgt:periodic-rep-1} & Project Periodic Report (first year) & WP1 & \coordshort{} & R & CO & 12 \\
\hline \ref{del:dissemplan1} & First Interim Report on Dissemination and Exploitation & \ref{wp:dissem} & \SAshort{} & R & PU & 12 \\
\hline \ref{mgt:periodic-rep-2} & Project Periodic Report (second year) & WP1 & \coordshort{} & R & CO & 24 \\
\hline \ref{del:dissemplan2} & Second Interim Report on Dissemination and Exploitation & \ref{wp:dissem} & \SAshort{} & R & PU & 24 \\
\hline \ref{mgt:periodic-rep-3} & Project Periodic Report (third year) & WP1 & \coordshort{} & R & CO & 36 \\
\hline \ref{del:eval3} & Report on Final Implementation of Benchmarks and Use Cases and Evaluation & WP7 & \SAshort{} & R & PU & 36 \\
\hline \ref{del:roadmapping} & Report on Technical Roadmap & WP7 & \SAshort{} & R & PU & 36 \\	
\hline \ref{del:pressrelease2} & Final Press Release Describing the \TheProject{} Results & \ref{wp:dissem} & \SAshort{} & DEC & PU & 36 \\
\hline \ref{del:website2} & Final Project Website / Presentation & \ref{wp:dissem} & \SAshort{} & DEC & PU & 36 \\
\hline \ref{del:dissemplan3} & Final Report on Dissemination and Exploitation & \ref{wp:dissem} & \SAshort{} & R & PU &  36 \\

\hline
\end{tabular}
\end{center}
\end{minipage}

%% WP titles and order are defined in deliverables.tex
%%% Workpackage style may be broken -- fix this!!

%% Local WP number counter - should possibly be global and hidden?

\newcounter{wpno}

\addtocounter{wpno}{1}

\begin{Workpackage}{\thewpno}
\wplabel{wp:management}
\WPTitle{\wpname{\thewpno}}
\WPStart{Month 1}
\WPParticipant{SA}{24}
\WPParticipant{UOD}{1}
\WPParticipant{IBM}{1}
\WPParticipant{SOPRA}{1}
\WPParticipant{SCCH}{1}
\WPParticipant{COGNI}{1}
\WPParticipant{UCM}{1}
\WPParticipant{FRQ}{1}
\WPParticipant{YAG}{1}


\begin{WPObjectives}
The objectives of \theWP{} are to undertake all project management activities, including:
\begin{compactitem}
\item
Monitoring the overall progress of the project.
\item
Ensuring the timely production of deliverables and other project outputs.
\item
Reporting to the European Commission on financial matters.
\item
Preparing and attending the annual project review meetings.

\item
Managing the project Advisory Board.
\end{compactitem}
\end{WPObjectives}

\begin{WPDescription}
This work package will perform all the activities related to monitoring
of progress towards the project milestones shown on Page~\pageref{sect:milestones}
and the deliverables listed on Page~\pageref{sect:deliverables},
assuring the quality of the deliverables, ensuring the
collation and distribution of the required periodic and other reports,
questionnaires and deliverables to
the European Commission, arranging project management meetings, tracking
the project budget in terms of expenditure and person-months,
obtaining financial certificates as required, convening project
management meetings, ensuring that important project documents
such as the project contract and the consortium agreement are
properly maintained and amended as necessary, ensuring that
contractual details are complied with, preparing any necessary amendment requests,
preparing for the periodic review meetings, and reviewing the research
results against the aims and objectives of the project.
It also involves managing and supporting the project
Advisory Board, including supporting attendance at project
meetings and obtaining feedback on the project direction and
results.
\end{WPDescription}

\begin{WPDeliverables}
\begin{compactitem}
\item
\ref{mgt:mailinglists}
(Month 1): 
Internal and external mailing lists.
\item
\ref{mgt:swrepository}
(Month 1): 
Internal software repository.
\item
\ref{mgt:periodic-rep-1}
(Month 12): 
Project Periodic Report (first year).
\item
\ref{mgt:periodic-rep-2}
 (Month 24): 
Project Periodic Report (second year).
\item
\ref{mgt:periodic-rep-3}
(Month 36): 
Project Periodic Report (third year).
\item
\ref{mgt:final-mgt-rep}
(Month 36): 
Project Final Report.
\end{compactitem}
\end{WPDeliverables}
\end{Workpackage}

\addtocounter{wpno}{1}
\begin{Workpackage}{\thewpno}
\wplabel{wp:vulnerability}
\WPTitle{\wpname{\thewpno}}
\WPStart{Month 1}
\WPParticipant{IBM}{28}
\WPParticipant{YAG}{28}
\WPParticipant{SCCH}{8}
\WPParticipant{USTAN}{8}
\WPParticipant{UOD}{6}
\WPParticipant{SOPRA}{5}
\WPParticipant{UCM}{3}
\WPParticipant{COGNI}{2}



\begin{WPObjectives}
The objectives of \theWP{} are to:
\begin{compactitem}
\item Develop static source code analysis methods to isolate the portions of the end user code that possibly contain security vulnerabilities.
\item Develop dynamic code analysis methods, based on symbolic execution and reinforcement learning that will profile the identified portions of the code to check for security vulnerabilities.
\item Augment the static and dynamic code analysis with runtime monitoring of the applications executed on distributed systems to identify the runtime security attacks on end user applications.
\item Develop novel source-to-source code refactoring transformations to repair the vulnerabilities identified by the static and dynamic analysis.
\item Develop mechanisms for the runtime adaptation of the application for the situation where runtime security vulnerabilities have been identified or are suspected.
\end{compactitem}
\end{WPObjectives}

\begin{WPDescription}
The aim of \theWP{} is to develop the foundations %techniques 
for identifying and repairing security vulnerabilities in C++ and Java code. In this workpackage we focus on generic techniques, whereas WP3 deals with adaptation of these techniques to the problem of AI-based distributed data analytics. To ensure that we can catch a wide range of security vulnerabilities, we will develop three layers of analysis. Initially we will perform static analysis (\ref{task:staticanalysis}) on the source code in order to identify the portions of code that are susceptible to vulnerabilities. We will then perform dynamic analysis based on symbolic execution (\ref{task:dynamicanalysis}) on the identified portions of code. These two phases will be augmented with runtime monitoring (\ref{task:runtime}) that will allow us to identify threats that can arise during the execution of the application and from dynamic unpredictable interactions, and would therefore be missed by static and dynamic analysis. The second part of the workpackage will be dedicated to developing \emph{self-healing} technology for repairing code vulnerabilities. This will be done both statically, using formalised source-to-source code transformations ({\ref{task:statichealing}}) that will be implemented in the user interface in WP6, and at runtime, using runtime adaptation of the application and environment (\ref{task:runtimehealing}).
\end{WPDescription}

\begin{Task}
\TaskTitle{Static Source Code Analysis for Identifying Security Vulnerabilities}
\TaskParticipant{YAG}{19}
\TaskParticipant{IBM}{3}
\TaskParticipant{SA}{2}
\TaskParticipant{UOD}{2}
\TaskParticipant{SOPRA}{1}
\TaskParticipant{COGNI}{1}
\TaskStart{1}
\TaskEnd{34}
\TaskResults{%
\ref{del:vul1},
\ref{del:vul2},
\ref{del:vul3}
}
\TaskHeader{}
\tasklabel{task:staticanalysis}
In this task, we will develop technology for static analysis of the application source code to identify potential security vulnerabilities. Static analysis of the source code is relatively cheap in terms of computational complexity, but has a drawback that it produces high volume of information and potentially a significant number of false positives and duplicate warnings. %The YAG-Suite includes supervised machine learning capabilities to support a first level of false positive reduction. 
We will build on the YAG-Suite existing capabilities and extend them with advanced vulnerability modelling and detection mechanisms, based on SAST and machine learning, that will detect a wider spectrum of vulnerabilities in AI based big data analytics applications, written in Java and C/C++ code, with associated false positive reduction training. We will also develop code mining capabilities with associated APIs to feed other tasks with source code metrics that will serve as an input to the dynamic vulnerability detection technology in~\ref{task:dynamicanalysis}, where the portions of code where potential vulnerabilities are identified will be further analysed. The output of this task will also feed into~\ref{task:statichealing} as part of the inputs to self healing, and~\ref{task:iui}, the user interface where the potential vulnerabilities will be displayed to the end users in a structured way. This task will proceed in three phases. In the first phase, we will develop an initial version of the static analysis infrastructure for AI-based distributed data analytics targeting the Java language (\ref{del:vul1}). In the second phase, we will build on the experience of the first phase and extend the tool to address additional issues that arise from execution of Java applications on public clouds, and we will also produce an initial version of the analysis for C++ (\ref{del:vul2}). \taskbreak In the third and final phase, we will produce the final version of the static analysis infrastructure for Java and C++, addressing the issues that arise in execution of applications on public clouds (\ref{del:vul3}). \YAGshort{} will lead this task, providing expertise in static analysis based on machine learning and building on their YAG-Suite toolset for identifying vulnerabilities in source code. \IBMshort{} will provide the link between static analysis and dynamic analysis that follows it in the \TheProject{} tool chain. \SAshort{} will contribute with formalising rules for detection of vulnerabilities that will be a part of refactoring-based self-healing. \UODshort{} will provide feedback on the issues related to AI-based big data analytics, while \COGNIshort{} will look into the link of static analysis and extension of the applications with the authentication infrastructure. \SOPRAshort{} will provide feedback from the use cases.


\end{Task}

\begin{Task}
\TaskTitle{Symbolic Execution and Supervised Learning for Identifying Security Vulnerabilities}
\TaskParticipant{IBM}{19}
\TaskParticipant{UOD}{2}
\TaskParticipant{YAG}{1}
\TaskParticipant{COGNI}{1}
\TaskParticipant{SOPRA}{1}
\TaskStart{1}
\TaskEnd{34}
\TaskResults{%
\ref{del:vul1},
\ref{del:vul2},
\ref{del:vul3}
}
\TaskHeader{}
\tasklabel{task:dynamicanalysis}
In this task, we will develop technology for dynamic detection of vulnerabilities in the end user code. Focusing on the portions of user code identified by the technologies in  \ref{task:staticanalysis}, we will apply IBM's ExpliSAT symbolic interpretation tool to verify the presence of security vulnerability patterns in the code. We will combine symbolic execution with white box fuzzing to find exploitable bugs. Combining static code analysis with dynamic detection will allow us to avoid the known problems of path explosion and memory growth when only symbolic execution is used, by allowing this technology to focus on relatively small portions of the user code and thus consistently making progress. To that end, the symbolic execution engine will query a fuzzer at run-time for information that would assist the symbolic interpreter make progress where it previously could not. In addition to symbolic execution, we will also develop technology for supervised machine learning post-processing of SAST warnings to further eliminate the possible false positives. This task will proceed in three phases. In the first phase, we will develop an initial version of the dynamic analysis technique for AI-based big data analytics based on ExpliSAT that will analyse identified portions of the code (\ref{del:vul1}). In the second phase, we will extend this with the capabilities for fuzzy querying and improve the linking with the static and runtime analysis (\ref{del:vul2}). In the third and final phase, we will produce the final version of dynamic analysis infrastructure, incorporating also ML-based post processing of warnings from the static analysis (\ref{del:vul3}). \IBMshort{} will lead this task, providing expertise in dynamic code analysis and building on their ExpliSAT symbolic interpretation tool. \UODshort{} will provide feedback on the issues for dynamic analysis arising from AI-based big data analytics. \YAGshort{} will provide the feedback from the static analysis phase and contribute to linking between the two phases. \COGNIshort{} will provide feedback arising from integrating authentication mechanisms into the applications, while \SOPRAshort{} will provide advice on the specific issues arising from use cases.
\end{Task}

\begin{Task}
\TaskTitle{Runtime Analysis for Detecting Vulnerabilities}
\TaskParticipant{SCCH}{4}
\TaskParticipant{IBM}{2}
\TaskParticipant{YAG}{2}
\TaskParticipant{UOD}{1}
\TaskParticipant{IBM}{1}
\TaskParticipant{SOPRA}{1}
\TaskStart{1}
\TaskEnd{34}
\TaskResults{%
\ref{del:vul1},
\ref{del:vul2},
\ref{del:vul3}
}
\TaskHeader{}
\tasklabel{task:runtime}
In this task, we will develop mechanisms for monitoring the runtime execution of the applications. This will allow us to identify vulnerabilities that arise from the unexpected interaction of data processing nodes during computations and which could, therefore, not be identified by the static and dynamic analysis mechanisms. We will identify which data has to be collected at runtime in order to detect anomalies, i.e. when there are nodes in the distributed parallel system that interact with other nodes in a malicious way. This requires the specification of monitoring agents. Based on this, we will then develop anomaly detection algorithms and their specification by further agents. The focus will be on identifying if the interaction sequence of any node is in line with the specification of the system. This task will proceed in three phases. In the first phase, we will produce the initial version of the runtime analysis to collect the information from the running application (\ref{del:vul1}). In the second phase, we will extend this analysis to collect additional information related to execution of AI-based data analytics on public clouds (\ref{del:vul2}). In the third and final phase, we will produce the final version of the analysis addressing the issues that arise in the execution of these analytics on public clouds (\ref{del:vul3}). \SCCHshort{} will lead this task, contributing with their expertise in system monitoring and information collection. \YAGshort{} will contribute with providing information from static analysis that is relevant to runtime analysis. \UODshort{} will provide feedback from the specific issues related to AI-based big data analytics and provide the link with the static analysis phase. \IBMshort{} will contribute to linking of the output of the dynamic analysis phase with runtime monitoring. \SOPRAshort{} will provide advice on the issues arising from the use cases considered. 
\end{Task}

\begin{Task}
\TaskTitle{Refactoring for Self Healing}
\TaskParticipant{USTAN}{6}
\TaskParticipant{YAG}{4}
\TaskParticipant{UOD}{1}
\TaskParticipant{SOPRA}{1}
%\TaskParticipant{SCCH}{1}
\TaskParticipant{UC3M}{1}
\TaskStart{1}
\TaskEnd{34}
\TaskResults{%
\ref{del:vul1},
\ref{del:vul2},
\ref{del:vul3}
}
\TaskHeader{}
\tasklabel{task:statichealing}

In \theTask{}, we will develop refactorings that will transform C++ programs into equivalent forms with introduced security patterns (from~\ref{task:patterns}). The refactorings will be defined as source-to-source semi-automatic program transformations. 
%
%
%In \theTask{}, we will develop a number of program transformations to support the end-user by providing tool-support through the  secure-aware programming methodology from \ref{task:methodology} in WPX. These program transformations will be implemented as source-to-source program \emph{refactorings} in T6.1 WP6 and will operate at the program source level. The transformations will be proved correct in Task XX WPX. 
%
The refactorings themselves will be based on a set of formal refactoring rules (which will be defined in terms of the refactoring's pre- and post-conditions, together with a set of transformation rules). The output of the refactoring will be an equivalent application with vulnerabilities repaired, based on the practice identified in the secure patterns in~\ref{task:patterns}, where the refactorings will focus on the introduction of such patterns into the source-code. The result will be a functionally equivalent program with increased security and decreased vulnerability. %In this task we will develop a number of program transformation rules based on transforming the target language, together with their implementation, feeding in to~\ref{task:iui}. 
We will consider the outputs from~\ref{task:staticanalysis} and \ref{task:dynamicanalysis} to provide the refactoring tool with the static and dynamic analysis information required to repair them.
%Prototype rules will also be developed for other languages, such as Java and Python. 
%
%
This task will proceed in three phases. In the first phase we will identify transformations for C++  that will introduce the patterns from~\ref{task:patterns} into C++ applications and use-cases (from~\ref{wp:usecases}), by defining the refactoring's  pre- and post-conditions, together with their formal transformations rules (\ref{del:vul1}). In the second phase, we will produce implementations of the refactorings defined in the first phase, for C++ (\ref{del:vul2}).
 In the third phase, we will integrate our implementations into the dashboard user-interface from~\ref{task:iui} and provide further refactorings for some of the advanced patterns identified in~\ref{task:patterns}.
We will also explore prototype transformation rules for some of the simpler patterns for Java (\ref{del:vul3}). \SAshort{} will lead this task, contributing with their expertise in software refactoring. \YAGshort{} will contribute with the expertise in static analysis, helping formulate pre- and post-conditions that the code has to satisfy for refactorings. \UODshort{} will contribute with the expertise in big data analytics, identifying specific issues that arise in these applications while \SOPRAshort{} will provide advice based on the use cases considered.

\end{Task}



%\begin{Task}
%	\TaskTitle{Correctness of Refactorings for Self-Healing}
%	\TaskParticipant{USTAN}{24}
%	\TaskParticipant{SCCH}{1}
%	\TaskParticipant{IBM}{1}
%	\TaskStart{1}
%	\TaskEnd{34}
%	\TaskResults{%
%		%%\ref{del:model1}
%	}
%	\TaskHeader{}
%	\tasklabel{task:runtimehealing}
%In \theTask, we will formalise the refactorings from~\ref{task:statichealing} that introduce the security patterns (from~\ref{task:patterns})  developing mechanised proofs of their functional correctness. This will improve confidence to the end-user that the refactorings will not change the functional behaviour of the programs being refactored. Proofs that the refactorings increase the security properties of the program will not be handled here, but instead given in T4.X.
%
%The mechanisations will focus on a tractable subset of the target language, in order to demonstrate functional  equivalence between the original program and the refactored output, based on the identified semantics representation. This will involve identifying a representation of the semantics for the subset of the target language and defining an equivalence relation between the original and refactored programs. We will provide mechanised proofs in the form of implementations in a dependently typed language, such as Idris.
%
%This task will proceed in a number of phases. In the \emph{first} phase, we will identify the formal semantics of a tractable subset of the target language. In the \emph{second} phase, we will define an equivalence relation between the original program and the refactored output w.r.t. to the semantics identified in the first phase. Finally, in the \emph{third} phase will produce mechanised proofs of functional correctness for number of the refactorings produced  in~\ref{task:statichealing}, w.r.t. to the semantics and equivlence relations from the previous phases.
%
%\end{Task}


\begin{Task}
\TaskTitle{Runtime Adaptation for Self Healing}
\TaskParticipant{SCCH}{4}
\TaskParticipant{IBM}{4}
\TaskParticipant{YAG}{2}
\TaskParticipant{UC3M}{2}
\TaskParticipant{SOPRA}{1}
\TaskStart{1}
\TaskEnd{34}
\TaskResults{%
\ref{del:vul1},
\ref{del:vul2},
\ref{del:vul3}
}
\TaskHeader{}
\tasklabel{task:runtimehealing}
In this task we will develop mechanisms for runtime adaptation of the end user applications in the case when security violations are detected or suspected or when a suspicious agents has been detected that should be removed from the system. Both cases require the partial interrupt of the distributed application and/or infrastructure, the rollback to a consistent state and the restart after some modifications. We will also formulate the repairability proof obligations on the grounds of the research through which it will be specified, specifying after how many steps a secure situation is expected to be restored. 
This task will proceed in three phases. In the first phase, we will develop an initial runtime adaptation infrastructure that will target the applications deployed on private clouds (\ref{del:vul1}). In the second phase, we will improve this infrastructure by addressing the issues that arise with deployment on public clouds, using the additional runtime information from the second phase of~\ref{task:runtime} (\ref{del:vul2}). In the third and final phase, we will develop the final version of the runtime healing infrastructure for hybrid clouds (\ref{del:vul3}). \SCCHshort{} will lead this task, contributing with their expertise on runtime adaptivity. \IBMshort{} will provide feedback on linking between the runtime adaptation and dynamic vulnerabilities checking, helping with identifying conditions under which threats can arise. \UODshort{} will provide advice on specific issues related to big data analysis while \SOPRAshort{} will make sure the issues relevant to the specific use cases are consider in the development of the infrastructure.
\end{Task}


%\begin{Task}
%%\TaskTitle{Parallel Computation and Control for High-Level Modelling Languages} % for Aerospace and Automotive Industries}
%\TaskTitle{Advanced Vulnerability Detection in Source Code}
%\TaskParticipant{UOD}{1}
%
%\TaskStart{1}
%\TaskEnd{34}
%\TaskResults{%
%%\ref{del:model1}
%}
%\TaskHeader{}
%\tasklabel{task:vulnerability}
%
%In \theTask, we will develop advanced vulnerability detection technology for C/C++ code. We plan to apply IBM's ExpliSAT symbolic interpretation tool to discover known security vulnerability patterns in C/C++code. Moreover, to avoid known limitations of the symbolic execution approach (e.g. path explosion, memory growth, etc.), we will develop a technology that leverages the combination of symbolic execution, static code analysis and white box fuzzing to find exploitable bugs. The idea is to perform symbolic execution as the main technique to discover vulnerabilities combined with tailored fuzzing techniques in specific small areas for the purpose of assisting the symbolic execution engine to consistently make progress. For that end, the symbolic execution engine will query a fuzzer at run-time for information that would assist the symbolic interpreter make progress where it previously couldn't. Static code analysis will assist to find suspected vulnerability locations prior to the invocation of the ExpliSAT tool and to control "hard to execute" use cases.
%
%This task will proceed in three phases ...
%
%\end{Task}
%
%\begin{Task}
%\TaskTitle{Vulnerability detection in Source Code based on static analysis and machine learning}
%\TaskParticipant{YAG}{1}
%
%\TaskStart{1}
%\TaskEnd{34}
%\TaskResults{%
%%\ref{del:model1}
%}
%\TaskHeader{}
%\tasklabel{task:sast}
%
%Starting from the YAG-Suite capabilities, in \theTask we will provide and improve the advanced vulnerability detection technology which is acheivable with SAST for JAVA and C/C++ source code. To avoid the recurring problem of false positives and duplicate warnings raising from scanning, the results generated by Static Analysis will be automatically qualified by a supervised machine learning based post-processing of SAST warnings. 
%The qualification process will support a risk based approach and provide decision making information for each warning such as its relevancy and its impact on CVSS metrics such as confidentiality, integrity, etc.
%Vulnerability detection will also include a remediation focus to provide options for fixing vulnerabilities and extract, when available, examples of correct source code as (and when) found in the application.
%The YAG-Suite will offer an API to partners so that they can query the SAST results they need.
%Static code analysis will assist to find suspected vulnerability locations prior to the invocation of the ExpliSAT tool and after, in order to refine ”hard to detect” vulnerabilities.
%
%
%\end{Task}
%
%\begin{Task}
%\TaskTitle{Anomaly Detection and Adaptation for Self-Healing}
%\TaskParticipant{SCCH}{1}
%
%\TaskStart{1}
%\TaskEnd{34}
%\TaskResults{%
%%\ref{del:model1}
%}
%\TaskHeader{}
%\tasklabel{task:scch}
%In a first step it will be analysed which run-time data has to be collected through monitoring machines that are needed for the detection of any anomaly, i.e. there are agents in the distributed concurrent system that interact with the other agents in a malicious way. This requires the specification of monitoring agents. The second step addresses the development of anomaly detection algorithms and their specification by further agents. The focus will be on discovery, whether the interaction sequence of any agent is in line with the specified system. The third step
%addresses adaptation algorithm in case a security violation is detected or a suspicious agents
%has been detected that should be removed from the system. Both cases require the partial interrupt of the distributed concurrent system, the rollback to a consistent state and the restart after some modifications. The last step undertaken in this task is the formulation of repairability proof obligations on the grounds of the research through which it will be specified, after how many steps a secure situation is expected to be restored.
%\end{Task}







\begin{WPDeliverables}
  \begin{compactitem}
  \item \ref{del:vul1} (Month 10) : Software on Initial Techniques for Vulnerabilities Detection and Self Healing.
  \item \ref{del:vul2} (Month 24) : Report on Refined Techniques for Vulnerabilities Detection and Self Healing.
  \item \ref{del:vul3} (Month 34): Report on Final Techniques for Vulnerabilities Detection and Self Healing.
\end{compactitem}
\end{WPDeliverables}

\end{Workpackage}

\addtocounter{wpno}{1}
\begin{Workpackage}{\thewpno}
\wplabel{wp:securityBigData}
\WPTitle{\wpname{\thewpno}}
\WPStart{Month 1}
\WPParticipant{SCCH}{1}
\WPParticipant{UOD}{1}
\WPParticipant{COGNI}{1}
\WPParticipant{USTAN}{1}
\WPParticipant{UCM}{1}
\WPParticipant{YAG}{1}
\WPParticipant{FRQ}{1}

\begin{WPObjectives}
The objectives of \theWP{} are to:
\begin{compactitem}
\item Identify security risks in systems for distributed storage that are used in large-scale distributed machine learning based on big data;
%, targetting both distributed NoSQL databases such as ScyllaDB and MongoDb and plain distributed filesystems such as Hadoop HDFS; 
\item Identify security risks in distributed machine-learning based processing of big data, targetting systems based on distributed parallel patterns such as MapReduce;
\item Develop novel mechanisms for evaluating and reducing privacy leakage in distributed machine learning processing systems;
%secure and privacy-preserving mechanisms for data release in machine learning based big data processing; \vjcomment{SCCH part}
\item Develop a novel framework for privacy-preserving transfer of knowledge extracted from big private data sets to other parties;
%techniques for privacy-secured knowledge sharing in machine learning based AI systems; \vjcomment{SCCH part}
\item Support semi-automatic repairing of the user code by integrating the self-healing mechanisms from WPXX into security and privacy-leakage diagnostics mechanisms for machine-learning based big data storage and analysis;
\end{compactitem}
\end{WPObjectives}

\begin{WPDescription}
  The goal of this workpackage is to provide mechanisms for identifying security problems in distributed big data storage and processing, focusing on machine-learning based data analytics. It will exploit the technologies for the source-code level analysis using symbolic execution (WPXX) to identify security risks that appear both in the individual data storage and processing agents in isolation, as well as from the interaction of these agents in distributed setting. In task \ref{task:storage} we will focus on the problem of identifying vulnerabilities in systems for storage of big data used to train the machine-learning models. These systems most often use distributed NoSQL databases, such as ScyllaDB or MongoDB, or plain distributed filesystems such as HDFS that use very basic security mechanisms. Complementary to that, task \ref{task:processing} focuses on machine-learning based distributed analysis of big data. We restrict our attention to the systems that are based on parallel patterns (most of which being built on top of MapReduce paradigm), as parallel patterns encapsulate all interaction between different distributed agents, therefore localising the possible source of vulnerabilities in the distributed systems. This work will feed into WPXX, where security properties of such systems will be formally established and proved. Task \ref{task:privacyLeakage} focuses on evaluating and reducing privacy-leakage in distributed machine learning, by investigating optimal noise-adding mechanisms for that. Task \ref{task:knowledgeSharing} investigates knowledge sharing in distributed machine learning, developing techniques for privacy-preserving transfer of knowledge extrated from large data sets. Finally, task \ref{task:healing} investigates implementing self-healing mechanisms developed in WPXX into the security and privacy-leakage diagnostics techniques developed in this workpackage. Collectively, the diagnostics and repair mechanisms for security and privacy of machine learning data analytics from this work package will feed feed into the refactoring-based self-healing methodology from WP6.   XXXX   Add large picture, pyramid etc. 
\end{WPDescription}

\begin{Task}
\TaskTitle{Identifying Vulnerabilities in Distributed Data Storage for Machine Learning}
\TaskParticipant{UOD}{1}

\TaskStart{1}
\TaskEnd{36}
\TaskResults{%
%\ref{del:model1}
}
\TaskHeader{}
\tasklabel{task:storage}
In \theTask, we will define the methodology and conduct an analysis of vulnerabilities in big data storage such as NoSQL databases and file systems such as HDFS.  We will select examples of  NoSQL databases constructed in C++ or  based on the JVM ( Java and Scala for instance) and identify vulnerabilities in the source code that could lead to security risks. Open source projects will be selected so that we can exploit the technologies for the source-code level analysis using symbolic execution (WPXX) to identify security risks that appear both in the individual data storage and processing agents in isolation, as well as from the interaction of these agents in distributed setting. In particular, side projects (such as language drivers) will be examined as these provide an attack vector that may be outside the open source projects control. 
\UCM will perform vulnerability analysis of C++ storage software components.
 \end{Task}

 \begin{Task}
 \TaskTitle{Identifying Vulnerabilities in Distributed Pattern-Based Machine Learning Data Analytics}
 \TaskParticipant{UOD}{1}
 
 \TaskStart{1}
 \TaskEnd{36}
 \TaskResults{%
 %\ref{del:model1}
 }
 \TaskHeader{}
 \tasklabel{task:processing}
 In \theTask, we will define the methodology and conduct an analysis of vulnerabilities in big data analysis systems.  These systems provide their own security challenges as data is distributed across a great many systems and whilst analysis is in process the data may be transferred across nodes.  We will not tackle the security of the data whilst it is on the move (on the wire and the network stack) as that are the subject of other research areas.  The task will follow a similar pattern to that of \ref{task:storage} selecting candidates systems from those written in C++ or based on the JVM (Hadoop, Spark, Flink, Storm) exploiting source-code level analysis using symbolic execution (WPXX) to expose security vulnerabilities.
\UCM will perform vulnerability analysis of C++ processing software components.
\end{Task}
 
\begin{Task}
  \TaskTitle{Evaluation and Reduction in Data Leakage for Distributed Machine Learning Data Analytics}
  \TaskParticipant{SCCH}{1}
  
  \TaskStart{1}
  \TaskEnd{36}
  \TaskResults{%
  %\ref{del:model1}
  }
  \TaskHeader{}
  \tasklabel{task:privacyLeakage}
  In this task, we will develop mechanisms to both evaluate and reduce information leakage in distributed machine learning data analytics systems. The methods for evaluation of privacy leakage will be investigated in terms of relationship between sensitive data and released public data. Our approach is to employ a stochastic model for approximating the uncertain mapping between released noise added data and private data. The stochastic model facilitates a variational approximation of privacy-leakage in-terms of mutual information between sensitive private data and released public data. For reducing information leakage, we will develop optimal noise adding mechanisms for preserving privacy in distributed machine learning. We will derive analytically the noise distribution that maximizes a given utility function or equivalently minimizes a given data distortion function. Different use cases may demand different utility functions and thus optimal noise adding mechanism is derived for each utility function separately.    XXXX link T3.3 and T 3.4
 \end{Task}

 \begin{Task}
  \TaskTitle{Knowledge Sharing in Distributed AI}
  \TaskParticipant{SCCH}{1}
  
  \TaskStart{1}
  \TaskEnd{36}
  \TaskResults{%
  %\ref{del:model1}
  }
  \TaskHeader{}
  \tasklabel{task:knowledgeSharing}
  This task will develop an analytical framework to study and optimize the privacy-preserving transfer of knowledge extracted from a large set of labelled private data owned by a party to another party owning a few labelled data samples. The goal is to answer the question: how can a model be transferred from a source to a target domain while preserving privacy of both source and target domains? An information theoretic approach is considered to quantify transferability of knwoledge from source to target domain in-terms of mutual information between source and target data. The privacy secured knowledge sharing framework facilitates development of transfer and multi-task machine learning algorithm while optimizing the privacy-transferability tradeoff.
 \end{Task}

 \begin{Task}
  \TaskTitle{Self-Healing in Distributed Data Processing Systems}
  \TaskParticipant{SCCH}{1}
  
  \TaskStart{1}
  \TaskEnd{36}
  \TaskResults{%
  %\ref{del:model1}
  }
  \TaskHeader{}
  \tasklabel{task:healing}
  In \theTask, we will integrate self-healing mechanisms developed in WPXX into the tools and techniques developed in this work package. Collectively, the diagnostics and repairs of security vulnerabilities developed in this work package will feed into the refactoring-based self-healing methodology from WP6. A sample use case from one of the technologies surveyed in \ref{task:storage}, \ref{task:processing}in  will be selected to demonstrate the application of self healing to that technology. 
 \end{Task}
 
 
 \begin{Task}
  \TaskTitle{Privacy-preserving Biometric Data}
  \TaskParticipant{SCCH}{1}
   \TaskParticipant{COGNI}{1}
  
  \TaskStart{1}
  \TaskEnd{36}
  \TaskResults{%
  %\ref{del:model1}
  }
  \TaskHeader{}
  \tasklabel{task:privacyBiometrics}
  Bearing in mind that users' biometric data are irrevocable when exposed, it is very important to protect their privacy. In this task, new methods will be designed, developed and evaluated for preserving the privacy of the biometric data used for continuous authentication in WP5. Main aim of privacy-preserving biometric authentication is to enable users to verify themselves without disclosing raw and sensitive biometric information. For doing so, current privacy weaknesses and threats in biometric authentication will be analyzed, and novel privacy-preserving methods will be designed and developed to secure the implementation of biometric-based continuous authentication, such as, features' transformation, cancelable biometrics, pseudo-identities, etc.
 \end{Task}
 
 
\begin{Task}
\TaskTitle{Privacy vulnerability detection in Source Code from static analysis}
\TaskParticipant{YAG}{1}

\TaskStart{1}
\TaskEnd{36}
\TaskResults{%
%\ref{del:model1}
}
\TaskHeader{}
\tasklabel{task:PrivacyFromSAST}
In \theTask, we will improve the static analysis tool and its post-processing algorithms to extend vulnerability detection to distributed data specific vulnerabilities, with a particular focus on privacy and anonymization leaks. We will use as an input the description of the specific security and privacy vulnerabilities of distributed Machine learning to be identified in the task XX. We will integrate them in the static analysis detection and qualification algorithms. Programming languages to be scanned will be JAVA and C/C++.

The SAST and post processing algorithms will be developed to support design flaws in order to detect anonymization vulnerabilities and risks.

The vulnerability detection will include a business oriented semantics approach in order to support multilevel privacy and sensitive data requirements.

 \end{Task}


\begin{WPDeliverables}
  \begin{compactitem}
    \item XX
%\item \ref{del:model1} (Month 10): Report on Initial Block-Diagram Modelling, Patterns and Code Synthesis
\end{compactitem}
\end{WPDeliverables}
\end{Workpackage}

\addtocounter{wpno}{1}
\begin{Workpackage}{\thewpno}
\wplabel{wp:securityContracts}
\WPTitle{\wpname{\thewpno}}
\WPStart{Month 1}
\WPParticipant{USTAN}{1}
\WPParticipant{UOD}{1}
\WPParticipant{SCCH}{1}
\WPParticipant{IBM}{1}
\WPParticipant{UCM}{1}
\WPParticipant{YAG}{1}


\begin{WPObjectives}
The objectives of \theWP{} are to:
\begin{compactitem}
%% Why just embedded? Are these domains correct? KH
\item Develop a formal high-level multi-level specification language; % able to capture different levels of abstraction; %and security contracts
\item Define abstraction/refinement as a bidirectional transformation between specifications across levels;
% How is distribution considered?
%BOTH for source code and code patterns that represent known vulnerabilities 
%and relate different levels of abstraction (from models to code)
\item Define a notion of behavioural equivalence between code fragments (within/between language(s)), able to capture refactorings, and suitable to formalise security and privacy-preserving equivalences;
\item Define a logic to capture properties of interest  %interpreted over (distributed) traces of execution; 
including security properties and privacy norms;
\item Develop an integrated verification framework, including theorem provers and SMT solvers, to prove the correctness of the above and 
%, and capable of searching for (a subset of) traces of execution that satisfy certain parameters (meet a certain threshold for instance);
 explore an interplay between runtime verification and exhaustive formal verification;
\item Develop a novel mechanism for automated vulnerability localisation, containment, recovery and repair.

\end{compactitem}

%Needs to be related to WP2 and vulnerabilities 
\end{WPObjectives}

\begin{WPDescription}
\theWP{} explores how rigorous formal methods can %continuously 
provide systematic guarantees of security and privacy in 
software systems at different levels of abstraction.
%our software systems.
%Establishing whole-system security for end-user software is crucial but particularly challenging for distributed systems.
%The scale of distributed systems requires the development of methods for security assurance that are modular, compositional and 
%incremental. As these systems evolve over time, and are often built using existing service components exporting an API that can be used by components developed later,  
%security cannot be verified once but has to be done incrementally as the systems evolve.
%We need  to specify and ensure the security of the whole system and not just individual components or abstraction layers within a system. 
We define the abstractions to present to the programmer, provided by the operating system, architecture, and so on (Task~\ref{task:contracts}), and introduce a specification language to capture the assumptions and guarantees of abstraction layers to facilitate reasoning about security across abstraction levels.
% (Task~\ref{task:formalverif}). 
The abstractions should expose security properties in a form understandable to (non-security specialist) programmers. We adopt Abstract State Machines (ASMs), given our experience in extracting ASMs from source code automatically. 
The ability to move between layers with this approach and target the extraction of vulnerabilities from code for verification at higher levels will be key.
Similarly, we will define this as a model for our functionally equivalent code, aka refactorings, and automatically search for alternative refactorings such that their security guarantees are preserved, or conversely, detected vulnerabilities are avoided (Task~\ref{task:formalverif}).
Concerning privacy, complementary to the work on privacy-preserving machine learning algorithms of WP3, we will consider privacy expectations expressed using context-relative informational norms, formalised in logic (e.,g., FOL) and automatically checked across traces of execution (Task~\ref{task:formalverif}).  
By explicitly modelling the computer system and the abilities of adversaries, formal methods can prove that the computer system is secure against all possible attacks (up to modelling assumptions). This provides high assurance of system security, even against as-yet-unknown attacks (Task~\ref{task:attackmodels}).
%as formal approaches in practice with the use of tools such as the ones from IBM, static analyser from YAG and code extractor from SCCH. 
\theWP{} explores the integration of formal methods with the practical techniques from WP2, supports the tools from WP6 and is evaluated throughout with the use cases described in WP7.
We will further investigate the gaps or intersections between some of these approaches, enable their interoperability and hence avoid missing potential vulnerabilities at different layers of abstraction (%Task~\ref{task:dyncontracts} and 
Task~\ref{task:ContractsSastAssessment}). 
For functional security requirements we will explore the use of metrics such as confidentiality, integrity and availability to assess them (Task~\ref{task:RequirementsSastAssessment}).


\end{WPDescription}


\begin{Task}
\TaskTitle{Security Contracts support and specification}
\TaskParticipant{UCM}{1}

\TaskStart{1}
\TaskEnd{27}
\TaskResults{%
%\ref{del:model1}
}
\TaskHeader{}
\tasklabel{task:contracts}

In \theTask, we define mechanisms to represent security contracts
in such a way that they can be mapped to concrete syntax in different
programming languages. We will also support the extraction of those contracts
from existing source code allowing iterative approaches where contracts
are refined by developers.
In this task we will first add to the \textsf{eKnows} module for semi-automated extraction of ASMs from source code (which currently works for Java code) the capability to extract ASMs from C++ code. Since the module is based on abstract syntax trees and designed for multi-language reverse engineering, this first step is relatively straightforward.
%can be completed in a relatively short time. 
We will then further develop the module to perform more sophisticated higher-level abstractions, taking into account the semantics of the libraries used in the targeted distributed data analytics systems as well as that of its common programming patterns. The security contracts can then be expressed in the logics for ASMs (and its planned temporal extension) over the extracted ASMs, at the different levels of abstraction required. This will give us a unifying formalism for the rigorous verification of the envisaged security contracts.  
%In addition, we will focus on the dynamic analysis of security contracts for
%vulnerabilities that cannot be formally verified without running the software component. 
%Note that C++ and Java code will just be refinements of the extracted ASMs. Plus, given the abstraction mechanism used for the ASMs extraction, we will already have the mapping to the concrete syntax of the programming languages.
\end{Task}

\begin{Task}
\TaskTitle{Formal Verification and Reasoning about Refactorings}
\TaskParticipant{USTAN}{1}

\TaskStart{1}
\TaskEnd{27}
\TaskResults{%
%\ref{del:model1}
}
\TaskHeader{}
\tasklabel{task:formalverif}

In \theTask, we will formalise extracted contracts from source code using concurrent ASMs, and define a semantics over ASM runs using the true-concurrent model of event structures. 
We will formalise abstraction/refinement as a bidirectional transformation between ASM specifications across layers, and use them to define 
security-preserving transformations as well as a notion of behavioural equivalence between code fragments (within/between languages) and refactorings.
We will define a logic to capture properties of interest  including security properties (based on distributed stochastic temporal logics or separation
logic) and privacy norms (FOL).
We will explore how combining theorem provers and SMT solvers allows us to prove the correctness of behavioural equivalences on the one side, and search for a subset of traces of execution that satisfy certain criteria including security properties and privacy norms.
\end{Task}

%\begin{Task}
%	\TaskTitle{Correctness of Refactorings for Self-Healing}
%	\TaskParticipant{USTAN}{24}
%	\TaskParticipant{SCCH}{1}
%	\TaskParticipant{IBM}{1}
%	\TaskStart{1}
%	\TaskEnd{34}
%	\TaskResults{%
%		%%\ref{del:model1}
%	}
%	\TaskHeader{}
%	\tasklabel{task:runtimehealing}
%In \theTask, we will formalise the refactorings from~\ref{task:statichealing} that introduce the security patterns (from~\ref{task:patterns})  developing mechanised proofs of their functional correctness. This will improve confidence to the end-user that the refactorings will not change the functional behaviour of the programs being refactored. Proofs that the refactorings increase the security properties of the program will not be handled here, but instead given in T4.X.
%
%The mechanisations will focus on a tractable subset of the target language, in order to demonstrate functional  equivalence between the original program and the refactored output, based on the identified semantics representation. This will involve identifying a representation of the semantics for the subset of the target language and defining an equivalence relation between the original and refactored programs. We will provide mechanised proofs in the form of implementations in a dependently typed language, such as Idris.
%
%This task will proceed in a number of phases. In the \emph{first} phase, we will identify the formal semantics of a tractable subset of the target language. In the \emph{second} phase, we will define an equivalence relation between the original program and the refactored output w.r.t. to the semantics identified in the first phase. Finally, in the \emph{third} phase will produce mechanised proofs of functional correctness for number of the refactorings produced  in~\ref{task:statichealing}, w.r.t. to the semantics and equivlence relations from the previous phases.
%
%\end{Task}


%Chris' text
%In \theTask, we will formalise the refactorings from TX.X showing proof sketches of their correctness. We will identify a well-formed semantics for a subset of the target language and show an equivalence relation for a selected number of refactorings over this language. The proof sketches will be mechanised, using a theorem prover such as COQ or Isabelle, or by using a dependent typed approach, such as Idris. The proofs of correctness will show that the refactored program is (functionally) equivalent to the original, w.r.t. to the functional semantics defined. 

\begin{Task}
\TaskTitle{Attack Models and Security Repairability of Proof Obligations}
\TaskParticipant{SCCH}{1}

\TaskStart{1}
\TaskEnd{27}
\TaskResults{%
%\ref{del:model1}
}
\TaskHeader{}
\tasklabel{task:attackmodels}

The aim of \theTask\ is to define classes of security contract requirements for distributed data analytics applications, and formalise repair proof obligations for ASMs with respect to these security contract requirements. In a first step towards this, the industrial use cases will be analysed for security threats from two different angles concerning secrets in the data and processes that are to be protected, and anticipated actions of potential attackers. Regarding secrets this will be formalised through static and dynamic constraints, for which subformulae are identified that by themselves do not contain any secrets. Regarding potential attacks, attacker models for each of the security constraints will be developed. In a second step, the one-step logic for reasoning about concurrent systems of ASM specifications will be used to formalise the security constraints. In addition, for each identified attack the anticipated approach of the attacker will be specified by a corresponding ASM.
\end{Task}


%\begin{Task}
%\TaskTitle{Dynamic Analysis of Security Contracts}%% Added to Task 1
%\TaskParticipant{UOD}{1}
%
%\TaskStart{1}
%\TaskEnd{27}
%\TaskResults{%
%%\ref{del:model1}
%}
%\TaskHeader{}
%\tasklabel{task:dyncontracts}
%
%In \theTask, we will focus on the dynamic analysis of security contracts for
%vulnerabilities that cannot be formally verified without running the software component.
%%cannot be proved by formal verification without running the software component.
%\end{Task}



\begin{Task}
\TaskTitle{Analysis-based Risk Assessment for Source Code Security Breaches}
\TaskParticipant{YAG}{1}

\TaskStart{1}
\TaskEnd{27}
\TaskResults{%
%\ref{del:model1}
}
\TaskHeader{}
\tasklabel{task:ContractsSastAssessment}

On the one side, we will use dynamic analysis of security contracts for
vulnerabilities that cannot be formally verified without running the software component.
As a complementary approach to extend formal verification of security contracts, %in \theTask, 
we will provide a novel way to assess the risk that a security contract is not met by an application. This will include:

\begin{itemize}
\item The parsing and analysis of the security contracts; %resulting from present \theWP / T4.1 and T4.2
    \item Mapping of the security contracts with the potentially impacted source code;
    \item Designing an interface between formal verification and static analysis to collect and include formally verified properties of the source code in the vulnerability assessment algorithms;
    \item Explore overall risk assessment that a security contract is not met, integrating formally proven properties to static analysis outputs and machine learning detected uncertain security properties;
    \item Provide remediation proposals to fix vulnerabilities.
\end{itemize}

\end{Task}


\begin{Task}
\TaskTitle{Assessment of security requirements based on CVSS metrics}
\TaskParticipant{YAG}{1}

\TaskStart{1}
\TaskEnd{27}
\TaskResults{%
%\ref{del:model1}
}
\TaskHeader{}
\tasklabel{task:RequirementsSastAssessment}

As another approach to extend formal verification of security contracts, in \theTask\ we can prototype a semi-automated assessment of functional security requirements, using the CVSS metrics such as confidentiality, integrity and availability.
We will develop a domain specific language to capture the functionalities description, allocate requirements and map with the source code. Each vulnerability which breaks a requirement will be identified and allocated decision making information for the user.
\end{Task}


\begin{WPDeliverables}
  \begin{compactitem}
    \item XX
%\item \ref{del:model1} (Month 10): Report on Initial Block-Diagram Modelling, Patterns and Code Synthesis
\end{compactitem}
\end{WPDeliverables}
\end{Workpackage}

\addtocounter{wpno}{1}
\begin{Workpackage}{\thewpno}
\wplabel{wp:securityContracts}
\WPTitle{\wpname{\thewpno}}
\WPStart{Month 1}
\WPParticipant{UCY}{1}


\begin{WPObjectives}
The objectives of \theWP{} are to:
\begin{compactitem}
%% Why just embedded? Are these domains correct? KH
\item Solve the world
\end{compactitem}
\end{WPObjectives}

\begin{WPDescription}
\theWP{} addresses the problem of solving the world.
\end{WPDescription}

\begin{Task}
%\TaskTitle{Parallel Computation and Control for High-Level Modelling Languages} % for Aerospace and Automotive Industries}
\TaskTitle{Solving the World}
\TaskParticipant{UOD}{1}

\TaskStart{1}
\TaskEnd{27}
\TaskResults{%
%\ref{del:model1}
}
\TaskHeader{}
\tasklabel{task:solve}

In \theTask, we will solve the world. 
This task will proceed in three phases. 
\end{Task}



\begin{WPDeliverables}
  \begin{compactitem}
    \item XX
%\item \ref{del:model1} (Month 10): Report on Initial Block-Diagram Modelling, Patterns and Code Synthesis
\end{compactitem}
\end{WPDeliverables}
\end{Workpackage}

\addtocounter{wpno}{1}
\begin{Workpackage}{\thewpno}
\wplabel{wp:methodology}
\WPTitle{\wpname{\thewpno}}
\WPStart{Month 1}
\WPParticipant{USTAN}{1}
\WPParticipant{YAG}{1}


\begin{WPObjectives}
The objectives of \theWP{} are to:
\begin{compactitem}
\item Develop new refactorings that repair security vulnerabilities.
\item Develop new refactorings that rewrite source-code so that it conforms to a security-aware coding standard.
\item Develop new refactorings that introduce security-aware patterns into the source-code.
\item Develop new security-aware patterns.
\item Develop a new security-aware software engineering methodology for the development of secure big-data applications.
\item Develop a new C++ coding standard for the development of security-aware big-data applications.


\end{compactitem}
\end{WPObjectives}

\begin{WPDescription}
This workpackage will investigate new software engineering techniques and tools for the development of security-aware big-data applications. We will develop new end-user refactoring tool support in \ref{task:refactoring}; new security-aware software development patterns in \ref{task:patterns}, that will be provided to the user in the form of a new software library, with refactoring tool-support for their introduction in \ref{task:refactoring}; a new security-aware software engineering methodology in \ref{task:methodology}, encapsulating new software engineering methods and techniques from all the tools developed as part of the \TheProject{} project. In \ref{task:standards}, we will develop a new C++ coding standard for the development of secure big-data applications in C++. Finally, in \ref{task:interoper}, we will enable interoperability between the tools developed in the other workpackages to form a coherent software tool-chain and methodology for the end-user.


\end{WPDescription}

\begin{Task}
%\TaskTitle{Parallel Computation and Control for High-Level Modelling Languages} % for Aerospace and Automotive Industries}
\TaskTitle{Refactorings for Secure Big-Data Applications}
\TaskParticipant{UOD}{1}

\TaskStart{1}
\TaskEnd{27}
\TaskResults{%
%\ref{del:model1}
}
\TaskHeader{}
\tasklabel{task:refactoring}

In  \theTask{}, we will develop semi-automatic refactorings to support the end-user by providing tool-support through the  secure-aware programming methodology from \ref{task:methodology}. These refactorings will operate at the program source level, where a refactoring will be implemented as a source-to-source program transformation. 

The refactorings themselves will be developed based on a set of formal refactoring rules (which will be defined in terms of the refactoring's pre- and post-conditions, together with a set of transformation rules). The output of the refactoring will be an equivalent application, with increased security, decreased vulnerability and higher conformance to the security-aware coding standards (from \ref{task:standards}).

The refactoring tool will be integrated into the ParaFormance refactoring tool, supporting both Eclipse and Visual-Studio for C++ applications. A prototype implementation for Java-based applications will also be produced. The results of this task will be used to refactor the use-cases in \ref{wp:usecases}.

This task will proceed in three phases. In the \emph{first} phase we will identify transformations for C++ applications that will introduce the patterns from \ref{task:patterns} into C++ applications, by defining their pre- and post-conditions, together with their formal transformations rules. In the \emph{second} phase, we will produce implementations of the refactorings defined in the first phase, implemented in the ParaFormance tool-chain, for C++. In this phase, we will consider the outputs from the Advanced Vulnerability Detection from \ref{task:vulnerability} and the Self-Healing from T XX (as defined in \ref{wp:vulnerability}) to provide the refactoring tool with the static analysis information required to discovery vulnerabilities and information on how to repair them.
 In the \emph{third} phase, we will produce refactorings that further refactor the code so that it conforms to the standards outlined in \ref{task:standards}, where possible. We will also produce prototype implementations of a tractable set of the refactorings from all the phases for Java-like languages.


%
%In particular, in this task we will produce new refactorings that will:
%
%\begin{itemize}
%	\item Repair vulnerable code, by taking the output of the Self Healing tooling from XX, 
%	\item Introduce security-aware patterns, based on the implementation of the patterns in Task X.
%	\item Rewrite source-code so that it conforms to the standard as defined in Task X.
%\end{itemize}
\end{Task}

\begin{Task}
	%\TaskTitle{Parallel Computation and Control for High-Level Modelling Languages} % for Aerospace and Automotive Industries}
	\TaskTitle{Security-aware Software Development Patterns}
	\TaskParticipant{UOD}{1}
	
	\TaskStart{1}
	\TaskEnd{27}
	\TaskResults{%
		%\ref{del:model1}
	}
	\TaskHeader{}
	\tasklabel{task:patterns}
	
In \theTask{} we will formalise and implement patterns of security that arise in typical big-data applications. As part of this process, we will investigate the relationship between common security threats and their source-level solutions, producing a high-level domain-specific language (DSL) for describing a set of design patterns for the end-user. This DSL will be made available to the end-user developer as a library of security-aware patterns and will be used by the refactoring tooling from \ref{task:refactoring}. The DSL will be implemented in C++, using advanced features from the latest C++ standards, abstracting away the low-level details of programming security models.

The task will proceed in three phases. In the \emph{first} phase, we will identify a set of \emph{fundamental} patterns for repairing security vulnerabilities. These patterns will reported as high-level schemas, encapsulating the side-conditions to the pattern, the semantics of the pattern, and design models of the the pattern's implementation.
In the \emph{second} phase, we will provide implementations of the fundamental security patterns from the first phase, implemented as a C++ library. In the \emph{third} and final phase, we will extend the fundamental set of patterns, to include some advanced security patterns, and their prototype implementations in C++. If time permits, we will also provide prototype implementations of the fundamental patterns for Java-like languages. 
\end{Task}

\begin{Task}
	%\TaskTitle{Parallel Computation and Control for High-Level Modelling Languages} % for Aerospace and Automotive Industries}
	\TaskTitle{New Coding Standards for Security-Aware Applications}
	\TaskParticipant{UOD}{1}
	
	\TaskStart{1}
	\TaskEnd{27}
	\TaskResults{%
		%\ref{del:model1}
	}
	\TaskHeader{}
	\tasklabel{task:standards}
	
	In \theTask, we will develop new C++ coding and security standard to facilitate programming of secure big-data applications. The task will proceed in two phases. First, we will identify those properties that arise in security-aware applications, such as secrecy of variables, branching vulnerabilities, etc.; secondly, we will implement a new security-aware programming standard, encapsulating the properties from the first phase, to create a new C++ coding standard. 
	\begin{itemize}
		\item Identify the language-properties relating to security 
		\item vulnerabilities, branching, etc.? non-functional properties?
		\item create a new C++ coding style/standard.
	\end{itemize}
\end{Task}

\begin{Task}
	%\TaskTitle{Parallel Computation and Control for High-Level Modelling Languages} % for Aerospace and Automotive Industries}
	\TaskTitle{Secure Data Fabrication}
	\TaskParticipant{UOD}{1}
	
	\TaskStart{1}
	\TaskEnd{27}
	\TaskResults{%
		%\ref{del:model1}
	}
	\TaskHeader{}
	\tasklabel{task:fabrication}
	
	In \theTask, we will produce new tool-support for the fabrication of security sensitive data in the context of big-data. The input of this tool will be based on X from WPX. 
\end{Task}

\begin{Task}
	%\TaskTitle{Parallel Computation and Control for High-Level Modelling Languages} % for Aerospace and Automotive Industries}
	\TaskTitle{A Methodology for the Development of Secure Applications}
	\TaskParticipant{UOD}{1}
	
	\TaskStart{1}
	\TaskEnd{27}
	\TaskResults{%
		%\ref{del:model1}
	}
	\TaskHeader{}
	\tasklabel{task:methodology}
	
	In \theTask, we will produce a new software engineering methodology for the development of secure-aware big-data applications. In this task, we will look to assess the suitability of the tools and techniques developed within the project to support different software development models. 
	In the first phase of this task, we will assess existing  software engineering methods and their applicability to develop secure-aware applications for big data.
	In the second phase of this task, we will assess the applicability of the individual tools and techniques developed on the \TheProject{} project, and their applicability to a secure-aware software methodology. 
	Finally, in the final phase of this task, we will ensure interoperability of the tools produced in WP2, WP3, WP4, WP5 and WP6, producing an interoperable secure-aware tool-chain
	\begin{itemize}
		\item requirements capture (software engineering based?)
		\item software development models?
		\item assess applicability of tools ?
	\end{itemize}
\end{Task}


\begin{Task}
	\TaskTitle{How \YAG can contribute. Here are some ideas of what we could do:}
	\TaskParticipant{YAG}{1}
	
	\TaskStart{1}
	\TaskEnd{27}
	\TaskResults{%
		%\ref{del:model1}
	}
	\TaskHeader{}
	\tasklabel{task:SastToFeedRefactoring}
	
	In \theTask, we can investigate how our way of doing static analysis can feed the semi automated refactoring process and tool. For instance we can search and detect specific properties of the source code, be they certain or uncertain (code smells), correlate them to find if a certain pattern is met and feed the refactoring.
	We also can provide decision making information out of static analysis on "which code to refactor", on detected pre conditions as well as providing potential different options, based on semantics, to feed the refactoring.
	\color{blue} \textbf{Limitation:} It will not be possible to modify the static analysis intermediate representation to adapt to refactoring specific needs.
\end{Task}


%\begin{Task}
%	%\TaskTitle{Parallel Computation and Control for High-Level Modelling Languages} % for Aerospace and Automotive Industries}
%	\TaskTitle{Interoperability of the \TheProject{} Tool-Chain}
%	\TaskParticipant{UOD}{1}
%	
%	\TaskStart{1}
%	\TaskEnd{27}
%	\TaskResults{%
%		%\ref{del:model1}
%	}
%	\TaskHeader{}
%	\tasklabel{task:interoper}
%	
%	In \theTask, we will ensure the interoperability of the tools produced in WP2, WP3, WP4, WP5 and WP6. 
%	
%	\begin{itemize}
%		\item verification tools
%		\item self-healing
%		\item end-users tools
%	\end{itemize}
%\end{Task}




\begin{WPDeliverables}
  \begin{compactitem}
    \item XX
%\item \ref{del:model1} (Month 10): Report on Initial Block-Diagram Modelling, Patterns and Code Synthesis
\end{compactitem}
\end{WPDeliverables}
\end{Workpackage}

\addtocounter{wpno}{1}
\begin{Workpackage}{\thewpno}
\wplabel{wp:usecases}
\WPTitle{\wpname{\thewpno}}
\WPStart{Month 1}
\WPParticipant{SOPRA}{1}
\WPParticipant{COGNI}{1}
\WPParticipant{FRQ}{1}
\WPParticipant{YAG}{1}



\begin{WPObjectives}
The objectives of \theWP{} are to:
\begin{compactitem}
\item	Determine the overall requirements for tools and use cases that will be developed in \TheProject{} and success metrics for demonstrating that the project achieves the determined objectives;
\item	Identify and implement real-world use cases from air traffic management, healthcare and banking domains that will demonstrate \TheProject{} tools and techniques;
\item	Evaluate the \TheProject{} technologies and techniques on the developed use cases and against the expected results;
\item	Provide a technical roadmap for long-term development and use of the \TheProject tools and technologies.
\end{compactitem}
\end{WPObjectives}

\begin{WPDescription}
The purpose of this work package is to evaluate the \TheProject technology on real-world use cases from the air traffic, healthcare and banking domains. In all these domains, ensuring security of data analytics, privacy of data access, as well as ensuring the quality of data is of utmost importance. In this workpackage, we will i) define the requirements and success metrics that the \TheProject{} technologies and use cases will need to satisfy in order to achieve the desired results, as set out in project objectives, and to achieve the expected impacts (T7.1); ii) adapt the existing and develop new use cases that will demonstrate that the \TheProject{} tools and techniques can support development of secure and privacy preserving distributed data analytics application for end users (T7.2); iii)  evaluate the effectiveness and benefits that our techniques will bring in terms of the success metrics tidentified in T7.1 (T7.3); and, iv) provide a roadmap for addressing the issues that will arise from the usage of the \TheProject{} technology for future-generation
secure distributed systems (T7.4).
\end{WPDescription}

\begin{Task}
%\TaskTitle{Parallel Computation and Control for High-Level Modelling Languages} % for Aerospace and Automotive Industries}
\TaskTitle{Requirements and Success Metrics}
\TaskParticipant{UOD}{1}

\TaskStart{1}
\TaskEnd{27}
\TaskResults{%
%\ref{del:model1}
}
\TaskHeader{}
\tasklabel{task:requirements}
\theTask{} will determine the requirements of the \TheProject{} project as a whole.  It will identify the main technical and functional challenges that must be addressed by the \TheProject{} approach. The identified  requirements will be fed into each of the technical work packages, forming a foundation for the design and implementation of the associated methods and tools. Complementing the success criteria for the overall expected impacts (Section~\ref{sect:impacts}), application-specific success criteria for each already existing use case, plus the associated measurement criteria, will be defined as part of the task deliverables. This will be done by determining Key Performance Indicators, KPIs, creating clear definitions, undertaking baseline measurement of existing use cases in the first phase of the task, resulting in \ref{del:req1} (M3). This will later be refined in the second phase to take into account feedback from the initial implementations of the use cases, resulting in \ref{del:req2} (M14). This will allow a precise measure of success to be defined for each use case, without over-generalisation.
\end{Task}

\begin{Task}
  \TaskTitle{Use Cases}
\TaskStart{1}
\TaskEnd{27}
\TaskResults{%
%\ref{del:model1}
}
\TaskHeader{}
\tasklabel{task:usecases}
theTask{} will develop the use case applications that will be used to evaluate the \TheProject{} technology in a realistic setting. They will be used to test our technology as a whole in realistic conditions, using both synthetic data produced using data fabrication, as well as realistic data coming from the use cases.

\textbf{Air Traffic Management.} 

To address the use case described in section \ref{sec:atm} the Frequentis MosaiX SWIM aviation integration platform is used which provides tools and applications that ANSPs need to overcome these challenges and to facilitate the exchange of information between all industry stakeholders for better-informed decision making and shared situational awareness. It is specifically designed to assist the Aviation industry with its migration to SWIM and to facilitate the exchange of information between all industry stakeholders for better informed decision making and the creation of new applications and services. This reduces vendor lock-in by allowing customers to substitute components and incorporate new technologies in the future. The outcome of this project will ensure that one of the key benefits, is ensured in a comprehensive security solution.
In addition, the customisable metrics and dashboards will be extended by the machine learning outcomes produced within this project including early-deviation-recognition and security pattern discovery. MosaiX SWIM represents a fundamental shift from today’s monolithic solutions where adopting new technologies requires re-engineering the whole system.
The use case of drone deconfliction will be use for the use case. To perform strategic deconfliction the flight path of one of the conflicting drones needs to be altered at the leg between two way points at which the conflict takes place. One drone should climb, the other drone should descend. A conflict occurs if two or more drones flying along their planned flight paths come into close proximity with each other. To be able to perform 4D conflict detection the following information needs to be known: Drone flight path way points in 3D space and the planned time of arrival for each way point. For tactical conflict resolution the shortest vector of way from an approaching other drone shall be calculated and executed immediately. However, such capability is expected to be built into the drone itself, not requiring any external input or network.

\emph{We want to prove that we can secure an distribute coding base ranging over several different domains without compromising the overall ecosystem.}


\begin{itemize}
    \item Languages
        \begin{itemize}
            \item Java
            \item JavaScript
			\item Python
        \end{itemize}
    \item Platform and Technology
        \begin{itemize}
            \item Elasticsearch
            \item Azure Data Lake Analytics
            \item Azure AI
        \end{itemize}
    \item Storage
        \begin{itemize}
            \item Azure Data Lake
			\item Azure SQL Database
            \item My-SQL
        \end{itemize}
\end{itemize}

\textbf{Digital Banking.}

Digital banking is the digitization or transferring online of all the traditional banking activities and processing services that were historically only available to customers when physically inside of a bank branch. Customers are now performing touch-less transactions with money deposits, withdrawals, and fund transfers executed on mobile devices. The transaction touch-point is now migrated to customers mobile phones and tablets. Merchants that traditionally would have had a fix address (place of business) are now mobile as their own mobile devices are now converted into card terminals. The traditional trusted end-point security model is now disrupted into a \emph{zero trust architecture}. The addition of Internet-of-Things transactions i.e. train access gates, unmanned shops and push button ordering devices the stable environment of banking is now open to the revolution in the information age.

\emph{We want to secure each component to ensure the specific step in the digital banking process is secure and trusted. The core issue is that the different processing steps are no-longer solely under the control of the traditional banking ecosystem.}

\begin{itemize}
    \item Languages
        \begin{itemize}
            \item C++
            \item Python using C++ libraries
        \end{itemize}
    \item Platform and Technology
        \begin{itemize}
            \item Azure Data Factory
            \item Azure Data Lake Analytics
            \item Azure Synapse Analytics
            \item Azure Databricks
            \item Azure AI
            \item AWS Fargate
            \item AWS Athena
        \end{itemize}
    \item Storage
        \begin{itemize}
            \item Azure Data Lake
            \item Azure Databricks
            \item AWS Lake Formation
            \item AWS S3
            \item AWS Redshift
        \end{itemize}
\end{itemize}

\textbf{Healthcare.} https://www.overleaf.com/project/5e5e45121e493b000149fe20

Global healthcare is becoming a major impact on the global environment as related to economical impact, ease of travel between countries. The world is globalising at an ever-increasing rate. The rise in health care costs globally is causing a disruption of the traditional healthcare model as patient demographics is now evolving to a model of medical cover in home country with cover globally. This evolving consumer expectations is generating new market complexities. Complex health and technology ecosystems are forcing new data sharing agreements and requirements onto the global health care providers. People now invest in value-add care, advanced digital technologies and innovative care delivery requirements. At the core of this is a massive set of ever-increasing data interchange and processing requirements. Global healthcare needs a \emph{zero trust architecture} to support these uncertainties in security and build a smart health ecosystem that can meet the current and future healthcare model.

\begin{itemize}
    \item Languages
        \begin{itemize}
            \item C++
            \item Python using C++ libraries
        \end{itemize}
    \item Platform and Technology
        \begin{itemize}
            \item Azure Data Factory
            \item Azure Data Lake Analytics
            \item Azure Synapse Analytics
            \item Azure Databricks
            \item Azure AI
            \item Azure Bot Service
            \item Azure AI
            \item AWS Fargate
            \item AWS Athena
            \item Google Cloud Dataproc
            \item Google Cloud BigQuery
        \end{itemize}
    \item Storage
        \begin{itemize}
            \item Azure Data Lake
            \item Azure Databricks
            \item AWS Lake Formation
            \item AWS S3
            \item AWS Redshift
            \item Google Cloud Data Lake
        \end{itemize}
\end{itemize}

This is a extension of the work we are completing for Europe via the SERUMS project (\url{https://www.serums-h2020.org/})

%% investigate (https://cordis.europa.eu/project/id/883335)

\emph{We want to show that by securing the processing into the core healthcare ecosystem at the edge it indirectly secures the internal zero trust architecture security model.}

\end{Task}

\begin{Task}
  \TaskTitle{Evaluation}
\TaskStart{1}
\TaskEnd{27}
\TaskResults{%
%\ref{del:model1}
}
\TaskHeader{}
\tasklabel{task:evaluation}
\theTask{} will evaluate the \TheProject{} tools and technologies against the overall project requirements and success criteria that were identified in T7.1, using the representative metrics and use cases that were developed in T7.2. We aim to especially evaluate those aspects that contribute to the creation of impact by the industrial partners from the selected domains. We will evaluate both how much the security and privacy of the data and data analytics models is improved using the \TheProject{} techniques, using simulated cyber-attacks for testing, and how easy it is for end users to use the \TheProject{} methodology for developing secure distributed data analytics applications. The lessons learned from this task will provide feedback into different technical work packages, steering the development of \TheProject{} technologies in the subsequent phases of the project, as well as into the roadmapping task (T7.5). The task will proceed in three phases, evaluating initial (phase 1), refined (phase 2) and final use cases (phase 3). Results will be reported in \ref{del:eval1} (M12), \ref{del:eval2} (M25) and \ref{del:eval3} (M36), respectively. 
\end{Task}

\begin{Task}
  \TaskTitle{Roadmapping}
  \TaskParticipant{UOD}{1}
  \TaskResults{
  \ref{del:roadmap}.
  }
  \TaskHeader{}
  \tasklabel{task:roadmapping}
  \theTask{} will use the outcomes of T7.3 to undertake roadmapping activities that are aimed at situating future \TheProject{} technologies in the context of emerging technical developments in development of secure distributed data analytics applications and in broadening the long-term impact of the \TheProject{} project. This roadmap will consider new and emerging developments in security, deep-learning, data lakes, blockchains, cloud computing etc. as they relate to the secure analytics of big data. It will highlight possible future applications of the \TheProject{} technology, and indicate the further work that needs to be done to bring the \TheProject{} technologies to market. The result of this task will be a report describing the technical roadmap, highlighting the issues for future research and use of \TheProject{} technology, also being used as input for T8.2 (Exploitation and Use). 
  \end{Task}
  

\begin{Task}
  \TaskTitle{What about Assessment of business oriented sensitive data protection ?}
  \TaskParticipant{YAG}{1}
  \TaskResults{
  \ref{del:roadmap}.
  }
  \TaskHeader{}
  \tasklabel{task:BusinessDataFlaws}
  \color{blue} \textbf{(TBD)}
  Dealing with data, the YAG-Suite could support a certain level of business oriented semantics for multilevel security requirements allocation, detecting potential sensitive business data leaks and start building the bridge between business views (for instance Air Traffic Management data vs Unmanned Traffic Management data if it makes sense that they have different security requirements) and programmatic views.
  In \TheProject{} it would be interesting to include such customization in the use cases.
  
  \end{Task}

\begin{WPDeliverables}
  \begin{compactitem}
    \item XX
%\item \ref{del:model1} (Month 10): Report on Initial Block-Diagram Modelling, Patterns and Code Synthesis
\end{compactitem}
\end{WPDeliverables}
\end{Workpackage}

\addtocounter{wpno}{1}
\begin{Workpackage}{\thewpno}
\wplabel{wp:dissem}
\WPTitle{\wpname{\thewpno}}
\WPStart{Month 1}
\WPParticipant{UOD}{1}


\begin{WPObjectives}
The objectives of \theWP{} are to:
\begin{compactitem}
  \item Disseminate research results to the scientific community;
  \item Ensure awareness of the results in the user community;
  \item Raise general public awareness of the \TheProject{} project, using an Open Science model;
  \item Define individual exploitation plans;
  and
  \item Manage existing and new intellectual property.
\end{compactitem}
\end{WPObjectives}

\begin{WPDescription}
As described in detail in Section~\ref{sect:dissemination} (page~\pageref{sect:dissemination}), this work package entails the dissemination of research results, the construction of an exploitation plan for the knowledge acquired during the course of the \TheProject{} project, 
the establishment of/extension of the user community for the \TheProject{} tools and techniques,
plus communication activities aimed at improving public awareness of the \TheProject{} project.
It also includes IPR management and data management.
Scientific and technical work aimed directly at ensuring the publication of the project results will be carried out within the relevant technical workpackages.
\end{WPDescription}

\begin{Task}
\TaskTitle{Dissemination and Communication Activities}


\TaskParticipant{USTAN}{1}

\TaskResults{%
\ref{del:pressrelease1},
\ref{del:website1},
\ref{del:dissemplan1},
\ref{del:dissemplan2},
\ref{del:pressrelease2}.
\ref{del:website2}.
}

\TaskStart{1}
\TaskEnd{36}
\TaskHeader{}

\vjcomment{All partners should have 10\% of their effort here.}

This task comprises all forms of direct dissemination and public communication activities.
%
\textbf{Dissemination activities} will primarily involve the production of high-quality scientific and technical research papers and associated presentations as described in Section~\ref{sect:dissemination}.
It will also involve the production of a project website, including visitor analysis and monitoring tools, promotion through social media (e.g., twitter, facebook, linkedin), technical workshop organisation, creation of advertisement materials such as flyers, posters, and electronic feeds as well as their distribution, and the engagement with key bodies such as \hipeac, ET4HPC, PRACE etc.
\TheProject{} will organise at least one open technical workshop each year
(preferably co-located with a major conference or other meeting).
The consortium's academic partners will include project methodologies and achievements in their undergraduate-, graduate- and PhD-level teaching activities within their curricula, will provide web-based short courses and recorded lectures on specific \TheProject topics, and will organise student workshops and summer schools.
Furthermore, \TheProject will disseminate its results towards standardisation bodies and working groups.
%
\textbf{Communication activities} will include the production of press releases, outreach activities (seminars, keynote talks, media interviews, CORDIS press releases), general information on the project website, and the use of social media to ensure wider engagement with the general public.
News articles will be produced by experienced professional staff at relevant partners and communicated to local, national and international media, as appropriate.
At least two press releases will be generated in the course of the project.
\end{Task}

\begin{Task}
\TaskTitle{Exploitation and Use}
\TaskParticipant{UOD}{1}

\TaskResults{%
\ref{del:data-mgt-plan},
\ref{del:dissemplan1},
\ref{del:dissemplan2}.
}

\TaskStart{1}
\TaskEnd{36}
\TaskHeader{}


This task involves producing, refining and updating the exploitation plan for the project as a whole, starting with the draft exploitation plans that have been outlined earlier on page~\pageref{sect:exploitation-plan}.
Exploitable results that will be produced in the course of \TheProject may lead to commercial innovation activities, to advances in scientific knowledge, and/or to advances in education, as appropriate.
It is the principle of all exploitation activities to use research results to create value within all participating organisations and thus to improve their competitive advantages.
Hence, this task aims at preparing the transfer of the technology developed in \TheProject{} to the project partners and to other academic and industrial partners that could gain technical, commercial and research benefit from the project results.

In order for the exploitation to be effective, an integrated approach will be necessary, combining experience and expertise from the development department and solution management, and the involvement of a user base represented by the consortium partners.
An integral part of this exploitation task will be the \TheProject{} use cases (see~\ref{wp:eval}) which will serve as validation points throughout the project lifetime.

This task also includes the continuous analysis of transfer opportunities and the evaluation of the advancement of the research results against the user requirements/needs throughout the project.
All necessary adjustments to the project plan will be communicated to project partners  in order to ensure the best possible outcome.
\end{Task}

\begin{Task}
\TaskTitle{IPR Management}

\TaskParticipant{SA}{1}

\TaskResults{%
\ref{del:dissemplan1};
\ref{del:dissemplan2}
}
\TaskStart{1}
\TaskEnd{36}
\TaskHeader{}

This task involves producing and maintaining a register of exploitable results that will be produced in the course of \TheProject{}, including recording each exploitable foreground IPR result, which partners are involved in generating that IPR, and how the result could be exploited in each case.
The register will be confidential, but available to all partners.
\end{Task}

\begin{WPDeliverables}
\begin{compactitem}
\item \ref{del:pressrelease1} (Month 3): Press Release Announcing the start of the \TheProject{} project.
\item \ref{del:website1} (Month 3): Initial Project Website / Presentation.
\item \ref{del:data-mgt-plan} (Month 6): Data Management Plan.
\item \ref{del:dissemplan1} (Month 12): First Interim Report on Dissemination \& Exploitation, including Exploitation Plan, Communication Activities, User Community Building, IPR Management and First Project Workshop.
\item \ref{del:dissemplan2} (Month 24): Second Interim Report on Dissemination \& Exploitation, including Exploitation Plan, Communication Activities, User Community Building, IPR Management, and Second Project Workshop.
\item \ref{del:pressrelease2} (Month 36): Final Press Release Describing the \TheProject{} Results.
\item \ref{del:website2} (Month 36): Final Project Website / Presentation.
\item \ref{del:dissemplan3} (Month 36): Final Report on Dissemination \& Exploitation, including Exploitation Plan, Communication Activities, User Community Building, IPR Management, and Final Project Workshop.\end{compactitem}
\end{WPDeliverables}
\end{Workpackage}


\endinput




% \TODO{Milestones need to be discussed and then described here.}

\bigskip\bigskip\bigskip
%\draftpage
\pagebreak
\fbox{\begin{minipage}{\textwidth}

\begin{center}\Large\bf
Critical Risks for Implementation
\label{sect:risks}
\end{center}
\end{minipage}}

\bigskip
Steps have already been taken to reduce the level of risk within the overall implementation plan.  The table below
identifies the main residual risks that are foreseen.  This register will be maintained
and updated as necessary during the project in order to minimise risk and so to maximise its successful completion.

\bigskip

%\begin{tabular}{| p{3.2cm} | p{1.8cm} | p{1.5cm} | p{10.3cm}  |}  \hline
%\begin{longtable}{| p{3.2cm} | p{1.8cm} | p{1.5cm} | p{10.3cm}  |}  \hline
\begin{longtable}{| p{3.5cm} | p{1.5cm} | p{11.8cm}  |}  \hline
\textbf{Description of risk} & \textbf{WPs\newline involved} & \textbf{Proposed Risk-mitigation measures} \\ \hline
\multicolumn{3}{l}{\ }
\\\hline
XX (Milestone XX). 
\par\vspace{1ex}
\textbf{Severity: Medium}
\par
\textbf{Likelihood: Medium}&
WP2--\ref{wp:eval}  &  
This is an early milestone that defines the overall project direction.  
\\\hline
\end{longtable}
%\newpage

%\vspace{-6pt}
\subsection{Management Structure and Procedures (Figure~\ref{fig:management})}
\label{sect:mgt}

\eucommentary{Describe the organisational structure and the decision-making ( including a list of milestones (table 3.2a)).\\
Explain why the organisational structure and decision-making mechanisms are appropriate to the complexity and scale of the project.\\
Describe, where relevant, how effective innovation management will be addressed in the management structure and work plan.\\
Describe any critical risks, relating to project implementation, that the stated project's objectives may not be achieved. Detail any risk mitigation measures. Please provide a table with critical risks identified and mitigating actions (table 3.2b).}

Responsibility for the overall management and technical
direction of the project will rest with the \emph{Project Coordinator}
(Dr Juliana Bowles, \SA{}) who will be the primary point of
contact with the European Commission. Responsibility for
individual work packages will rest with the \emph{Work Package Team
Leader (WTL)} identified below, who will report to the Project Coordinator.
Where a work package is split across more than one
institution, the day-to-day management of each task will be handled
locally, with the task manager reporting to the WTL.   
% Disputes
% will be resolved at the lowest possible level by an independent
% adjudicator (for disputes between WTLs this will be the Project
% Coordinator, unless he is involved in the dispute).
%
In order to ensure good integration of the project and sound
overall management, the Project Coordinator will convene
annual technical workshops containing representatives from
the entire project team.   These workshops will be open to
invited external researchers/industrialists, including members
of the \emph{Project Advisory Board}, and will usually be
accompanied by a physical meeting of the \emph{Project Steering Committee}. In
addition, the Project Coordinator will convene management
meetings involving the relevant partners and members of the
Project Advisory Board, as necessary and appropriate. These meetings will be
conducted either using video-conferencing, through a
teleconference, or in person, as appropriate and with due consideration to
cost, urgency and effectiveness.  Technical teams
working on a work package that is spread across sites will
coordinate through email, video-conferencing, telephone and
scheduled meetings.  Finally, the research teams will maintain
regular contact with the Project Coordinator and each other
through regular email reports and telephone conversations.
Progress will be carefully monitored with progress reports and
monitoring documents open to inspection by the EU project
monitoring officer. In the event of a serious and urgent matter
involving all partners, the Project Coordinator may also
convene an Extraordinary meeting of the Steering Committee.
%
All project documentation (whether managerial, legal or
technical) will be maintained through a centralised electronic
repository, accessible to all consortium members on an open
basis, and incorporating audit trails concisely recording
reasons for changes etc.  We propose to use either GIT or SVN,
which provide suitable low-cost,  low-overhead solutions that all partners are
familiar with.  Our technical reports will form the basis for
the public deliverables that appear on the project web site.

\begin{figure}[t!]
%\begin{wrapfigure}{l}{0.7\textwidth}
%\vspace{-0.75in}
%\hspace{-1in}
\begin{center}
\centerline{\hspace{1in} 
\hbox to \columnwidth{\hss\includeimage[height=5.7in,trim={0 0 0 2.5cm},clip]{Management}\hss}
}
\end{center}
\vspace{-1.8in}
\caption{Management Structure}
\label{fig:management}
\end{figure}
%\end{wrapfigure}

%\TODO{Update this to reflect the consortium.}

\subsubsection*{Technical Steering Committee}
\vspace{-6pt}

The \emph{Technical Steering Committee} comprises the WTLs, plus the
Project Coordinator (who will act as chair).  Its purpose is to ensure
the effective running of the project on a day-to-day basis, and to
coordinate work across work packages.  In particular the Technical Steering
Committee will be
responsible for the implementation of the directives of the Project Steering
Committee, for the
guidance and monitoring of the technical WPs, for coordination among
WPs, for  the timely preparation, approval, and forwarding to the Commission
of the deliverables produced by the WPs, and for the resolution of conflicts
amongst WPs.  It will meet on a regular basis, usually through a monthly
teleconference.  Meetings may also be convened on request by any member.
% but also in person where necessary.  
Each member of the Technical Steering Committee has one vote,
which may be made by proxy, or in absentia, if necessary.  
Decisions are taken by consensus, if possible, otherwise by majority vote, 
with the
Project Coordinator retaining the casting vote.

\subsubsection*{Project Steering Committee}
\vspace{-6pt}

The \emph{Project Steering Committee} comprises one representative from each partner
(usually the PI), and is chaired by the Project Coordinator.  
The purpose of this committee is to decide
the general technical direction of the project.  It will also
take major decisions on project finances, addition of partners, removal of non-performing
partners, IPR issues, reallocation of workload etc.  It will meet in person
at least once per year, supplemented by more regular teleconference
meetings as required. Extraordinary meetings may also be convened on request by any partner.
Each representative has one vote, which
may be made by proxy if necessary.  Decisions are taken by
consensus, if possible, otherwise by majority vote, with the
Project Coordinator retaining the casting vote.

\subsubsection*{Project Advisory Board}
\vspace{-6pt}

The Project Advisory Board comprises a small group of invited
academics and industrialists who will provide input to the
project on general technical trends and directions, and advise
the steering committee where required.  The initial composition
of the Advisory Board will be determined at the outset of the
project, but we expect to include academic experts from the data-intensive, high-performance and cloud computing, as well as machine learning, compilation, software-defined infrastructures and optimisation domains. We also expect to include senior representatives from the automotive, AI and IoT industry domains. The Coordinator is authorized to 
execute with each member of the EEAB a non-disclosure agreement, which 
terms shall be not less stringent than those stipulated in this 
Consortium Agreement, no later than 30 calendar days after their nomination 
or before any confidential information will be exchanged, whichever date is earlier. 
We have invited senior representatives from Aarhus University, SAP Institute for Digital Government, Ericsson, TypeSafe, British Telecom,
the oil\&gas industries, the Cloud Competency Centre (Dublin),
	and Scottish Enterprise.

\pagebreak
\subsubsection*{Work Package Team Leaders}
\vspace{-6pt}

Work package team leaders (WTLs) are responsible for tracking progress within their work package,
developing metrics for each deliverable at the outset of each
task, ensuring that results are properly reviewed against these
metrics, and consequently providing feedback to the Project Coordinator on the achievement of goals. 
%
WTLs have been chosen on the basis of managerial experience, technical expertise and
commitment to the work package programmes. % , as shown below.

%\begin{figure}[t]
\begin{center}
\begin{tabular}{cc}
\begin{tabular}{|l|l|}\hline
\textbf{WP} & \textbf{WTL}  \\ \hline
WP1 &  Juliana Bowles (\coordshort{}) \\
\hline
\end{tabular}
\end{tabular}
\end{center}
%\caption{Work package team leaders (WTLs).}
%\label{fig:wtls}
%\end{figure}


%\pagebreak
\subsubsection*{Principal Investigators}
\vspace{-6pt}

One principal investigator (PI) will be nominated by each partner.
The PI is responsible for properly managing the budget allocated to the partner and for performing
all the tasks that are carried out by that partner, reporting to the appropriate WTLs where necessary.  PIs
also act as line managers for the researchers/developers employed on the project by the partner.
PIs will usually also act as WTLs for the main WPs that are carried out at that site, and may be allocated their own
technical tasks. They will normally be the partner's representative on the steering committee.
They have been chosen for their technical expertise and experience of line management and budget handling.

\newcounter{partic}

\begin{center}
\begin{tabular}{cc}
\begin{tabular}{|l|l|l|}\hline
& \textbf{Partner} & \textbf{PI} \\ \hline
\addtocounter{partic}{1}
\thepartic & \participantshort{\thepartic} &  Juliana Bowles \\\hline
\end{tabular}
\quad\quad&\quad\quad
\begin{tabular}{|l|l|l|}\hline
& \textbf{Partner} & \textbf{PI} \\ \hline
\addtocounter{partic}{1}
\thepartic & \participantshort{\thepartic} & XX \\\hline
\end{tabular}
\end{tabular}
\end{center}

\vspace{12pt}
\subsubsection*{Project Coordinator}
\vspace{-6pt}

The Project Coordinator is \emph{Dr Juliana Bowles}.  Her role
is to act as the primary point of contact with the European
Commission, to receive feedback on research results from each
work package, to ensure the project maintains effective
progress towards the project objectives based on these results,
to produce any required  project management reports, to ensure
that deliverables are produced according to the planned
schedule and delivered to the Commission and project reviewers
as required, and to resolve disputes between project partners
as and when these arise.  He will convene regular management
and technical meetings, monitor progress on each work package,
collate deliverables, and maintain good contact with each site,
in addition to producing the annual management reports, and
ensuring that each site produces the required financial (audit)
certificates.  He will also be responsible for ensuring that
the Consortium Agreement (including IPR issues, voting rules and the conflict resolution procedures)
and any other legal documents are properly prepared and managed. This will be
done through \SAshort{} \emph{Research and Enterprise
Services}, who have significant expertise in preparing such
agreements.

\vspace{12pt}
\subsubsection*{Project Administrator}
\vspace{-6pt}

The Project Coordinator will be supported in his management
duties by a part-time \emph{Project Administrator} (to be appointed
from staff already employed by \SA{}) and located at \SA{}.  The Administrator
must possess both strong organisational skills and a
sufficient level of technical expertise in order to communicate
management requirements to the Partners, but will not be
involved in the management of day-to-day RTD activities.

\vspace{12pt}
\subsubsection*{Consortium agreement}
\vspace{-6pt}

% The relationship between all partners will be fixed in a Consortium Agreement based on the following principles:

% In order to have a management system applicable through all phases of the
% project, a reasonable approach is to have straight, clear and direct
% management and organization protocol at all levels. This is particularly
% relevant given the challenging financial and industrial policy
% constraints. Therefore, in order to have clearly assigned
% responsibilities, to avoid any friction and to progress as per the
% project plan, the responsibilities and authorities of the project manager
% and the team members will be unambiguous.

The partners will be bound by a formal consortium agreement that is
planned to be signed prior to the beginning of the project of the project,
and in which their roles, responsibilities and mutual obligations will be
defined both for the project life and, where relevant, beyond.  This will
formalise key issues including conflict resolution, IPR procedures, governance structure
etc.  %It will be based on the model consortium agreement issued by the European Commission.
The Digital Europe version of the DESCA, including the European Commission's
inputs will constitute the basis for such consortium agreement.

%\pagebreak
% \vspace{-6pt}
\subsubsection*{Conflict Resolution}
\label{conflict-resolution}
If conflicts arise during the execution of the project, they will be resolved according to the following principles:
% 
% \begin{itemize}
% \item
They will first be addressed within the relevant WP through discussion chaired by the WTL;
% \item
If this fails, the issue will be presented by the WTL either to the Technical Steering Committee
or to the Project Steering Committee, depending on the nature of the problem (technical or business/strategic).
% \item
The relevant board will attempt to resolve the issue through the usual voting procedure.
% \end{itemize}
%
Technical issues between WPs will also be addressed by the Technical Steering Committee.
%  As noted above, the TSC of the project consists of the WP leaders (chaired by the technical coordinator), and the GA consists of representatives of each partner (chaired by the management coordinator).
Any conflicts that cannot be resolved through the principles above will
be handled according to the dispute resolution provision set forth in the
Consortium Agreement.

% \subsubsection*{Management Costs}

% We have budgeted for the Project Administrator at 25\% effort
% over the lifetime of the project (i.e. 9 person-months) at
% \SAshort{}, plus small-scale management effort as required by each support to support the
% project through the preparation of reports for review meetings and other project-level
% management tasks.
% We have also budgeted for the cost of running annual Advisory Committee meetings, including travel
% support for unfunded Advisory Committee members, at \euros{} XX p.a. for a total cost of \euros{} XX.
% We have budgeted an additional
% \euros{}~4,800 for travel by the Project Coordinator (estimated
% as an additional two trips per annum).
% In order to minimise costs and time expenditure, as far as
% possible, all project management activities will be carried out using
% low-cost means such as email, Skype, telephone or
% video-conferencing, and any managerial travel will normally be
% combined with technical or research meetings.  Each site with expenditure
% in excess of \euros{}~325,000 also requires specific costs
% to cover the preparation of the required financial certificates, which will
% generally involve subcontracting an external financial auditor.

\draftpage
\subsection{Consortium as a Whole}
\eucommentary{\begin{itemize}
\item
Describe the consortium. How will it match the project's objectives? How do the members complement one another (and cover the value chain, where appropriate)? In what way does each of them contribute to the project? How will they be able to work effectively together?
\item
If applicable, describe the industrial/commercial involvement in the project to ensure exploitation of the results and explain why this is consistent with and will help to achieve the specific measures which are proposed for exploitation of the results of the project (see section 2.3).
\item
Other countries: If one or more of the participants requesting EU funding is based in a country that is not automatically eligible for such funding (entities from Member States of the EU, from Associated Countries and from one of the countries in the exhaustive list included in General Annex A of the work programme are automatically eligible for EU funding), explain why the participation of the entity in question is essential to carrying out the project
\end{itemize}
}

\begin{figure}[t]
\begin{center}
%\includeimage[scale=0.6,angle=0]{BlessConsortium.pdf}
\end{center}
\vspace{-0.3in}
\caption{Areas of Partner Expertise}
\label{fig:consortium}
\end{figure}

\textbf{Figure~\ref{fig:consortium}} shows the areas of expertise that
are relevant to this project and the consortium partners that
possess that expertise.  All partners span multiple areas,
providing technical depth within the consortium, and avoiding
knowledge gaps.  Within the areas, each partner possesses
complementary expertise, but with enough knowledge overlap to
ensure tight cohesion of the consortium. The consortium
comprises both academic and industrial expertise.
%  from XX
% highly-respected partners.
The consortium links the world-leading technical expertise of
the participating groups on XX.


\paragraph*{Integration of the Consortium.}
%\vspace{-6pt}

Several of the partners already have close working relationships through
recent and ongoing research projects (e.g. \rephrase).
The teams share common technical interests and several are active members in e.g. the
HiPEAC network of excellence (\SAshort{};).
Work Packages have been designed to foster close collaboration
between teams at different organisations, with multiple groups involved in all of
the technical and evaluation work packages. The tasks in 
each work package have been allocated on the basis of
technical expertise and ability. All tasks have been designed to involve multi-site
collaboration and/or the exchange of information, which is
intended to promote healthy interaction between the partners.
Finally, in order to ensure good integration between the partners,
we propose to run at least one technical workshop each year, and
have also requested funds to allow researchers from each group
to visit other groups on a regular basis.  We anticipate that
all of the \TheProject{} researchers will participate in these
technical workshops and exchanges.  We also intend to publish a significant
number of research papers and technical reports deriving from our joint research,
and to collaborate on joint tool production. 
We thus foresee a necessary and close level of integration between
the \TheProject{} partners.

\paragraph{Industrial/Commercial Involvement.}

\TheProject{} directly involves one large company (\IBMshort{}), 
%% sloppy -- KH
XX SMEs.
These organisations have included draft exploitation plans that will directly use the results of the project
as part of their ongoing business strategies and commercial development.
% follows an industrially-inspired agenda and addresses a key and topical challenge in
The \TheProject{} project further engages directly with industry through its
dissemination, user community and outreach activities 
% a dedicated  workpackage (WP7) 
% that are aimed at promoting
will promote the
\TheProject{} tools, technologies and above all \emph{mindset} and \emph{methodology} to a wider user base,
especially through industry-focused events. % that should attract C/C++ programmers and Simulink/SCADE users.
The objective is to ensure widespread uptake of the project results in a broad base of potential industrial
software developers targeting a variety of commercially important domains. % including telecommunications, 3D modelling and the automotive sector.
This will be assisted by including major industry participants on its Advisory Committee,
by actively engaging with the C, C++communities, and by engaging with the ISO C++ Standard Committee and ITU FG-DPM.
% and with the upcoming C++17 design.
% and by actively engaging with new coding standards for parallel and data-intensive applications, as well as C++.
% Finally, members of the consortium are also active members of the ISO C++ Standards Committee. This will
% ensure that the techniques, approach and methodology developed in the \TheProject{} will
% have a broad exposure through the C++ community, and through the evolution of the C++ language
% design itself.

% \khcomment{More needed here.}

\paragraph{Project Management Expertise.}

Members of the team have been heavily involved in running
various national and international research projects.  The
Project Coordinator, Dr Juliana Bowles, has obtained numerous
research grants and awards from national and international
bodies...

As described in the partner descriptions below,
%% Should add activities for INRIA, CodePlay etc.
most of the other partners have been extensively involved in previous and
ongoing EU projects at both technical and managerial
levels, and have dedicated experienced senior staff on the \TheProject{} project.
%XX
%
This experience will be called on as necessary to resolve any
managerial problems that may arise during the course of the
project.


%\bigskip
%\bigskip
\subsection{Resources to be Committed}

\eucommentary{Please provide the following:
\begin{itemize}
\item
a table showing number of person/months required (table 3.4a)
\item
a table showing 'other direct costs' (table 3.4b) for participants where those costs exceed 15\% of the personnel costs (according to the budget table in section 3 of the administrative proposal forms)
\end{itemize}}

%\newpage

%\landscape

\fbox{\begin{minipage}{\textwidth}

\begin{center}\Large\bf
Summary of staff effort
\end{center}
\end{minipage}}


\bigskip

\TODO{Update this once the WPs are finalised.}

\newcommand{\wpleader}{\textbf}

%\noindent\makebox[\textwidth][c]{% 
\begin{center}
\begin{minipage}{14cm}
\begin{tabular}{| p{0.9cm} | p{1.5cm} | c | c | c | c | c | c | c | c | c |}  \hline
\textbf{Partic.} & \textbf{Partic.} 
& \multicolumn{8}{c|}{\textbf{Work package}} &
 \textbf{Total} \\
\textbf{no.} & \textbf{short} & WP1 & WP2 & WP3 & WP4& WP5 & WP6 & WP7 & WP8 & 
 \textbf{PMs} \\
 & \textbf{name} &
 &   &  &   &  &  &   &  &
 \\
\hline

\textbf{1} & \shortparticipant{1} & 
%\wpleader{9} & 2 &  15 & \wpleader{24} &   & 3 &  &   & 53  &
\wpleader{20} & 8 & 2  & \wpleader{17} & 1 & 10 & 3 & 5 & \textbf{66}
\\\hline

\textbf{2} & \shortparticipant{2} &
1 & 6 & \wpleader{26} & 0 & 5.5 & 4 & 2.5 & \wpleader{8} & \textbf{54}
 \\\hline

\textbf{3} & \shortparticipant{3} &
1 & \wpleader{28} & 2 & 5 & 0 & 3 & 16 & 5 & \textbf{54}
 \\\hline

\textbf{4} & \shortparticipant{4} &
1 & 5 & 7 & 2.5 & 2.5 & 10 & \wpleader{37} & 7 & \textbf{72}
 \\\hline

\textbf{5} & \shortparticipant{5} &
1 & 9 & 18 & 12 & 4 & 1 & 3 & 5 & \textbf{53}
 \\\hline

\textbf{6} & \shortparticipant{6} &
1 & 2 & 0 & 0 & \textbf{29} & 5 & 3 & 5 & \textbf{44}
 \\\hline
\textbf{7} & \shortparticipant{7} &
1 & 3 & 10 & 14 & 0 & \wpleader{32} & 3 & 5 & \textbf{68}
 \\\hline

\textbf{8} & \shortparticipant{8} &
1 & 0 & 0 & 0 & 6 & 5 & \textbf{27} & 5 & \textbf{44}
 \\\hline

\textbf{9} & \shortparticipant{9} &
1 & 28 & 2 & 6 & 0 & 3 & 3 & 5 & \textbf{48}
 \\\hline

\multicolumn{2}{|c|}{\textbf{Total PM}} & 
\textbf{28} & \textbf{89} & \textbf{67} & \textbf{56.5} & \textbf{48} & \textbf{73} & \textbf{97.5} & \textbf{50} & \textbf{501}

\\\hline
\end{tabular}
\end{minipage}
\end{center}

\endinput

\fbox{\begin{minipage}{\textwidth}

\begin{center}\Large\bf
Other direct cost items
\end{center}
\end{minipage}}

\bigskip

\begin{tabular}{|r|l|p{9cm}|}
\hline
\textbf{} & \textbf{Cost (\euros)} & \textbf{Justification} \\\hline
\textbf{Travel} & & \\\hline
\textbf{Equipment} & & \\\hline
% \textbf{Other goods and services} & & \\\hline
\textbf{Total} & \\\cline{1-2}
\end{tabular}


% \begin{figure}[ht!]
% %\vspace{-0.5in}
% \begin{center}
% \centerline{\hspace{1in} \hbox to \columnwidth{\hss%\includeimage[height=11.5cm]{PM-partner}
% \hss}}
% \end{center}
% %\vspace{-2.25in}
% %\caption{Effort per Partner}
% \label{fig:effort-partner}
% \end{figure}



% \begin{figure}[ht!]
% %\vspace{-0.5in}
% \begin{center}
% \centerline{\hspace{1in} \hbox to \columnwidth{\hss%\includeimage[height=11.5cm]{PM-WP}
% \hss}}
% \end{center}
% %\vspace{-2.25in}
% %\caption{Effort per Workpackage}
% \label{fig:effort-wp}
% \end{figure}


%\endlandscape



%\pagebreak
\subsubsection{Management Level Description of Resources and Budget}
\vspace{-6pt}

\TODO{This needs to be updated in line with the rest of the
project.}

The project will employ XX person-months of effort over three
years, comprising one or more full-time or part-time researchers at each site
plus one part-time project administrator at \SAshort{}, % and one part-time web designer at \INRIAshort,  Not in PMs?
20\% of the Project Coordinator and 10\% of the WTLs. The researchers will be supported by
the necessary dedicated computing equipment,
% the
% usual basic research equipment (workstations and/or laptop
% computers, network facilities, printers, dedicated file
% servers, etc.) funded from the project overheads, by
% special-purpose heterogeneous hardware necessary to carry out
% the research, 
by funding to enable the necessary travel to scientific and technical conferences, trade shows, 
project meetings and other project-related events, and by the funding that is needed to establish/enhance existing
industrial and academic contacts and to establish a user
community for the \TheProject{} tools and technologies.
%
The quoted budget includes all relevant national social and
other legitimate employment costs as permitted under the rules
governing EU Horizon 2020 ICT projects, including costs of
healthcare, social security and pensions provision, in line with national norms for each site.
%
Sufficient travel funding is also needed to support good
collaboration between the groups, including attendance at the
annual technical project meetings, plus individual visits between
sites. We have budgeted approximately \euros{}~8,000 per
site per year (varied in line with previous
costs at each site)  to cover, for example:
% \begin{itemize}
% \item 
attending two project workshops at \euros{}~600 each;
% \item
attending the annual Project Review Meeting at \euros{}~600;
% \item
one 1-week inter-site visit at \euros{}~750 each;
% \item
attending two conferences per year within the EU at \euros{}~1,000 each;
% \item
attending one conference per year outside the EU at \euros{}~1,500;
%\item
three conference fees per year at \euros{}~650 each.
% \end{itemize}
%
\noindent
In addition, \SAshort{} has budgeted \euros{}~1,500 per year to cover attendance at
IFIP Working Group meetings, visits to industrial concerns for dissemination purposes,
attendance at developer conferences, demonstrations etc. to promote the project,
and \euros~1,000 per year to support travel that is related to the management of the
project.
Wherever possible, travel for different purposes will be
combined into a single trip. We have also budgeted \euros~1,000
per year at \SAshort{} to support attendance by the Project Advisory Board members at
the annual Advisory Board Meetings, where this cannot be met from other sources.

\TODO{Add any specialist equipment.  We might add a serious
  multi-core/multi-GPU machine.}

%\pagebreak
\subsubsection{Additional Partner Costs}
\vspace{-6pt}

\paragraph{Testbed systems and development servers.}
\SAshort{} has budgeted \euros{} XX for a central
server to support the various software and document repositories that are needed by the project, to run the project website and to provide access
to the shared research data that will be generated by the project.

\paragraph{Open Access Publication Fees.}
Each academic partner has budgeted approximately \euros{5,000} to
support gold open access publication for key project publications
(representing about 10-20\% of the expected project output).  The
budget will be pooled if not used by a specific partner, and used to
support further gold open access publication by other partners or
other dissemination activities, as necessary to maximise the overall
success of the project.  There will be no charge for green open access
publication, which will be used for the remainder of the project
publications.

%\vspace{-12pt}
\label{bibliography}
\addcontentsline{toc}{section}{References}

\bibliographystyle{abbrv}
\bibliography{bibliography}


%% Write macro to split Sections 1-3
\Split{1-3}

% ---------------------------------------------------------------------------
%  Section 4: Members of the Consortium
% ---------------------------------------------------------------------------

\newpage

\eucommentary{Page limits do not apply.}

\section{Members of the Consortium}

\eucommentary{Please provide, for each participant, the following (if available):\\
\begin{itemize}
\item
a description of the legal entity and its main tasks, with an explanation of how its profile matches the tasks in the proposal;
\item
a curriculum vitae or description of the profile of the persons, including their gender, who will be primarily responsible for carrying out the proposed research and/or innovation activities;
\item
a list of up to 5 relevant publications, and/or products, services (including widely-used datasets or software), or other achievements relevant to the call content;
\item
a list of up to 5 relevant previous projects or activities, connected to the subject of this proposal;
\item
a description of any significant infrastructure and/or any major items of technical equipment, relevant to the proposed work;
\item
[any other supporting documents specified in the work programme for this call.]
\end{itemize}}

\subsection{Participants}
\Participant{SA}{(\url{http://www.st-andrews.ac.uk})}

\begin{wrapfigure}{R}{2cm}
\vspace{-3.95cm}
\hfill \includeimage{logos/st-andrews-logo.jpg}
\vspace{-1cm}
\end{wrapfigure}

\label{sec:participantUSTAN}

%===============================================================================
The \SAlong{} is the third-oldest in the English-speaking world (founded 1413).
The School of Computer Science was likewise one of the earliest Computer Science departments in the world (founded 1972).
It has established an excellent reputation for its pioneering research in e.g.,
parallel computing, software engineering, programming language design,
software architectures, theoretical computer science and
distributed/mobile systems.  This research expertise has been
recognised through the award of numerous research grants and
awards from the UK and the European Commission.

\vspace{10pt}
\textbf{The \SAlong{} coordinates the \TheProject{} project and
leads work packages WPXX and participates in WPXX on XX.}
\vspace{10pt}

\paragraph{Dr Juliana Bowles} \url{http://www.}


\subsubsection*{Relevant publications}
\begin{itemize}
\item XX
\end{itemize}

\pagebreak
\subsubsection*{Relevant Research Projects}

\begin{itemize}
% \item
% Automatic Prediction of Resource Bounds for Embedded Systems (EmBounded, IST-2004-510255, 2005-2008, \url{http://www.embounded.org});
\item XX
\end{itemize}

\Participant{IBM}{(\url{http://www.research.ibm.com/labs/haifa/)}}


\begin{wrapfigure}{R}{4cm}
\vspace{-2cm}
\hfill \includeimage[width=4cm]{logos/ibm.jpg}
\vspace{-1cm}
\end{wrapfigure}

\ 

%\vspace{24pt}
For more than sixty years, IBM Research, as the world's largest IT research organisation has been the innovation engine of the IBM corporation. Since the beginning of 2000, IBM has spent \$75 billion in R\&D, enabling IBM to deliver key innovations and maintain U.S. patent leadership for the 21st consecutive
year in 2013.
IBM also participates in and contributes to the work of standards consortia, alliances, and formal national and international standards organisations. 
The IBM Haifa Research Lab is IBM's largest research laboratory outside of the United States, 
employing almost 500 researchers, the majority of whom hold doctorate and master degrees in computer science, electrical engineering, mathematics, and related fields. Since its founding in 1972, HRL has conducted world class research vital to IBM's success. R\&D projects are being executed today in areas such as Cognitive computing, Healthcare and Life Sciences, Verification Technologies, Telco, Machine Learning, Cloud Computing, Multimedia, Active Management, Information Retrieval, Programming Environments and Information and Cyber Security. The Quality and Security Department of the IBM Research Haifa includes researchers in the fields of cyber security and privacy. As a multi-disciplinary research area, researchers come from different research domains including verification, data analytic, operating systems and runtime systems, languages and compilers, network systems, and protocols and cloud technologies.
HRL has a long history of successful participation in EU projects. A partial list includes the following: SMESEC (H2020), REPHRASE (H2020), SHARCS (H2020), PINCETTE (FP7, coordinator), RESERVOIR (FP7, coordinator), CloudWave (FP7, coordinator), SHADOWS (FP6, coordinator), CASPAR (FP6), HiPEAC (FP6 + FP7), SARC (FP6), ACOTES (FP7), MilePost (FP7), HYPERGENES (FP7), HERMES (FP7), SAPIR(FP6, coordinator), PROSYD (FP6, coordinator), Modelplex (FP6, coordinator).
The Quality and Security Department develops advanced tools and technologies spanning the entire spectrum of Functional Verification, Code Analysis and Cyber Security.

As a global leader in IT security, IBM offers the strategies, capabilities, and technologies necessary to help organizations in the private and public sectors preemptively protect the organisation from threats and address the complexities and growing costs of security risk management and compliance. IBM is helping to solve essential security challenges including:
\begin{itemize}
\item
  Better secure data and protect privacy
\item
	Control network access and help assure resilience
\item
	Defend mobile and social workplace
\item
	Manage third-party security compliance
\item
	Address new complexity of cloud and virtualization
\item
	Build a risk-aware culture
\end{itemize} 	
To facilitate a comprehensive offering IBM is continuously investing in emerging technologies in the area of security intelligence and has a wide variety of security products and services. IBM is recognized in the industry as a leader in IT cyber security.
In recent years, IBM has acquired several cyber security start-ups in Israel increasing its R\&D presence in the region. IBM has recently also announced the establishment of a Cyber Center of Excellence (CCoE) in Beer-Sheva, Israel. The IBM Haifa Research Lab, as a well recognized IBM research facility and the largest one outside of the US, is collaborating with European research facilities to support the buildup of the CCoE as well as the acquired cyber security start-ups.

\vspace{10pt}
\textbf{IBM leads work package WPXX}

\vspace{10pt}

\paragraph{Michael Vinov}     

\subsubsection*{List of Publications}


%Formal verification:
\begin{itemize}
\item XX
\end{itemize}

\subsubsection*{Relevant Research Projects}
\begin{itemize}
\item
Validating Changes and Upgrades in Embedded Software (PINCETTE, ICT-257647);
\item
CloudWave: Agile Service Engineering for the Future Internet (CloudWave, ICT-610802);
\item
Refactoring Parallel Heterogeneous Resource-Aware Applications --- a Software Engineering Approach (RePhrase, ICT-644235);
\item
Secure Hardware-Software Architectures for Robust Computing Systems (SHARCS, ICT-322014).

\end{itemize}

\Participant{SCCH}{(http://www.scch.at)}

\begin{wrapfigure}{R}{4cm}
\vspace{-2cm}
\hfill \includeimage[width=4cm]{logos/SCCH.jpg}
\vspace{-1cm}
\end{wrapfigure}

\SCCH{} (\url{www.scch.at}) is an Austrian research and technology organisation, funded in 1999 by several institutes of Johannes Kepler University, Linz. Its primary focus is on applied research in the fields of software and data science. 
In projects with partner companies state-of-the-art research results are applied to practical industrial projects to increase and maintain their competitiveness. One of the focus areas is Data Analysis Systems (DAS), specialising on the advancement and application of methods for the analysis and modelling of complex and massive sensor data in its (industrial) application context. Particular application domains include
\begin{inparaenum}[a)]
\item modelling, prognosis, forecast and control of systems
\item industrial fault detection, diagnosis and prognosis
\item the discovery of knowledge and structure in industrial processes.
\end{inparaenum}
\SCCHshort{} is or was involved in the EU-funded projects H2020 ALOHA (Grant Agreement 780788), TRESSPASS (SEC-2016-2017-2, Proposal Nr. 787120) RePhrase (ICT-644235), ParaPhrase (IST-2011-288570), ADVANCE (IST-2010-248828), SECO, and FACETS, where in ALOHA and TRESSPASS SCCH contributes with its expertise in deep learning and the latter are concerned with parallel computation.  The participating investigators are project managers or researchers for relevant applied research projects in the fields of machine learning, parallelisation, and scheduling, and are active in the research community by publishing and reviewing for journals and conferences.

\vspace{10pt}
\textbf{\SCCHshort{} leads WPXX}
\vspace{10pt}

\subsubsection*{List of Publications}

\begin{itemize}

\item XX
  
\end{itemize}

\subsubsection*{Relevant Research Projects}

\begin{itemize}

\item RePhrase (H2020, ICT-644235) - Refactoring Parallel Heterogeneous Resource-Aware Applications. SCCH's role: reinforcement learning based dynamic scheduling and industrial use cases

\item ParaPhrase (IST-2011-288570) - Parallel Patterns for Adaptive Heterogeneous Multicore Systems. SCCH's role: use cases in the area of machine learning

\item ADVANCE (IST-2010-248828) - Asynchronous and Dynamic Virtualisation through performance ANalysis to support Concurrency Engineering, \url{http://www.project-advance.eu});

\item ALOHA (H2020 Grant Agreement 780788) - Software framework for runtime-Adaptive and secure deep Learning On Heterogeneous Architectures. SCCH's role: develop deep transfer learning based methods for surveillance applications

\item TRESSPASS (H2020 SEC-2016-2017-2, Proposal Nr. 787120) - robusT Risk basEd Screening and alert System for PASSengers and luggage. SCCH's role: develop deep learning based methods for security applications

\end{itemize}

% ============================

\Participant{SOPRA}{(\url{https://www.soprasteria.co.uk/)}}


\begin{wrapfigure}{R}{6cm}
\vspace{-2cm}
\hfill \includeimage[width=6cm]{logos/Sopra-Steria-logo2.png}
\vspace{-1cm}
\end{wrapfigure}

\ 

%\vspace{24pt}

Sopra-Steria, since 1968, supports the primary business areas of consulting services, systems integration, integration of ERP, implementation of applications, as well as providing technical support to users and application maintenance and outsourcing services and operation of professional processes.

Has experience in:

\begin{itemize}
    \item Cyber Security via: (\url{https://www.soprasteria.com/services/cybersecurity})
    \item Artificial Intelligence via:\\ (\url{https://www.soprasteria.com/services/technology-services/artificial-intelligence})
    \item Internet of Things (IoT) via:\\ (\url{https://www.soprasteria.com/services/technology-services/internet-of-things})
\end{itemize}

Supports following industries:
\begin{itemize}
    \item Aerospace
    \item Defense and Security
    \item Energy Utilities
    \item Financial Services
    \item Insurance and Social
    \item Government
    \item Retail
    \item Telecommunication, Media and Entertainment
    \item Transport
\end{itemize}



\vspace{10pt}
\textbf{Sopra-Steria leads work package WPXX}

\vspace{10pt}

\paragraph{Andreas Francois Vermeulen - Head of Analytics, Digital Consultancy Services within the United-Kingdom.
}     

\subsubsection*{List of Publications}

\begin{itemize}
\item The SERUMS tool-chain: ensuring security and privacy of medical data in smart patient-centric healthcare systems. (IEEE Big Data), Los Angeles, December 2019, IEEE Press. DOI: 10.1109/BigData47090.2019.9005600
\end{itemize}

\subsubsection*{Relevant Research Projects}
\begin{itemize}
\item SERUMS (\url{https://www.serums-h2020.org/})
\end{itemize}

% ============================


\subsection{Third parties involved in the project (including use of third party resources)}

No third parties are involved in the project.

% ---------------------------------------------------------------------------
%  Section 5: Ethics and Security
% ---------------------------------------------------------------------------

\newpage

\section{Ethics and Security}

\subsection{Ethics}

The proposal raises no specific ethical concerns.


\subsection{Security}

Please indicate if your proposal involves:

\begin{itemize}
\item
activities or results raising security issues: NO
\item
'EU-classified information' as background or results: NO
\end{itemize}

%% Write macro to split Sections 4-5
\Split{4-5}

%% Finalise batch file
\immediate\write\BatchFile{exit}% 
\immediate\closeout\BatchFile% 

\newpage

\label{bibliography}
\addcontentsline{toc}{section}{References}

%\bibliographystyle{abbrv}
%\bibliography{bibliography_ustan}
%\bibliography{bibliography_scch}

\end{document}
